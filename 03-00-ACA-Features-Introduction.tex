% move all configuration stuff into includes file so we can focus on the content
\documentclass[aspectratio=169,hyperref={pdfpagelabels=false,colorlinks=true,linkcolor=white,urlcolor=blue},t]{beamer}

%%%%%%%%%%%%%%%%%%%%%%%%%%%%%%%%%%%%%%%%%%%%%%%%%%%%%%%%%%%%%%%%%%%%%%%%%%%%%%%%%%
%%%%%%%%%%%%%%%%%%%%%%%%%%%%%%%%%%%%%%%%%%%%%%%%%%%%%%%%%%%%%%%%%%%%%%%%%%%%%%%%%%
% packages
\usepackage{pict2e}
\usepackage{epic}
\usepackage{amsmath,amsfonts,amssymb}
\usepackage{units}
\usepackage{fancybox}
\usepackage[absolute,overlay]{textpos} 
\usepackage{media9} % avi2flv: "C:\Program Files\ffmpeg\bin\ffmpeg.exe" -i TuneFreqFilterbank.avi -b 600k -s 441x324 -r 15 -acodec copy TuneFreqFilterbank.flv
\usepackage{animate}
\usepackage{gensymb}
\usepackage{multirow}
\usepackage{silence}
\usepackage{tikz}
\usepackage[backend=bibtex,style=ieee]{biblatex}
\AtEveryCitekey{\iffootnote{\tiny}{}}
\addbibresource{include/references}

%%%%%%%%%%%%%%%%%%%%%%%%%%%%%%%%%%%%%%%%%%%%%%%%%%%%%%%%%%%%%%%%%%%%%%%%%%%%%%%%%%
%%%%%%%%%%%%%%%%%%%%%%%%%%%%%%%%%%%%%%%%%%%%%%%%%%%%%%%%%%%%%%%%%%%%%%%%%%%%%%%%%%
% relative paths
\graphicspath{{graph/}}


%%%%%%%%%%%%%%%%%%%%%%%%%%%%%%%%%%%%%%%%%%%%%%%%%%%%%%%%%%%%%%%%%%%%%%%%%%%%%%%%%%
%%%%%%%%%%%%%%%%%%%%%%%%%%%%%%%%%%%%%%%%%%%%%%%%%%%%%%%%%%%%%%%%%%%%%%%%%%%%%%%%%%
% units
\setlength{\unitlength}{1mm}

%%%%%%%%%%%%%%%%%%%%%%%%%%%%%%%%%%%%%%%%%%%%%%%%%%%%%%%%%%%%%%%%%%%%%%%%%%%%%%%%%%
%%%%%%%%%%%%%%%%%%%%%%%%%%%%%%%%%%%%%%%%%%%%%%%%%%%%%%%%%%%%%%%%%%%%%%%%%%%%%%%%%%
% theme & layout
\usetheme{Frankfurt}
\beamertemplatenavigationsymbolsempty
%\setbeamertemplate{frametitle}[smoothbars theme]
\setbeamertemplate{frametitle}
{
    \begin{beamercolorbox}[ht=1.8em,wd=\paperwidth]{frametitle}
        \vspace{-.1em}%
        \hspace{.2em}{\strut\insertframetitle\strut}
        
        \hspace{.2em}\small\strut\insertframesubtitle\strut
        %\hfill
        %\includegraphics[height=.8cm,keepaspectratio]{CenterMusicTechnology-solid-2lines-white-CoAtag}
        
    \end{beamercolorbox}
    \begin{textblock*}{100mm}(11.6cm,.7cm)
        \includegraphics[height=.8cm,keepaspectratio]{logo_GTCMT_black}
    \end{textblock*}
}

% set this to ensure bulletpoints without subsections
\usepackage{remreset}
\makeatletter
\@removefromreset{subsection}{section}
\makeatother
\setcounter{subsection}{1}

%---------------------------------------------------------------------------------
% appearance
\setbeamercolor{structure}{fg=gtgold}
\setbeamercovered{transparent} %invisible
\setbeamercolor{bibliography entry author}{fg=black}
\setbeamercolor*{bibliography entry title}{fg=black}
\setbeamercolor*{bibliography entry note}{fg=black}
\setbeamercolor{frametitle}{fg=black}
\setbeamercolor{title}{fg=black}

%\usepackage{pgfpages}
%\setbeameroption{show notes}
%\setbeameroption{show notes on second screen=right}
%---------------------------------------------------------------------------------
% fontsize
\let\Tiny=\tiny

%%%%%%%%%%%%%%%%%%%%%%%%%%%%%%%%%%%%%%%%%%%%%%%%%%%%%%%%%%%%%%%%%%%%%%%%%%%%%%%%%%
%%%%%%%%%%%%%%%%%%%%%%%%%%%%%%%%%%%%%%%%%%%%%%%%%%%%%%%%%%%%%%%%%%%%%%%%%%%%%%%%%%
% warnings
\pdfsuppresswarningpagegroup=1
\WarningFilter{biblatex}{Patching footnotes failed}
\WarningFilter{latexfont}{Font shape}
\WarningFilter{latexfont}{Some font shapes}
\WarningFilter{gensymb}{Not defining}


%%%%%%%%%%%%%%%%%%%%%%%%%%%%%%%%%%%%%%%%%%%%%%%%%%%%%%%%%%%%%%%%%%%%%%%%%%%%%%%%%%
%%%%%%%%%%%%%%%%%%%%%%%%%%%%%%%%%%%%%%%%%%%%%%%%%%%%%%%%%%%%%%%%%%%%%%%%%%%%%%%%%%
% title information
\title[]{Introduction to \textbf{Audio Content Analysis}}   
\author[alexander lerch]{alexander lerch} 
%\institute{~}
%\date[Alexander Lerch]{}
%\titlegraphic{\vspace{-16mm}\includegraphics[width=\textwidth,height=3cm]{title}}

%%%%%%%%%%%%%%%%%%%%%%%%%%%%%%%%%%%%%%%%%%%%%%%%%%%%%%%%%%%%%%%%%%%%%%%%%%%%%%%%%%
%%%%%%%%%%%%%%%%%%%%%%%%%%%%%%%%%%%%%%%%%%%%%%%%%%%%%%%%%%%%%%%%%%%%%%%%%%%%%%%%%%
% colors
\definecolor{gtgold}{HTML}{96caff} %0e7eed {rgb}{0.88,0.66,1,0.06} [234, 170, 0]/256
\definecolor{darkgray}{rgb}{.1, .1, .25}
\definecolor{lightblue}{HTML}{0e7eed}
\definecolor{highlight}{rgb}{0, 0, 1} %_less!40

%%%%%%%%%%%%%%%%%%%%%%%%%%%%%%%%%%%%%%%%%%%%%%%%%%%%%%%%%%%%%%%%%%%%%%%%%%%%%%%%%%
%%%%%%%%%%%%%%%%%%%%%%%%%%%%%%%%%%%%%%%%%%%%%%%%%%%%%%%%%%%%%%%%%%%%%%%%%%%%%%%%%%
% relative paths
\graphicspath{{../ACA-Plots/graph/}}


%%%%%%%%%%%%%%%%%%%%%%%%%%%%%%%%%%%%%%%%%%%%%%%%%%%%%%%%%%%%%%%%%%%%%%%%%%%%%%%%%%
%%%%%%%%%%%%%%%%%%%%%%%%%%%%%%%%%%%%%%%%%%%%%%%%%%%%%%%%%%%%%%%%%%%%%%%%%%%%%%%%%%
% units
\setlength{\unitlength}{1mm}

%%%%%%%%%%%%%%%%%%%%%%%%%%%%%%%%%%%%%%%%%%%%%%%%%%%%%%%%%%%%%%%%%%%%%%%%%%%%%%%%%%
%%%%%%%%%%%%%%%%%%%%%%%%%%%%%%%%%%%%%%%%%%%%%%%%%%%%%%%%%%%%%%%%%%%%%%%%%%%%%%%%%%
% math
\DeclareMathOperator*{\argmax}{argmax}
\DeclareMathOperator*{\argmin}{argmin}
\DeclareMathOperator*{\atan}{atan}
\DeclareMathOperator*{\arcsinh}{arcsinh}
\DeclareMathOperator*{\sign}{sign}
\DeclareMathOperator*{\tcdf}{tcdf}
\DeclareMathOperator*{\si}{sinc}
\DeclareMathOperator*{\princarg}{princarg}
\DeclareMathOperator*{\arccosh}{arccosh}
\DeclareMathOperator*{\hwr}{HWR}
\DeclareMathOperator*{\flip}{flip}
\DeclareMathOperator*{\sinc}{sinc}
\DeclareMathOperator*{\floor}{floor}
\newcommand{\e}{{e}}
\newcommand{\jom}{\mathrm{j}\omega}
\newcommand{\jOm}{\mathrm{j}\Omega}
\newcommand   {\mat}[1]    		{\boldsymbol{\uppercase{#1}}}		%bold
\renewcommand {\vec}[1]    		{\boldsymbol{\lowercase{#1}}}		%bold

%%%%%%%%%%%%%%%%%%%%%%%%%%%%%%%%%%%%%%%%%%%%%%%%%%%%%%%%%%%%%%%%%%%%%%%%%%%%%%%%%%
%%%%%%%%%%%%%%%%%%%%%%%%%%%%%%%%%%%%%%%%%%%%%%%%%%%%%%%%%%%%%%%%%%%%%%%%%%%%%%%%%%
% media9
\newcommand{\includeaudio}[1]{
\href{run:audio/#1.mp3}{\includegraphics[width=5mm, height=5mm]{graph/SpeakerIcon}}}

\newcommand{\includeanimation}[4]{{\begin{center}
                        \animategraphics[autoplay,loop,scale=.7]{#4}{animation/#1-}{#2}{#3}        
                        \end{center}
                        \addreference{matlab source: \href{https://github.com/alexanderlerch/ACA-Plots/blob/master/matlab/animate#1.m}{matlab/animate#1.m}}}
                        \inserticon{video}}
                        
%%%%%%%%%%%%%%%%%%%%%%%%%%%%%%%%%%%%%%%%%%%%%%%%%%%%%%%%%%%%%%%%%%%%%%%%%%%%%%%%%%
%%%%%%%%%%%%%%%%%%%%%%%%%%%%%%%%%%%%%%%%%%%%%%%%%%%%%%%%%%%%%%%%%%%%%%%%%%%%%%%%%%
% other commands
\newcommand{\question}[1]{%\vspace{-4mm}
                          \setbeamercovered{invisible}
                          \begin{columns}[T]
                            \column{.9\textwidth}
                                \textbf{#1}
                            \column{.1\textwidth}
                                \vspace{-8mm}
                                \begin{flushright}
                                     \includegraphics[width=.9\columnwidth]{graph/question_mark}
                                \end{flushright}
                                \vspace{6mm}
                          \end{columns}\pause\vspace{-12mm}}

\newcommand{\toremember}[1]{
                        \inserticon{lightbulb}
                        }

\newcommand{\matlabexercise}[1]{%\vspace{-4mm}
                          \setbeamercovered{invisible}
                          \begin{columns}[T]
                            \column{.8\textwidth}
                                \textbf{matlab exercise}: #1
                            \column{.2\textwidth}
                                \begin{flushright}
                                     \includegraphics[scale=.5]{graph/logo_matlab}
                                \end{flushright}
                                %\vspace{6mm}
                          \end{columns}}

\newcommand{\addreference}[1]{  
                  
                    \begin{textblock*}{\baselineskip }(.98\paperwidth,.5\textheight) %(1.15\textwidth,.4\textheight)
                         \begin{minipage}[b][.5\paperheight][b]{1cm}%
                            \vfill%
                             \rotatebox{90}{\tiny {#1}}
                        \end{minipage}
                   \end{textblock*}
                    }
                    
\newcommand{\figwithmatlab}[1]{
                    \begin{figure}
                        \centering
                        \includegraphics[scale=.7]{#1}
                        %\label{fig:#1}
                    \end{figure}
                    
                    \addreference{matlab source: \href{https://github.com/alexanderlerch/ACA-Plots/blob/main/matlab/plot#1.m}{plot#1.m}}}
\newcommand{\figwithref}[2]{
                    \begin{figure}
                        \centering
                        \includegraphics[scale=.7]{#1}
                        \label{fig:#1}
                    \end{figure}
                    
                    \addreference{#2}}  
                                    
\newcommand{\inserticon}[1]{
                    \begin{textblock*}{100mm}(14.5cm,7.5cm)
                        \includegraphics[height=.8cm,keepaspectratio]{graph/#1}
                    \end{textblock*}}            

%%%%%%%%%%%%%%%%%%%%%%%%%%%%%%%%%%%%%%%%%%%%%%%%%%%%%%%%%%%%%%%%%%%%%%%%%%%%%%%%%%
%%%%%%%%%%%%%%%%%%%%%%%%%%%%%%%%%%%%%%%%%%%%%%%%%%%%%%%%%%%%%%%%%%%%%%%%%%%%%%%%%%
% counters
\newcounter{i}
\newcounter{j}
\newcounter{iXOffset}
\newcounter{iYOffset}
\newcounter{iXBlockSize}
\newcounter{iYBlockSize}
\newcounter{iYBlockSizeDiv2}
\newcounter{iXBlockSizeDiv2}
\newcounter{iDistance}

\newcommand{\IEEELink}{https://ieeexplore.ieee.org/servlet/opac?bknumber=9965970}



\subtitle{Module 3.0: Feature Extraction~---~Introduction and Pre-processing}

%%%%%%%%%%%%%%%%%%%%%%%%%%%%%%%%%%%%%%%%%%%%%%%%%%%%%%%%%%%%%%%%%%%%%%%%%%%%
\begin{document}
    % generate title page
	

\begin{frame}
    \titlepage
    %\vspace{-5mm}
    \begin{flushright}
        \href{http://www.gtcmt.gatech.edu}{\includegraphics[height=.8cm,keepaspectratio]{logo_GTCMT_black}}
    \end{flushright}
\end{frame}


    \section[overview]{lecture overview}
        \begin{frame}{introduction}{overview}
            \begin{block}{corresponding textbook section}
                    \href{http://ieeexplore.ieee.org/xpl/articleDetails.jsp?arnumber=6331120}{Chapter 3~---~Instantaneous Features}: pp.~31--35
            \end{block}

            \begin{itemize}
                \item   \textbf{lecture content}
                    \begin{itemize}
                        \item   introduction to the concept of features
                        \item   audio pre-processing for feature extraction
                    \end{itemize}
                \bigskip
                \item<2->   \textbf{learning objectives}
                    \begin{itemize}
                        \item   describe the process of feature extraction
                        \item   list possible pre-processing option and explain potential use cases
                    \end{itemize}
            \end{itemize}
            \inserticon{directions}
        \end{frame}

    \section[intro]{introduction}
        \begin{frame}{instantaneous features}{introduction}
            remember the flow chart of a general ACA system:
            \vspace{-3mm}
            \begin{figure}
                \begin{footnotesize}
				\begin{picture}(96,26)
					\setcounter{iXOffset}{0}
					\setcounter{iYOffset}{5}
					\setcounter{iXBlockSize}{28}
					\setcounter{iYBlockSize}{16}
					\setcounter{iYBlockSizeDiv2}{8}
					\setcounter{iDistance}{8}
	
					\addtocounter{iYOffset}{\value{iYBlockSizeDiv2}}
					\addtocounter{iYOffset}{-2}
	
					%\addtocounter{iXOffset}{-1}
					\put(\value{iXOffset}, \value{iYOffset})
						{\text{{\shortstack[c]{audio\\ signal}}}}
					\addtocounter{iXOffset}{1}
	
					\addtocounter{iYOffset}{2}
					\addtocounter{iXOffset}{\value{iDistance}}
	
					\put(\value{iXOffset}, \value{iYOffset})
						{\vector(1,0){\value{iDistance}}}
	
					\addtocounter{iXOffset}{\value{iDistance}}
					\addtocounter{iYOffset}{-\value{iYBlockSizeDiv2}}
					
					\put(\value{iXOffset}, \value{iYOffset})
						{\framebox(\value{iXBlockSize}, \value{iYBlockSize}) {\color{highlight}{\shortstack[c]{feature\\ extraction}}}}
	
					\addtocounter{iXOffset}{\value{iXBlockSize}}
					\addtocounter{iYOffset}{\value{iYBlockSizeDiv2}}
	
					\put(\value{iXOffset}, \value{iYOffset})
						{\vector(1,0){\value{iDistance}}}
	
					\addtocounter{iXOffset}{\value{iDistance}}
					\addtocounter{iYOffset}{-\value{iYBlockSizeDiv2}}
	
					\put(\value{iXOffset}, \value{iYOffset})
						{\framebox(\value{iXBlockSize}, \value{iYBlockSize}) {\shortstack[c]{decision,\\ interpretation,\\ classification,\\ inference}}}
	
					\addtocounter{iXOffset}{\value{iXBlockSize}}
					\addtocounter{iYOffset}{\value{iYBlockSizeDiv2}}
	
					\put(\value{iXOffset}, \value{iYOffset})
						{\vector(1,0){\value{iDistance}}}
	
					\addtocounter{iXOffset}{\value{iDistance}}
					\addtocounter{iYOffset}{-2}
	
					\addtocounter{iXOffset}{1}
					\put(\value{iXOffset}, \value{iYOffset})
						{\text{{\shortstack[c]{meta\\ data}}}}
					
				\end{picture}
\end{footnotesize}

            \end{figure}
            
            \vspace{-2mm}
            \pause
            \textbf{feature}:
            \begin{itemize}
                \item<2->   \textit{terminology}: 
                    \begin{itemize}
                        \item   audio descriptor
                        \item   instantaneous/short-term/\color<3->{gtgold}{low-level feature}
                    \end{itemize}
                \item<2->   \textit{characteristics}:
                    \begin{itemize}
                        \item	not necessarily musically, perceptually, or semantically meaningful
                        \item	low-level: usually one value per block
                    \end{itemize}
            \end{itemize}
        \end{frame}
        
        \begin{frame}{instantaneous features}{feature}
            \toremember{}
            \begin{block}{a feature \ldots}
            \begin{itemize}
                \item   is task-specific, i.e.\ contains information relevant to the task,
                \bigskip
                \item   may be custom-designed, chosen from a set of established features, or learned from data,
                \bigskip
                \item   can be a representation of any data (audio, meta data, other features, ...),
                \bigskip
                \item   is not necessarily musically, perceptually, or semantically meaningful or interpretable
            \end{itemize}
            \end{block}
        \end{frame}
        
        \begin{frame}{instantaneous features}{feature example}
            waveform envelope of three different signals 
            
            \figwithmatlab{Waveforms}
            
            \vspace{-2mm}
            \begin{columns}
                \column{.16\textwidth}
                \column{.25\textwidth}\centering
                    \hspace{8mm}\includeaudio{excerpt_pop}
                \column{.25\textwidth}\centering
                    \includeaudio{excerpt_stringquartet}
                \column{.25\textwidth}\centering
                    \hspace{-10mm}\includeaudio{excerpt_speech}
                \column{.09\textwidth}
            \end{columns}
            
            \bigskip
            \begin{itemize}
                \item<2->   envelopes of waveforms can have distinct shape
                \item<2->[$\Rightarrow$] a feature describing envelope shape could help to distinguish these signal types
            \end{itemize}
            \inserticon{audio}
        \end{frame}

        \begin{frame}{instantaneous features}{feature extraction}
            \vspace{-5mm}
            \begin{columns}
            \column{.6\linewidth}
            \flushright{\includeaudio{sax_example}}
            \vspace{-5mm}
            \begin{figure}
            \vspace{-8mm}
                \includegraphics[height=10mm,width=.7\columnwidth]{waveform}\\ \vspace{-3mm}
                \includegraphics[scale=.08]{FeatureExtraction}
                
            \end{figure}
            \column{.4\linewidth}
                \begin{itemize}
                    \item   repeat for every block
                    \item   repeat for every feature: \textit{Spectral Centroid}, \textit{RMS}, \textit{MFCCs}, \ldots
                    \bigskip
                    \item[$\Rightarrow$] feature matrix per audio input
                \end{itemize}
            \end{columns}
            \inserticon{audio}
        \end{frame}

    \section[pre-proc]{audio pre-processing}
        \begin{frame}{instantaneous features}{audio pre-processing}
            \begin{itemize}
                \item   \textbf{pre-processing}: audio is treated before feature extraction (task dependent)
                \bigskip
                \item   \textbf{possible goals} 
                    \begin{itemize}
                        \item   \textit{reduce amount of data} (e.g., down-sampling)
                        \item   \textit{remove irrelevant information} (e.g., surround channels of multi-channel signal)
                        \item   \textit{remove information that might impact analysis} (e.g., DC offset)
                        \item   \textit{increase robustness} (e.g., normalization)
                    \end{itemize}
            \end{itemize}
        \end{frame}
        \begin{frame}{instantaneous features}{audio pre-processing examples 1/2}
            \begin{itemize}
                \item   \textbf{down-mixing}
                    \[
                        x(i) = \frac{1}{\mathcal{C}}\sum\limits_{c=0}^{\mathcal{C}-1}{x_c(i)} 
                    \]
                    \begin{itemize}
                        \item   \textit{variants}: different channel weights, $\nicefrac{\pi}{2}$ phase shift in one channel, \ldots
                    \end{itemize}
                    
                \bigskip
                \item<2->   \textbf{normalization}
                    \[
                        x(i) = \frac{x_s(i)}{\max\limits_{\forall i}\big(|x_s(i)|\big)} 
                    \]
                    \begin{itemize}
                        \item   \textit{variants}: RMS, LUFS normalization
                        \item   real-time?
                    \end{itemize}
            \end{itemize}
        \end{frame}
        \begin{frame}{instantaneous features}{audio pre-processing examples 2/2}
            \begin{itemize}
                \item   \textbf{filtering}
                    \begin{itemize}
                        \item   \textit{DC removal}
                            \[
                                x(i) = x_{\mathrm{DC}}(i) - \frac{1}{\mathcal{I}}\sum\limits_{i=0}^{\mathcal{I}-1}{x_{\mathrm{DC}}(i)} 
                            \]
                        \item   other filters: high/band pass, smoothing, ...
                    \end{itemize}
                \bigskip
                \item<2->   \textbf{sample rate reduction}
                \bigskip
                \item<3->   \textbf{quality enhancement} (denoising, etc.)
                \bigskip
                \item<3->   \ldots
            \end{itemize}
        \end{frame}
        
    \section{summary}
        \begin{frame}{summary}{lecture content}
            \begin{itemize}
                \item   \textbf{feature}
                    \begin{itemize}
                        \item   descriptor with condensed relevant information
                        \item   not necessarily interpretable by humans
                    \end{itemize}
                \bigskip
                \item   \textbf{low-level feature extraction}
                    \begin{itemize}
                        \item   usually extracted per short block of samples
                        \item   many features can be extracted from audio data, resulting in feature matrix
                    \end{itemize}
                \bigskip
                \item   \textbf{pre-processing}
                    \begin{itemize}
                        \item   remove irrelevant data,
                        \item   clean relevant data
                    \end{itemize}
            \end{itemize}
            \inserticon{summary}
        \end{frame}
\end{document}
