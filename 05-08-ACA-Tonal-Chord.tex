% move all configuration stuff into includes file so we can focus on the content
\documentclass[aspectratio=169,hyperref={pdfpagelabels=false,colorlinks=true,linkcolor=white,urlcolor=blue},t]{beamer}

%%%%%%%%%%%%%%%%%%%%%%%%%%%%%%%%%%%%%%%%%%%%%%%%%%%%%%%%%%%%%%%%%%%%%%%%%%%%%%%%%%
%%%%%%%%%%%%%%%%%%%%%%%%%%%%%%%%%%%%%%%%%%%%%%%%%%%%%%%%%%%%%%%%%%%%%%%%%%%%%%%%%%
% packages
\usepackage{pict2e}
\usepackage{epic}
\usepackage{amsmath,amsfonts,amssymb}
\usepackage{units}
\usepackage{fancybox}
\usepackage[absolute,overlay]{textpos} 
\usepackage{media9} % avi2flv: "C:\Program Files\ffmpeg\bin\ffmpeg.exe" -i TuneFreqFilterbank.avi -b 600k -s 441x324 -r 15 -acodec copy TuneFreqFilterbank.flv
\usepackage{animate}
\usepackage{gensymb}
\usepackage{multirow}
\usepackage{silence}
\usepackage{tikz}
\usepackage[backend=bibtex,style=ieee]{biblatex}
\AtEveryCitekey{\iffootnote{\tiny}{}}
\addbibresource{include/references}

%%%%%%%%%%%%%%%%%%%%%%%%%%%%%%%%%%%%%%%%%%%%%%%%%%%%%%%%%%%%%%%%%%%%%%%%%%%%%%%%%%
%%%%%%%%%%%%%%%%%%%%%%%%%%%%%%%%%%%%%%%%%%%%%%%%%%%%%%%%%%%%%%%%%%%%%%%%%%%%%%%%%%
% relative paths
\graphicspath{{graph/}}


%%%%%%%%%%%%%%%%%%%%%%%%%%%%%%%%%%%%%%%%%%%%%%%%%%%%%%%%%%%%%%%%%%%%%%%%%%%%%%%%%%
%%%%%%%%%%%%%%%%%%%%%%%%%%%%%%%%%%%%%%%%%%%%%%%%%%%%%%%%%%%%%%%%%%%%%%%%%%%%%%%%%%
% units
\setlength{\unitlength}{1mm}

%%%%%%%%%%%%%%%%%%%%%%%%%%%%%%%%%%%%%%%%%%%%%%%%%%%%%%%%%%%%%%%%%%%%%%%%%%%%%%%%%%
%%%%%%%%%%%%%%%%%%%%%%%%%%%%%%%%%%%%%%%%%%%%%%%%%%%%%%%%%%%%%%%%%%%%%%%%%%%%%%%%%%
% theme & layout
\usetheme{Frankfurt}
\beamertemplatenavigationsymbolsempty
%\setbeamertemplate{frametitle}[smoothbars theme]
\setbeamertemplate{frametitle}
{
    \begin{beamercolorbox}[ht=1.8em,wd=\paperwidth]{frametitle}
        \vspace{-.1em}%
        \hspace{.2em}{\strut\insertframetitle\strut}
        
        \hspace{.2em}\small\strut\insertframesubtitle\strut
        %\hfill
        %\includegraphics[height=.8cm,keepaspectratio]{CenterMusicTechnology-solid-2lines-white-CoAtag}
        
    \end{beamercolorbox}
    \begin{textblock*}{100mm}(11.6cm,.7cm)
        \includegraphics[height=.8cm,keepaspectratio]{logo_GTCMT_black}
    \end{textblock*}
}

% set this to ensure bulletpoints without subsections
\usepackage{remreset}
\makeatletter
\@removefromreset{subsection}{section}
\makeatother
\setcounter{subsection}{1}

%---------------------------------------------------------------------------------
% appearance
\setbeamercolor{structure}{fg=gtgold}
\setbeamercovered{transparent} %invisible
\setbeamercolor{bibliography entry author}{fg=black}
\setbeamercolor*{bibliography entry title}{fg=black}
\setbeamercolor*{bibliography entry note}{fg=black}
\setbeamercolor{frametitle}{fg=black}
\setbeamercolor{title}{fg=black}

%\usepackage{pgfpages}
%\setbeameroption{show notes}
%\setbeameroption{show notes on second screen=right}
%---------------------------------------------------------------------------------
% fontsize
\let\Tiny=\tiny

%%%%%%%%%%%%%%%%%%%%%%%%%%%%%%%%%%%%%%%%%%%%%%%%%%%%%%%%%%%%%%%%%%%%%%%%%%%%%%%%%%
%%%%%%%%%%%%%%%%%%%%%%%%%%%%%%%%%%%%%%%%%%%%%%%%%%%%%%%%%%%%%%%%%%%%%%%%%%%%%%%%%%
% warnings
\pdfsuppresswarningpagegroup=1
\WarningFilter{biblatex}{Patching footnotes failed}
\WarningFilter{latexfont}{Font shape}
\WarningFilter{latexfont}{Some font shapes}
\WarningFilter{gensymb}{Not defining}


%%%%%%%%%%%%%%%%%%%%%%%%%%%%%%%%%%%%%%%%%%%%%%%%%%%%%%%%%%%%%%%%%%%%%%%%%%%%%%%%%%
%%%%%%%%%%%%%%%%%%%%%%%%%%%%%%%%%%%%%%%%%%%%%%%%%%%%%%%%%%%%%%%%%%%%%%%%%%%%%%%%%%
% title information
\title[]{Introduction to \textbf{Audio Content Analysis}}   
\author[alexander lerch]{alexander lerch} 
%\institute{~}
%\date[Alexander Lerch]{}
%\titlegraphic{\vspace{-16mm}\includegraphics[width=\textwidth,height=3cm]{title}}

%%%%%%%%%%%%%%%%%%%%%%%%%%%%%%%%%%%%%%%%%%%%%%%%%%%%%%%%%%%%%%%%%%%%%%%%%%%%%%%%%%
%%%%%%%%%%%%%%%%%%%%%%%%%%%%%%%%%%%%%%%%%%%%%%%%%%%%%%%%%%%%%%%%%%%%%%%%%%%%%%%%%%
% colors
\definecolor{gtgold}{HTML}{96caff} %0e7eed {rgb}{0.88,0.66,1,0.06} [234, 170, 0]/256
\definecolor{darkgray}{rgb}{.1, .1, .25}
\definecolor{lightblue}{HTML}{0e7eed}
\definecolor{highlight}{rgb}{0, 0, 1} %_less!40

%%%%%%%%%%%%%%%%%%%%%%%%%%%%%%%%%%%%%%%%%%%%%%%%%%%%%%%%%%%%%%%%%%%%%%%%%%%%%%%%%%
%%%%%%%%%%%%%%%%%%%%%%%%%%%%%%%%%%%%%%%%%%%%%%%%%%%%%%%%%%%%%%%%%%%%%%%%%%%%%%%%%%
% relative paths
\graphicspath{{../ACA-Plots/graph/}}


%%%%%%%%%%%%%%%%%%%%%%%%%%%%%%%%%%%%%%%%%%%%%%%%%%%%%%%%%%%%%%%%%%%%%%%%%%%%%%%%%%
%%%%%%%%%%%%%%%%%%%%%%%%%%%%%%%%%%%%%%%%%%%%%%%%%%%%%%%%%%%%%%%%%%%%%%%%%%%%%%%%%%
% units
\setlength{\unitlength}{1mm}

%%%%%%%%%%%%%%%%%%%%%%%%%%%%%%%%%%%%%%%%%%%%%%%%%%%%%%%%%%%%%%%%%%%%%%%%%%%%%%%%%%
%%%%%%%%%%%%%%%%%%%%%%%%%%%%%%%%%%%%%%%%%%%%%%%%%%%%%%%%%%%%%%%%%%%%%%%%%%%%%%%%%%
% math
\DeclareMathOperator*{\argmax}{argmax}
\DeclareMathOperator*{\argmin}{argmin}
\DeclareMathOperator*{\atan}{atan}
\DeclareMathOperator*{\arcsinh}{arcsinh}
\DeclareMathOperator*{\sign}{sign}
\DeclareMathOperator*{\tcdf}{tcdf}
\DeclareMathOperator*{\si}{sinc}
\DeclareMathOperator*{\princarg}{princarg}
\DeclareMathOperator*{\arccosh}{arccosh}
\DeclareMathOperator*{\hwr}{HWR}
\DeclareMathOperator*{\flip}{flip}
\DeclareMathOperator*{\sinc}{sinc}
\DeclareMathOperator*{\floor}{floor}
\newcommand{\e}{{e}}
\newcommand{\jom}{\mathrm{j}\omega}
\newcommand{\jOm}{\mathrm{j}\Omega}
\newcommand   {\mat}[1]    		{\boldsymbol{\uppercase{#1}}}		%bold
\renewcommand {\vec}[1]    		{\boldsymbol{\lowercase{#1}}}		%bold

%%%%%%%%%%%%%%%%%%%%%%%%%%%%%%%%%%%%%%%%%%%%%%%%%%%%%%%%%%%%%%%%%%%%%%%%%%%%%%%%%%
%%%%%%%%%%%%%%%%%%%%%%%%%%%%%%%%%%%%%%%%%%%%%%%%%%%%%%%%%%%%%%%%%%%%%%%%%%%%%%%%%%
% media9
\newcommand{\includeaudio}[1]{
\href{run:audio/#1.mp3}{\includegraphics[width=5mm, height=5mm]{graph/SpeakerIcon}}}

\newcommand{\includeanimation}[4]{{\begin{center}
                        \animategraphics[autoplay,loop,scale=.7]{#4}{animation/#1-}{#2}{#3}        
                        \end{center}
                        \addreference{matlab source: \href{https://github.com/alexanderlerch/ACA-Plots/blob/master/matlab/animate#1.m}{matlab/animate#1.m}}}
                        \inserticon{video}}
                        
%%%%%%%%%%%%%%%%%%%%%%%%%%%%%%%%%%%%%%%%%%%%%%%%%%%%%%%%%%%%%%%%%%%%%%%%%%%%%%%%%%
%%%%%%%%%%%%%%%%%%%%%%%%%%%%%%%%%%%%%%%%%%%%%%%%%%%%%%%%%%%%%%%%%%%%%%%%%%%%%%%%%%
% other commands
\newcommand{\question}[1]{%\vspace{-4mm}
                          \setbeamercovered{invisible}
                          \begin{columns}[T]
                            \column{.9\textwidth}
                                \textbf{#1}
                            \column{.1\textwidth}
                                \vspace{-8mm}
                                \begin{flushright}
                                     \includegraphics[width=.9\columnwidth]{graph/question_mark}
                                \end{flushright}
                                \vspace{6mm}
                          \end{columns}\pause\vspace{-12mm}}

\newcommand{\toremember}[1]{
                        \inserticon{lightbulb}
                        }

\newcommand{\matlabexercise}[1]{%\vspace{-4mm}
                          \setbeamercovered{invisible}
                          \begin{columns}[T]
                            \column{.8\textwidth}
                                \textbf{matlab exercise}: #1
                            \column{.2\textwidth}
                                \begin{flushright}
                                     \includegraphics[scale=.5]{graph/logo_matlab}
                                \end{flushright}
                                %\vspace{6mm}
                          \end{columns}}

\newcommand{\addreference}[1]{  
                  
                    \begin{textblock*}{\baselineskip }(.98\paperwidth,.5\textheight) %(1.15\textwidth,.4\textheight)
                         \begin{minipage}[b][.5\paperheight][b]{1cm}%
                            \vfill%
                             \rotatebox{90}{\tiny {#1}}
                        \end{minipage}
                   \end{textblock*}
                    }
                    
\newcommand{\figwithmatlab}[1]{
                    \begin{figure}
                        \centering
                        \includegraphics[scale=.7]{#1}
                        %\label{fig:#1}
                    \end{figure}
                    
                    \addreference{matlab source: \href{https://github.com/alexanderlerch/ACA-Plots/blob/main/matlab/plot#1.m}{plot#1.m}}}
\newcommand{\figwithref}[2]{
                    \begin{figure}
                        \centering
                        \includegraphics[scale=.7]{#1}
                        \label{fig:#1}
                    \end{figure}
                    
                    \addreference{#2}}  
                                    
\newcommand{\inserticon}[1]{
                    \begin{textblock*}{100mm}(14.5cm,7.5cm)
                        \includegraphics[height=.8cm,keepaspectratio]{graph/#1}
                    \end{textblock*}}            

%%%%%%%%%%%%%%%%%%%%%%%%%%%%%%%%%%%%%%%%%%%%%%%%%%%%%%%%%%%%%%%%%%%%%%%%%%%%%%%%%%
%%%%%%%%%%%%%%%%%%%%%%%%%%%%%%%%%%%%%%%%%%%%%%%%%%%%%%%%%%%%%%%%%%%%%%%%%%%%%%%%%%
% counters
\newcounter{i}
\newcounter{j}
\newcounter{iXOffset}
\newcounter{iYOffset}
\newcounter{iXBlockSize}
\newcounter{iYBlockSize}
\newcounter{iYBlockSizeDiv2}
\newcounter{iXBlockSizeDiv2}
\newcounter{iDistance}

\newcommand{\IEEELink}{https://ieeexplore.ieee.org/servlet/opac?bknumber=9965970}



\subtitle{Module 5.8: Chord Detection}

%%%%%%%%%%%%%%%%%%%%%%%%%%%%%%%%%%%%%%%%%%%%%%%%%%%%%%%%%%%%%%%%%%%%%%%%%%%%
\begin{document}
    % generate title page
	

\begin{frame}
    \titlepage
    %\vspace{-5mm}
    \begin{flushright}
        \href{http://www.gtcmt.gatech.edu}{\includegraphics[height=.8cm,keepaspectratio]{logo_GTCMT_black}}
    \end{flushright}
\end{frame}


    \section[overview]{lecture overview}
        \begin{frame}{introduction}{overview}
            \begin{block}{corresponding textbook section}
                    \href{http://ieeexplore.ieee.org/xpl/articleDetails.jsp?arnumber=6331122}{Chapter 5~---~Tonal Analysis}: pp.~125--127
            \end{block}

            \begin{itemize}
                \item   \textbf{lecture content}
                    \begin{itemize}
                        \item   musical chords and harmony
                        \item   sample chord detection
                        \item   Hidden Markov Models (HMMs) and the Viterbi algorithm
                    \end{itemize}
                \bigskip
                \item<2->   \textbf{learning objectives}
                    \begin{itemize}
                        \item   name basic chords and describe the concept of chord inversions
                        \item   discuss commonalities and differences between chord \& key detection
                        \item   discuss the usefulness of HMMs for chord detection
                        \item   explain the Viterbi algorithm with an example
                    \end{itemize}
            \end{itemize}
            \inserticon{directions}
        \end{frame}
        
    \section{chords}
        \begin{frame}{musical pitch}{chords}
            \begin{itemize}
                \item	simultaneous use of several pitches $\Rightarrow$ \textbf{chords}
                \item	usually constructed of (major/minor) thirds
                \begin{figure}[t]
                    \centering
                    \includegraphics[scale=.85]{pitch_chords}
                \end{figure}
                
                \smallskip
                \item<2->	note:
                        \begin{itemize}
                            \item	chord type independent of pitch doubling, pitch order
                            \item	same label for keys and chords
                        \end{itemize}
            \end{itemize}
        \end{frame}
        
        \begin{frame}{musical pitch}{ chord inversion}
            \begin{itemize}
                \item	most common: root note is lowest note
                \item	otherwise: chord inversion
                \begin{figure}[t]
                    \centering
                    \includegraphics{pitch_chordinversions}
                \end{figure}
                
            \end{itemize}
        \end{frame}
        
        \begin{frame}{musical pitch}{ harmony}
            \begin{itemize}
                \item	key and tonal context define chord's \textit{harmonic function}
                \smallskip
                \item   examples:
                \begin{itemize}
                    \item	\textbf{tonic}:\\ chord on 1st scale degree (tonal center)
                    \item	\textbf{dominant}:\\ chord on 5th scale degree (often moves to tonic)
                    \item	\textbf{subdominant}:\\ chord on 4th scale degree
                    \item	\ldots
                \end{itemize}
            \end{itemize}
        \end{frame}
    \section{chord detection}
        \begin{frame}{chord detection}{introduction: key vs.\ chord detection}

            \begin{itemize}
                \item	\textbf{commonalities}
                    \begin{itemize}
                        \item<1->	chords are octave independent $\Rightarrow$ pitch chroma sufficient
                        \item<1->	process flow: pitch chroma extraction $+$ classification
                    \end{itemize}
                \bigskip
                \item<2->	\textbf{differences}
                    \begin{itemize}
                        \item	time frame for pitch chroma calculation
                        \item	templates
                        \item	number of templates/chords
                        \item	many results per song (time series)
                    \end{itemize}
            \end{itemize}
        \end{frame}
        \begin{frame}{pitch chroma}{introduction}
            \vspace{-3mm}
            \begin{itemize}
                \item	pitch class distribution: 12-dimensional vector
                \item   map all pitch class bands in all octaves to one
            \end{itemize}
            {\vspace{-12mm}\visible<2->{\begin{flushright}{\includeaudio{sax_example}}\end{flushright}}}
            \only<1>{\vspace{1mm}\figwithmatlab{PitchChromaGrouping}}
            \only<2>{\vspace{4mm}\figwithmatlab{PitchChroma}}
            \only<3->{
            
            \bigskip
            \bigskip
            \begin{block}{pitch chroma properties}
            \begin{itemize}
                \item	\textbf{no} octave information
                    \begin{itemize}
                        \item	no differentiation between prime and octave
                        \item   no info on inversion
                    \end{itemize}
                \item	robust, timbre-independent representation
            \end{itemize}
            \end{block}
            
            }  
            \vspace{50mm}
            \inserticon{audio}
        \end{frame}

        \begin{frame}{chord detection}{chord template}
            \begin{itemize}
                \item similar to key detection we can simply compare an extracted pitch chroma with a template
                    \begin{itemize}
                        \item	simplest possible template and distance: linear transformation  
                    
                            example~---~C major: $\Gamma(0,j) = [\nicefrac{1}{3},0,0,0,\nicefrac{1}{3},0,0,\nicefrac{1}{3},0,0,0,0]$
                        \smallskip
                        \item[$\Rightarrow$]	instantaneous chord likelihood:
                        \begin{equation*}
                            {\psi}(c,n) = \sum\limits_{j = 0}^{11}{\Gamma(c,j)\cdot \nu(j,n)}
                        \end{equation*}
                    \end{itemize}	
            \end{itemize}
        \end{frame}
        
        \begin{frame}{chord detection}{chord progression 1/2}
            apply \textbf{musical knowledge} to increase the result's robustness and accuracy:
            
            \begin{itemize}
                \item	different probabilities for different chord progressions (similar to key modulations), e.g.
                \begin{itemize}
                    \item	cadences: I-IV-V-I
                    \item	sequences: circle progression
                    
                \end{itemize}
            \end{itemize}

            $\Rightarrow$ model for \textit{chord progression probabilities}
            \begin{enumerate}
                \item<2->	\textit{analytical model} based on music theory
                    \begin{itemize}
                        \item	circle of fifths (?!)
                        \item	key profile correlation (?!)
                    \end{itemize}
                \item<3->	\textit{empirical model} based on data
                    \begin{itemize}	
                        \item	annotate audio
                        \item	symbolic score
                    \end{itemize}
            \end{enumerate}
        \end{frame}
        \begin{frame}{chord detection}{chord progression 2/2}
            \question{what properties do chord progression probabilities depend on}

            \begin{itemize}
                \item 	musical key
                \item	larger musical context (model order)
                \item	style
                \item   tempo/length??
            \end{itemize}
        \end{frame}
    \section[HMMs]{Hidden Markov Models \& Viterbi algorithm}
        \begin{frame}{chord detection}{markov chain}
            \begin{figure}
                \centering
                    \includegraphics[scale=.1]{graph/MarkovChain}
            \end{figure}
            \addreference{from: \url{https://commons.wikimedia.org/wiki/File:Markovkate_01.svg}}
            \begin{itemize}
                \item   two possible states E, A
                \item   transition probabilities to other state(s) and to self
                \item   sum of transition probabilities equals 1
            \end{itemize}
        \end{frame}
        
        \begin{frame}{chord detection}{hidden markov model: variables}
            \begin{itemize}
                \item	\textbf{states}:\\ unknown/hidden
                \smallskip
                \item	\textbf{transition probability}:\\ probability of transitioning from one state to the other
                \smallskip
                \item   \textbf{observations}:\\ measureable time series
                \smallskip
                \item	\textbf{emission probability}:\\ probability of a state given an observation
                \smallskip
                \item	\textbf{start probability}:\\ probability of the initial state
            \end{itemize}
        \end{frame}
        \begin{frame}{chord detection}{hidden markov model: variables}
            \vspace{-5mm}
            \begin{figure}
                \centering
                    \includegraphics[scale=.25]{graph/HiddenMarkovModel}
            \end{figure}
            \addreference{from \url{https://en.wikipedia.org/wiki/File:HiddenMarkovModel.svg}}
            \vspace{-5mm}
            \begin{footnotesize}
                \begin{itemize}
                    \item	X: states
                    \item	y: possible observations
                    \item	a: state transition probabilities
                    \item	b: emission probabilities
                \end{itemize}
            \end{footnotesize}
        \end{frame}
        \begin{frame}{chord detection}{hidden markov model: example (WP) 1/2}
            \begin{itemize}
                \item   \textbf{scenario}
                    \begin{itemize}
                        \item   doctor diagnoses fever by how patients feel
                        \item   patient may feel normal, dizzy, or cold
                        \item   patient visits multiple days in a row 
                    \end{itemize}
            \end{itemize}
            \question{what are the states and observations in this case}
            
            \begin{itemize}
                \item	\textbf{states} %: \textit{healthy}, \textit{fever}
                    \begin{itemize}
                        \item   \textit{healthy}
                        \item   \textit{fever}
                    \end{itemize}
                \item   \textbf{observations}: %\textit{normal}, \textit{cold}, \textit{dizzy}
                    \begin{itemize}
                        \item   \textit{normal}
                        \item   \textit{cold}
                        \item   \textit{dizzy}
                    \end{itemize}
            \end{itemize}
        \end{frame}
        \begin{frame}{chord detection}{hidden markov model: example (WP) 2/2}
            %\begin{columns}[T]
            %\column{.6\textwidth}
            \vspace{-5mm}
            \only<1>{
            \begin{itemize}
                \item   \textbf{start probabilities} (initial state assumption)
                    \begin{itemize}
                        \item   \textit{healthy}: $ 0.6$
                        \item   \textit{fever}: $0.4$
                    \end{itemize}
                \item<2->   \textbf{emission probabilities} (prob of obs given state)
                    \begin{itemize}
                        \item   \textit{healthy}: normal $0.5$, cold $0.4$, dizzy $0.1$
                        \item   \textit{fever}: : normal $0.1$, cold $0.3$, dizzy $0.6$
                    \end{itemize}
                \item<3->   \textbf{transition probabilities}
                    \begin{itemize}
                        \item   \textit{healthy}: healthy $0.7$, fever $0.3$
                        \item   \textit{fever}: : healthy $0.4$, fever $0.6$
                    \end{itemize}
            \end{itemize}
            }
            \only<2>{
            \begin{figure}
                \includegraphics[scale=.6]{viterbi_example}
            \end{figure}
            }
            %\column{.4\textwidth}
            %\only<4->{
                    %\includegraphics[scale=.35]{HmmExample}
            %}
            %\end{columns}
            %\addreference{from: \url{https://en.wikipedia.org/wiki/File:An_example_of_HMM.png}}
        \end{frame}
        \begin{frame}{chord detection}{hidden markov model: example (WP) 2/2}

            \begin{itemize}
                \item[]   \textbf{three observations}:\\ Day 1 \textit{normal} $\rightarrow$ Day 2 \textit{cold} $\rightarrow$ Day 3 \textit{dizzy}
            \end{itemize}
            \setcounter{i}{0}
            \whiledo{\value{i}<4}	
            {
                \only<\value{beamerpauses}>
                {
                    \begin{figure}
                    \centering
                        \includegraphics[scale=.5]{viterbi_example_\arabic{i}}
                    \end{figure}
                }
                \ifthenelse{\equal{\value{i}}{3}}{}{\pause}
                \stepcounter{i} 
            }	
            %\visible<1->{\addreference{from: \url{https://en.wikipedia.org/wiki/Viterbi_algorithm\#/media/File:Viterbi_animated_demo.gif}}}
        \end{frame}
        \begin{frame}{chord detection}{HMMs for chord detection}
            \begin{itemize}
                \item   states $\rightarrow$ chords
                \item   observations $\rightarrow$ pitch chroma
                \item   emission probability $\rightarrow$ trained with pitch chroma
                \item   transition probability $\rightarrow$ trained from dataset
                \item   start probability $\rightarrow$ chord statistics (style dependent?)
            \end{itemize}
        \end{frame}
                
    \section{summary}
        \begin{frame}{summary}{lecture content}
            \begin{itemize}
                \item   \textbf{chords}
                    \begin{itemize}
                        \item   combination of three or more pitches
                        \item   usually stacked thirds
                        \item   can be inverted
                    \end{itemize}
                \smallskip
                \item   \textbf{chord detection}
                    \begin{itemize}
                        \item   processing steps
                            \begin{itemize}
                                \item   pitch chroma extraction
                                \item   template matching
                                \item   chord transition model
                            \end{itemize}
                    \end{itemize}
                \smallskip
                \item   \textbf{Viterbi algorithm}
                    \begin{itemize}
                        \item   find globally optimal path through state space
                        \item   estimate state sequence with
                            \begin{itemize}
                                \item   emission probabilities 
                                \item   transition probabilities
                            \end{itemize}
                    \end{itemize}
            \end{itemize}
            \inserticon{summary}
        \end{frame}
\end{document}
