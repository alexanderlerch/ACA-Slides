% move all configuration stuff into one file so we can focus on the content
\documentclass[hyperref={pdfpagelabels=false,colorlinks=true,linkcolor=white,urlcolor=blue}]{beamer}

%%%%%%%%%%%%%%%%%%%%%%%%%%%%%%%%%%%%%%%%%%%%%%%%%%%%%%%%%%%%%%%%%%%%%%%%%%%%%%%%%%
%%%%%%%%%%%%%%%%%%%%%%%%%%%%%%%%%%%%%%%%%%%%%%%%%%%%%%%%%%%%%%%%%%%%%%%%%%%%%%%%%%
% packages
\usepackage{pict2e}
\usepackage{epic}
\usepackage{amsmath,amsfonts,amssymb}
\usepackage{units}
\usepackage{fancybox}
\usepackage[absolute,overlay]{textpos} 
\usepackage{media9} % avi2flv: "C:\Program Files\ffmpeg\bin\ffmpeg.exe" -i TuneFreqFilterbank.avi -b 600k -s 441x324 -r 15 -acodec copy TuneFreqFilterbank.flv
\usepackage{silence}
\usepackage[backend=bibtex,style=ieee]{biblatex}
\WarningFilter{biblatex}{Patching footnotes failed}
\AtEveryCitekey{\iffootnote{\tiny}{}}
\addbibresource{references}

%%%%%%%%%%%%%%%%%%%%%%%%%%%%%%%%%%%%%%%%%%%%%%%%%%%%%%%%%%%%%%%%%%%%%%%%%%%%%%%%%%
%%%%%%%%%%%%%%%%%%%%%%%%%%%%%%%%%%%%%%%%%%%%%%%%%%%%%%%%%%%%%%%%%%%%%%%%%%%%%%%%%%
% relative paths
\graphicspath{{graph/}}

%%%%%%%%%%%%%%%%%%%%%%%%%%%%%%%%%%%%%%%%%%%%%%%%%%%%%%%%%%%%%%%%%%%%%%%%%%%%%%%%%%
%%%%%%%%%%%%%%%%%%%%%%%%%%%%%%%%%%%%%%%%%%%%%%%%%%%%%%%%%%%%%%%%%%%%%%%%%%%%%%%%%%
% colors
\definecolor{gtgold}{HTML}{E0AA0F} %{rgb}{0.88,0.66,1,0.06} [234, 170, 0]/256

%%%%%%%%%%%%%%%%%%%%%%%%%%%%%%%%%%%%%%%%%%%%%%%%%%%%%%%%%%%%%%%%%%%%%%%%%%%%%%%%%%
%%%%%%%%%%%%%%%%%%%%%%%%%%%%%%%%%%%%%%%%%%%%%%%%%%%%%%%%%%%%%%%%%%%%%%%%%%%%%%%%%%
% math
\DeclareMathOperator*{\argmax}{argmax}
\DeclareMathOperator*{\argmin}{argmin}
\DeclareMathOperator*{\atan}{atan}
\DeclareMathOperator*{\arcsinh}{arcsinh}
\DeclareMathOperator*{\sign}{sign}
\DeclareMathOperator*{\tcdf}{tcdf}
\DeclareMathOperator*{\si}{sinc}
\DeclareMathOperator*{\princarg}{princarg}
\DeclareMathOperator*{\arccosh}{arccosh}
\DeclareMathOperator*{\hwr}{HWR}
\DeclareMathOperator*{\flip}{flip}
\DeclareMathOperator*{\sinc}{sinc}
\newcommand{\e}{{e}}
\newcommand{\jom}{\mathrm{j}\omega}
\newcommand{\jOm}{\mathrm{j}\Omega}
\newcommand   {\mat}[1]    		{\boldsymbol{\uppercase{#1}}}		%bold
\renewcommand {\vec}[1]    		{\boldsymbol{\lowercase{#1}}}		%bold

%%%%%%%%%%%%%%%%%%%%%%%%%%%%%%%%%%%%%%%%%%%%%%%%%%%%%%%%%%%%%%%%%%%%%%%%%%%%%%%%%%
%%%%%%%%%%%%%%%%%%%%%%%%%%%%%%%%%%%%%%%%%%%%%%%%%%%%%%%%%%%%%%%%%%%%%%%%%%%%%%%%%%
% media9
\newcommand{\includeaudio}[1]{{\includemedia[
                        addresource=audio/#1.mp3,
                        width=5mm,
                        height=5mm,
                        activate=onclick,
                        flashvars={
                            source=audio/#1.mp3  
                            &autoPlay=true
                        }]
                        {\includegraphics[width=5mm, height=5mm]{SpeakerIcon}}
                        {APlayer.swf}}}
\newcommand{\audioautoplay}[1]{{\begin{center}\includemedia[
                            addresource=audio/#1.mp3,
                            width=.1\linewidth,
                            height=.01\linewidth,
                            activate=pageopen,
                            flashvars={
                                source=audio/#1.mp3  
                                &autoPlay=true
                            }]
                            {}
                            {APlayer.swf}\end{center}}}

\newcommand{\includevideo}[1]{{\begin{center}\includemedia[
                        addresource=video/#1.mp4,
                        width=0.8\linewidth,
                        height=0.4\linewidth,
                        activate=onclick,
                        flashvars={
                            source=video/#1.mp4  
                            &autoPlay=true
                        }]
                        {}
                        {VPlayer.swf}\end{center}}}
\newcommand{\videowithmatlab}[1]{{\begin{center}\includemedia[
                        addresource=video/animate#1.mp4,
                        width=0.8\linewidth,
                        height=0.4\linewidth,
                        activate=onclick,
                        flashvars={
                            source=video/animate#1.mp4  
                            &autoPlay=true
                        }]
                        {}
                        {VPlayer.swf}\end{center}\addreference{matlab source: matlab/animate#1.m}}}
                        

%%%%%%%%%%%%%%%%%%%%%%%%%%%%%%%%%%%%%%%%%%%%%%%%%%%%%%%%%%%%%%%%%%%%%%%%%%%%%%%%%%
%%%%%%%%%%%%%%%%%%%%%%%%%%%%%%%%%%%%%%%%%%%%%%%%%%%%%%%%%%%%%%%%%%%%%%%%%%%%%%%%%%
% other commands
\newcommand{\question}[1]{\vspace{-4mm}
                          \setbeamercovered{invisible}
                          \begin{columns}
                            \column{.7\textwidth}
                                \textbf{#1}\\
                            \column{.3\textwidth}
                                \begin{flushright}
                                     \includegraphics[scale=.5]{question_mark}
                                \end{flushright}
                                \vspace{6mm}
                          \end{columns}\pause\vspace{-6mm}}

\newcommand{\toremember}[1]{\vspace{-4mm}
                          \begin{columns}
                            \column{.7\textwidth}
                                \textbf{#1}\\
                            \column{.3\textwidth}
                                \begin{flushright}
                                     \includegraphics[scale=.5]{exclamation_mark}
                                \end{flushright}
                                \vspace{6mm}
                          \end{columns}\vspace{-6mm}}

\newcommand{\matlabexercise}[1]{\vspace{-4mm}
                          \setbeamercovered{invisible}
                          \begin{columns}
                            \column{.7\textwidth}
                                \textbf{matlab exercise}: #1
                            \column{.3\textwidth}
                                \begin{flushright}
                                     \includegraphics[scale=.5]{logo_matlab}
                                \end{flushright}
                                \vspace{6mm}
                          \end{columns}}

\newcommand{\addreference}[1]{                    
                    \begin{textblock*}{\baselineskip }(1.15\textwidth,.5\textheight)
                        \rotatebox{90}{\tiny #1}
                    \end{textblock*}}
                    
\newcommand{\figwithmatlab}[1]{
                    \begin{figure}
                        \centering
                        \includegraphics{#1}
                        \label{fig:#1}
                    \end{figure}
                    
                    \addreference{matlab source: matlab/display#1.m}}                    
\newcommand{\figwithref}[2]{
                    \begin{figure}
                        \centering
                        \includegraphics{#1}
                        \label{fig:#1}
                    \end{figure}
                    
                    \addreference{#2}}                    

%%%%%%%%%%%%%%%%%%%%%%%%%%%%%%%%%%%%%%%%%%%%%%%%%%%%%%%%%%%%%%%%%%%%%%%%%%%%%%%%%%
%%%%%%%%%%%%%%%%%%%%%%%%%%%%%%%%%%%%%%%%%%%%%%%%%%%%%%%%%%%%%%%%%%%%%%%%%%%%%%%%%%
% units
\setlength{\unitlength}{1mm}

%%%%%%%%%%%%%%%%%%%%%%%%%%%%%%%%%%%%%%%%%%%%%%%%%%%%%%%%%%%%%%%%%%%%%%%%%%%%%%%%%%
%%%%%%%%%%%%%%%%%%%%%%%%%%%%%%%%%%%%%%%%%%%%%%%%%%%%%%%%%%%%%%%%%%%%%%%%%%%%%%%%%%
% counters
\newcounter{i}
\newcounter{iXOffset}
\newcounter{iYOffset}
\newcounter{iXBlockSize}
\newcounter{iYBlockSize}
\newcounter{iYBlockSizeDiv2}
\newcounter{iDistance}


%%%%%%%%%%%%%%%%%%%%%%%%%%%%%%%%%%%%%%%%%%%%%%%%%%%%%%%%%%%%%%%%%%%%%%%%%%%%%%%%%%
%%%%%%%%%%%%%%%%%%%%%%%%%%%%%%%%%%%%%%%%%%%%%%%%%%%%%%%%%%%%%%%%%%%%%%%%%%%%%%%%%%
% theme & layout
\usetheme{Frankfurt}
\beamertemplatenavigationsymbolsempty
%\setbeamertemplate{frametitle}[smoothbars theme]
\setbeamertemplate{frametitle}
{
    \begin{beamercolorbox}[ht=1.8em,wd=\paperwidth]{frametitle}
        \vspace{-.1em}%
        \hspace{.2em}{\strut\insertframetitle\strut}
        
        \hspace{.2em}\small\strut\insertframesubtitle\strut
        %\hfill
        %\includegraphics[height=.8cm,keepaspectratio]{CenterMusicTechnology-solid-2lines-white-CoAtag}
        
    \end{beamercolorbox}
    \begin{textblock*}{100mm}(8.3cm,.65cm)
        \includegraphics[height=.8cm,keepaspectratio]{CenterMusicTechnology-solid-2lines-white-CoAtag}
    \end{textblock*}
}

% set this to ensure bulletpoints without subsections
\usepackage{remreset}
\makeatletter
\@removefromreset{subsection}{section}
\makeatother
\setcounter{subsection}{1}

%---------------------------------------------------------------------------------
% appearance
\setbeamercolor{structure}{fg=gtgold}
\setbeamercovered{transparent} %invisible
\setbeamercolor{bibliography entry author}{fg=black}
\setbeamercolor*{bibliography entry title}{fg=black}
\setbeamercolor*{bibliography entry note}{fg=black}
%---------------------------------------------------------------------------------
% fontsize
\let\Tiny=\tiny

%%%%%%%%%%%%%%%%%%%%%%%%%%%%%%%%%%%%%%%%%%%%%%%%%%%%%%%%%%%%%%%%%%%%%%%%%%%%%%%%%%
%%%%%%%%%%%%%%%%%%%%%%%%%%%%%%%%%%%%%%%%%%%%%%%%%%%%%%%%%%%%%%%%%%%%%%%%%%%%%%%%%%
% warnings
\pdfsuppresswarningpagegroup=1

%%%%%%%%%%%%%%%%%%%%%%%%%%%%%%%%%%%%%%%%%%%%%%%%%%%%%%%%%%%%%%%%%%%%%%%%%%%%%%%%%%
%%%%%%%%%%%%%%%%%%%%%%%%%%%%%%%%%%%%%%%%%%%%%%%%%%%%%%%%%%%%%%%%%%%%%%%%%%%%%%%%%%
% title information
\title[]{MUSI-6201~---~Computational Music Analysis}   
\author[alexander lerch]{alexander lerch} 
%\institute{~}
%\date[Alexander Lerch]{}
\titlegraphic{\vspace{-16mm}\includegraphics[scale=.25]{title}}


\subtitle{Part 3.2: Fundamentals II}

%%%%%%%%%%%%%%%%%%%%%%%%%%%%%%%%%%%%%%%%%%%%%%%%%%%%%%%%%%%%%%%%%%%%%%%%%%%%
\begin{document}
    % generate title page
	

\begin{frame}
    \titlepage
    %\vspace{-5mm}
    \begin{flushright}
        \href{http://www.gtcmt.gatech.edu}{\includegraphics[height=.8cm,keepaspectratio]{logo_GTCMT_black}}
    \end{flushright}
\end{frame}


    \section[overview]{lecture overview}
        \begin{frame}{introduction}{overview}
            \begin{itemize}
                \item   \textbf{text book}  
                    \begin{itemize}
                        \item   \href{http://ieeexplore.ieee.org/xpl/ebooks/bookPdfWithBanner.jsp?fileName=6331119.pdf&bkn=6266785&pdfType=chapter}{\underline{\textit{Chapter 2: Fundamentals} (pp.~18--28)}}
                    \end{itemize}
                \item   \textbf{additional reading}  
                    \begin{itemize}
                        \item   Richard G.~Lyons, \textit{Understanding Digital Signal Processing}, 3rd, Prentice Hall/Pearson, 2011
                    \end{itemize}
                \bigskip
                \item<2->   \textbf{lecture content}
                    \begin{itemize}
                        \item<2->   block-based processing
                        \item<3->   correlation
                        \item<4->   Fourier Transform
                        \item<5->   other time-frequency transforms
                    \end{itemize}
            \end{itemize}
        \end{frame}
        
    \section[intro]{introduction}
        \begin{frame}{audio signals}{signal categories}
        \end{frame}

    \section[processing blocks]{block-based processing}
        \begin{frame}{block based processing}{system implementation}
            typical audio applications process chunks of audio data:
            \vspace{-4mm}
            \begin{columns}
                \column{.6\textwidth}
                    \begin{figure}
                        \centering
                        \begin{footnotesize}
				\begin{picture}(60,40)
					\setcounter{iXOffset}{0}
					\setcounter{iYOffset}{24}
					\setcounter{iXBlockSize}{16}
					\setcounter{iYBlockSize}{4}
					\setcounter{iYBlockSizeDiv2}{2}
					\setcounter{iDistance}{8}

					% block indices
					\put(\value{iXOffset}, \value{iYOffset})
						{\text{{\shortstack[c]{$n$}}}}
					\addtocounter{iYOffset}{-\value{iYBlockSize}}
					\addtocounter{iYOffset}{-\value{iYBlockSizeDiv2}}
					\put(\value{iXOffset}, \value{iYOffset})
						{\text{{\shortstack[c]{$n+1$}}}}
					\addtocounter{iYOffset}{-\value{iYBlockSize}}
					\addtocounter{iYOffset}{-\value{iYBlockSizeDiv2}}
					\put(\value{iXOffset}, \value{iYOffset})
						{\text{{\shortstack[c]{$n+2$}}}}
					\addtocounter{iYOffset}{-\value{iYBlockSize}}
					\addtocounter{iYOffset}{-\value{iYBlockSizeDiv2}}
					\put(\value{iXOffset}, \value{iYOffset})
						{\text{{\shortstack[c]{$n+3$}}}}

					% audio time line
					\setcounter{iYOffset}{30}
					\setcounter{iXOffset}{5}
					\put(\value{iXOffset}, \value{iYOffset})
						{\vector(1,0){50}}
	
					% blocks
					\addtocounter{iXOffset}{5}
					\addtocounter{iYOffset}{-\value{iYBlockSize}}
					\addtocounter{iYOffset}{-\value{iYBlockSizeDiv2}}
					\put(\value{iXOffset}, \value{iYOffset})
						{\framebox(\value{iXBlockSize}, \value{iYBlockSize})}
					\addtocounter{iXOffset}{5}
					\addtocounter{iYOffset}{-\value{iYBlockSize}}
					\addtocounter{iYOffset}{-\value{iYBlockSizeDiv2}}
					\put(\value{iXOffset}, \value{iYOffset})
						{\framebox(\value{iXBlockSize}, \value{iYBlockSize})}
					\addtocounter{iXOffset}{5}
					\addtocounter{iYOffset}{-\value{iYBlockSize}}
					\addtocounter{iYOffset}{-\value{iYBlockSizeDiv2}}
					\put(\value{iXOffset}, \value{iYOffset})
						{\framebox(\value{iXBlockSize}, \value{iYBlockSize})}
					\addtocounter{iXOffset}{5}
					\addtocounter{iYOffset}{-\value{iYBlockSize}}
					\addtocounter{iYOffset}{-\value{iYBlockSizeDiv2}}
					\put(\value{iXOffset}, \value{iYOffset})
						{\framebox(\value{iXBlockSize}, \value{iYBlockSize})}

	
					% lengths
					\linethickness{.03mm}
					\setcounter{iYOffset}{34}
					\setcounter{iXOffset}{10}
					\put(\value{iXOffset}, \value{iYOffset})
						{\line(0,-1){12}}
					\put(\value{iXOffset}, \value{iYOffset})
						{\vector(1,0){1}}
					\addtocounter{iXOffset}{5}
					\put(\value{iXOffset}, \value{iYOffset})
						{\line(0,-1){18}}
					\put(\value{iXOffset}, \value{iYOffset})
						{\vector(-1,0){1}}
					\addtocounter{iXOffset}{5}
					\addtocounter{iXOffset}{5}
					\put(\value{iXOffset}, \value{iYOffset})
						{\line(0,-1){30}}
					\put(\value{iXOffset}, \value{iYOffset})
						{\vector(1,0){1}}
					\addtocounter{iXOffset}{\value{iXBlockSize}}
					\put(\value{iXOffset}, \value{iYOffset})
						{\line(0,-1){30}}
					\put(\value{iXOffset}, \value{iYOffset})
						{\vector(-1,0){1}}

					\put(11, 34)
						{\text{$\mathcal{H}$}}
					\put(32, 34)
						{\text{$\mathcal{K}$}}
					\put(56, 28)
						{\text{$i$}}
				\end{picture}
\end{footnotesize}
	
                    \end{figure}
                \column{.4\textwidth}
                    \begin{itemize}
                        \item   $\mathcal{K}$: block length
                        \item   $\mathcal{H}$: hop length
                        \item   $n$: block index
                        \item   $i$: sample index
                    \end{itemize}
            \end{columns}
            \vspace{-2mm}
            \only<2>{
            \begin{itemize}
                \item   \textbf{reasons}:			
                    \begin{itemize}
                        \item	quasi-stationary signal properties
                        \item	internal block-based processing
                        \item	audio hardware characteristics (real-time systems)
                        \item	efficiency (memory allocation, SIMD)
                    \end{itemize}
            \end{itemize}}
            \only<3>{
            \begin{itemize}
                \item   \textbf{block boundaries}:
                    \begin{eqnarray*}
                        i_{\mathrm{s}}(n)	&=& i_{\mathrm{s}}(n-1) + \mathcal{H}\\
                        i_{\mathrm{e}}(n)		&=& i_{\mathrm{s}}(n) + \mathcal{K} - 1
                    \end{eqnarray*}
                \item   \textbf{overlap ratio}:
                    \begin{equation*}
                        o_{\mathrm{r}}	= \frac {\mathcal{K}-\mathcal{H}}{\mathcal{K}}
                    \end{equation*}
            \end{itemize}}
            \vspace{50mm}
        \end{frame}	

    \section[correlation]{correlation}
        \begin{frame}{correlation function}{introduction}
            \textbf{correlation function}: compute similarity between two \textit{stationary} signals $x$,$y$
            \begin{equation}
                r_\mathrm{xy}(\tau)=\mathcal{E}\lbrace x(t)y(t+\tau)\rbrace  
            \end{equation}  
            
            \begin{itemize}
                \item<2->	\textbf{continuous}:
                    \begin{equation*}
                        r_\mathrm{xy}(\tau) = \int\limits_{-\infty}^{\infty}{x(t)\cdot y(t+\tau)dt}
                    \end{equation*}
                \item<2->	\textbf{discrete}:
                    \begin{equation*}
                        r_\mathrm{xy}(\eta) = \sum\limits_{i=-\infty}^{\infty}{x(i)\cdot y(i+\eta)}
                    \end{equation*}
            \end{itemize}
        \end{frame}	

        \begin{frame}{correlation function}{animation}
            \vspace{-5mm}
            \begin{footnotesize}
                    \begin{eqnarray*}
                        r_\mathrm{xy}(\tau) &=& \int\limits_{-\infty}^{\infty}{x(t)\cdot y(t+\tau)dt}\\
                        r_\mathrm{xy}(\eta) &=& \sum\limits_{i=-\infty}^{\infty}{x(i)\cdot y(i+\eta)}
                    \end{eqnarray*}
            \end{footnotesize}
            \videowithmatlab{Correlation}
        \end{frame}

        \begin{frame}{correlation function}{examples}
            \question{draw the correlation function for}
            
            \only<2>
            {
			\begin{itemize}
				\item	rectangular window vs.
				\item	sine vs.
				\item	noise
			\end{itemize}
            }
            \pause
            \vspace{-7mm}
            \figwithmatlab{Correlation}
        \end{frame}	

        \begin{frame}{correlation function}{blocked correlation: animation}
            \videowithmatlab{BlockedCorrelation}
        \end{frame}	 

        \begin{frame}{correlation function}{normalization}
            \begin{equation*}\label{eq:corrnorm}
                \lambda_c = \frac{1}{\sqrt{\left(\sum\limits_{i=i_{\mathrm{s}}(n)}^{i_{\mathrm{e}}(n)}{x^2(i)}\right)\cdot \left(\sum\limits_{i=i_{\mathrm{s}}(n)}^{i_{\mathrm{e}}(n)}{y^2(i)}\right)}} 
            \end{equation*}
            
            \pause
            avoiding the triangular shape for blocked correlation:
            \pause
            \begin{enumerate}
                \item	modified normalization
                    \begin{equation*}
                        \lambda_c(\eta) = \frac{\mathcal{K}}{(\mathcal{K}-|\eta|)\cdot\sqrt{\left(\sum\limits_{i=i_{\mathrm{s}}(n)}^{i_{\mathrm{e}}(n)}{x^2(i)}\right)\cdot \left(\sum\limits_{i=i_{\mathrm{s}}(n)}^{i_{\mathrm{e}}(n)}{y^2(i)}\right)}} .
                    \end{equation*}
                \pause
                \item	different block lengths ($\mathcal{K},3\mathcal{K})$
                \pause
                \item	circular application
            \end{enumerate}
        \end{frame}	

        \begin{frame}{autocorrelation function}{definition \& properties}
            \toremember{correlation function with the signal itself}
            \begin{block}{autocorrelation function properties}
                \begin{itemize}
                    \item	{ACF} at lag $0$:\\
                    $r_{xx}(0,n) = 1$ if normalized, RMS otherwise
                
                    \item	maximum:\\
                    $|r_{xx}(\eta,n)| \leq r_{xx}(0,n)$ 
                    \item	symmetry:\\
                    $r_{xx}(\eta,n) = r_{xx}(-\eta,n)$
                    \item	periodicity:\\
                    The {ACF} of a periodic signal is periodic (period length of input signal)
                \end{itemize}	
            \end{block}
        \end{frame}	
        \begin{frame}{autocorrelation function}{matlab exercise}
            \matlabexercise{correlation}
            
            \begin{enumerate}
                \item   implement a Matlab function that computes the ACF for an arbitrary block length
                \item   compare results with matlab's \textsl{xcorr} function
                \item   consider only one half of the ACF and detect that highest local max that is not the absolute max
                \item   compute this ACF with overlapping blocks for the audio file \textsl{sax\_example.wav} 
                \item   plot that lag of the detected maxima over blocks and discuss the results
            \end{enumerate}
        \end{frame}
            
    \section[summary]{lecture summary}
        \begin{frame}{summary}{lecture content}
            %\begin{enumerate}
                %\item       what signals can be represented with the Fourier series?
                %\bigskip
                %\item<2->   why is the Fourier series helpful in describing signals?
                %\bigskip
                %\item<3->   what does the sampling theorem say?
                %\bigskip
                %\item<4->   can I perfectly reconstruct a quantized signal?
                %\bigskip
                %\item<5->   how does the quality of the quantization process depend on the wordlength?
                %\bigskip
                %\item<6->   why is the PDF a useful description for random signals?
            %\end{enumerate}
        \end{frame}
\end{document}

