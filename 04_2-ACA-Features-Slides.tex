% move all configuration stuff into one file so we can focus on the content
\documentclass[hyperref={pdfpagelabels=false,colorlinks=true,linkcolor=white,urlcolor=blue}]{beamer}

%%%%%%%%%%%%%%%%%%%%%%%%%%%%%%%%%%%%%%%%%%%%%%%%%%%%%%%%%%%%%%%%%%%%%%%%%%%%%%%%%%
%%%%%%%%%%%%%%%%%%%%%%%%%%%%%%%%%%%%%%%%%%%%%%%%%%%%%%%%%%%%%%%%%%%%%%%%%%%%%%%%%%
% packages
\usepackage{pict2e}
\usepackage{epic}
\usepackage{amsmath,amsfonts,amssymb}
\usepackage{units}
\usepackage{fancybox}
\usepackage[absolute,overlay]{textpos} 
\usepackage{media9} % avi2flv: "C:\Program Files\ffmpeg\bin\ffmpeg.exe" -i TuneFreqFilterbank.avi -b 600k -s 441x324 -r 15 -acodec copy TuneFreqFilterbank.flv
\usepackage{silence}
\usepackage[backend=bibtex,style=ieee]{biblatex}
\WarningFilter{biblatex}{Patching footnotes failed}
\AtEveryCitekey{\iffootnote{\tiny}{}}
\addbibresource{references}

%%%%%%%%%%%%%%%%%%%%%%%%%%%%%%%%%%%%%%%%%%%%%%%%%%%%%%%%%%%%%%%%%%%%%%%%%%%%%%%%%%
%%%%%%%%%%%%%%%%%%%%%%%%%%%%%%%%%%%%%%%%%%%%%%%%%%%%%%%%%%%%%%%%%%%%%%%%%%%%%%%%%%
% relative paths
\graphicspath{{graph/}}

%%%%%%%%%%%%%%%%%%%%%%%%%%%%%%%%%%%%%%%%%%%%%%%%%%%%%%%%%%%%%%%%%%%%%%%%%%%%%%%%%%
%%%%%%%%%%%%%%%%%%%%%%%%%%%%%%%%%%%%%%%%%%%%%%%%%%%%%%%%%%%%%%%%%%%%%%%%%%%%%%%%%%
% colors
\definecolor{gtgold}{HTML}{E0AA0F} %{rgb}{0.88,0.66,1,0.06} [234, 170, 0]/256

%%%%%%%%%%%%%%%%%%%%%%%%%%%%%%%%%%%%%%%%%%%%%%%%%%%%%%%%%%%%%%%%%%%%%%%%%%%%%%%%%%
%%%%%%%%%%%%%%%%%%%%%%%%%%%%%%%%%%%%%%%%%%%%%%%%%%%%%%%%%%%%%%%%%%%%%%%%%%%%%%%%%%
% math
\DeclareMathOperator*{\argmax}{argmax}
\DeclareMathOperator*{\argmin}{argmin}
\DeclareMathOperator*{\atan}{atan}
\DeclareMathOperator*{\arcsinh}{arcsinh}
\DeclareMathOperator*{\sign}{sign}
\DeclareMathOperator*{\tcdf}{tcdf}
\DeclareMathOperator*{\si}{sinc}
\DeclareMathOperator*{\princarg}{princarg}
\DeclareMathOperator*{\arccosh}{arccosh}
\DeclareMathOperator*{\hwr}{HWR}
\DeclareMathOperator*{\flip}{flip}
\DeclareMathOperator*{\sinc}{sinc}
\newcommand{\e}{{e}}
\newcommand{\jom}{\mathrm{j}\omega}
\newcommand{\jOm}{\mathrm{j}\Omega}
\newcommand   {\mat}[1]    		{\boldsymbol{\uppercase{#1}}}		%bold
\renewcommand {\vec}[1]    		{\boldsymbol{\lowercase{#1}}}		%bold

%%%%%%%%%%%%%%%%%%%%%%%%%%%%%%%%%%%%%%%%%%%%%%%%%%%%%%%%%%%%%%%%%%%%%%%%%%%%%%%%%%
%%%%%%%%%%%%%%%%%%%%%%%%%%%%%%%%%%%%%%%%%%%%%%%%%%%%%%%%%%%%%%%%%%%%%%%%%%%%%%%%%%
% media9
\newcommand{\includeaudio}[1]{{\includemedia[
                        addresource=audio/#1.mp3,
                        width=5mm,
                        height=5mm,
                        activate=onclick,
                        flashvars={
                            source=audio/#1.mp3  
                            &autoPlay=true
                        }]
                        {\includegraphics[width=5mm, height=5mm]{SpeakerIcon}}
                        {APlayer.swf}}}
\newcommand{\audioautoplay}[1]{{\begin{center}\includemedia[
                            addresource=audio/#1.mp3,
                            width=.1\linewidth,
                            height=.01\linewidth,
                            activate=pageopen,
                            flashvars={
                                source=audio/#1.mp3  
                                &autoPlay=true
                            }]
                            {}
                            {APlayer.swf}\end{center}}}

\newcommand{\includevideo}[1]{{\begin{center}\includemedia[
                        addresource=video/#1.mp4,
                        width=0.8\linewidth,
                        height=0.4\linewidth,
                        activate=onclick,
                        flashvars={
                            source=video/#1.mp4  
                            &autoPlay=true
                        }]
                        {}
                        {VPlayer.swf}\end{center}}}
\newcommand{\videowithmatlab}[1]{{\begin{center}\includemedia[
                        addresource=video/animate#1.mp4,
                        width=0.8\linewidth,
                        height=0.4\linewidth,
                        activate=onclick,
                        flashvars={
                            source=video/animate#1.mp4  
                            &autoPlay=true
                        }]
                        {}
                        {VPlayer.swf}\end{center}\addreference{matlab source: matlab/animate#1.m}}}
                        

%%%%%%%%%%%%%%%%%%%%%%%%%%%%%%%%%%%%%%%%%%%%%%%%%%%%%%%%%%%%%%%%%%%%%%%%%%%%%%%%%%
%%%%%%%%%%%%%%%%%%%%%%%%%%%%%%%%%%%%%%%%%%%%%%%%%%%%%%%%%%%%%%%%%%%%%%%%%%%%%%%%%%
% other commands
\newcommand{\question}[1]{\vspace{-4mm}
                          \setbeamercovered{invisible}
                          \begin{columns}
                            \column{.7\textwidth}
                                \textbf{#1}\\
                            \column{.3\textwidth}
                                \begin{flushright}
                                     \includegraphics[scale=.5]{question_mark}
                                \end{flushright}
                                \vspace{6mm}
                          \end{columns}\pause\vspace{-6mm}}

\newcommand{\toremember}[1]{\vspace{-4mm}
                          \begin{columns}
                            \column{.7\textwidth}
                                \textbf{#1}\\
                            \column{.3\textwidth}
                                \begin{flushright}
                                     \includegraphics[scale=.5]{exclamation_mark}
                                \end{flushright}
                                \vspace{6mm}
                          \end{columns}\vspace{-6mm}}

\newcommand{\matlabexercise}[1]{\vspace{-4mm}
                          \setbeamercovered{invisible}
                          \begin{columns}
                            \column{.7\textwidth}
                                \textbf{matlab exercise}: #1
                            \column{.3\textwidth}
                                \begin{flushright}
                                     \includegraphics[scale=.5]{logo_matlab}
                                \end{flushright}
                                \vspace{6mm}
                          \end{columns}}

\newcommand{\addreference}[1]{                    
                    \begin{textblock*}{\baselineskip }(1.15\textwidth,.5\textheight)
                        \rotatebox{90}{\tiny #1}
                    \end{textblock*}}
                    
\newcommand{\figwithmatlab}[1]{
                    \begin{figure}
                        \centering
                        \includegraphics{#1}
                        \label{fig:#1}
                    \end{figure}
                    
                    \addreference{matlab source: matlab/display#1.m}}                    
\newcommand{\figwithref}[2]{
                    \begin{figure}
                        \centering
                        \includegraphics{#1}
                        \label{fig:#1}
                    \end{figure}
                    
                    \addreference{#2}}                    

%%%%%%%%%%%%%%%%%%%%%%%%%%%%%%%%%%%%%%%%%%%%%%%%%%%%%%%%%%%%%%%%%%%%%%%%%%%%%%%%%%
%%%%%%%%%%%%%%%%%%%%%%%%%%%%%%%%%%%%%%%%%%%%%%%%%%%%%%%%%%%%%%%%%%%%%%%%%%%%%%%%%%
% units
\setlength{\unitlength}{1mm}

%%%%%%%%%%%%%%%%%%%%%%%%%%%%%%%%%%%%%%%%%%%%%%%%%%%%%%%%%%%%%%%%%%%%%%%%%%%%%%%%%%
%%%%%%%%%%%%%%%%%%%%%%%%%%%%%%%%%%%%%%%%%%%%%%%%%%%%%%%%%%%%%%%%%%%%%%%%%%%%%%%%%%
% counters
\newcounter{i}
\newcounter{iXOffset}
\newcounter{iYOffset}
\newcounter{iXBlockSize}
\newcounter{iYBlockSize}
\newcounter{iYBlockSizeDiv2}
\newcounter{iDistance}


%%%%%%%%%%%%%%%%%%%%%%%%%%%%%%%%%%%%%%%%%%%%%%%%%%%%%%%%%%%%%%%%%%%%%%%%%%%%%%%%%%
%%%%%%%%%%%%%%%%%%%%%%%%%%%%%%%%%%%%%%%%%%%%%%%%%%%%%%%%%%%%%%%%%%%%%%%%%%%%%%%%%%
% theme & layout
\usetheme{Frankfurt}
\beamertemplatenavigationsymbolsempty
%\setbeamertemplate{frametitle}[smoothbars theme]
\setbeamertemplate{frametitle}
{
    \begin{beamercolorbox}[ht=1.8em,wd=\paperwidth]{frametitle}
        \vspace{-.1em}%
        \hspace{.2em}{\strut\insertframetitle\strut}
        
        \hspace{.2em}\small\strut\insertframesubtitle\strut
        %\hfill
        %\includegraphics[height=.8cm,keepaspectratio]{CenterMusicTechnology-solid-2lines-white-CoAtag}
        
    \end{beamercolorbox}
    \begin{textblock*}{100mm}(8.3cm,.65cm)
        \includegraphics[height=.8cm,keepaspectratio]{CenterMusicTechnology-solid-2lines-white-CoAtag}
    \end{textblock*}
}

% set this to ensure bulletpoints without subsections
\usepackage{remreset}
\makeatletter
\@removefromreset{subsection}{section}
\makeatother
\setcounter{subsection}{1}

%---------------------------------------------------------------------------------
% appearance
\setbeamercolor{structure}{fg=gtgold}
\setbeamercovered{transparent} %invisible
\setbeamercolor{bibliography entry author}{fg=black}
\setbeamercolor*{bibliography entry title}{fg=black}
\setbeamercolor*{bibliography entry note}{fg=black}
%---------------------------------------------------------------------------------
% fontsize
\let\Tiny=\tiny

%%%%%%%%%%%%%%%%%%%%%%%%%%%%%%%%%%%%%%%%%%%%%%%%%%%%%%%%%%%%%%%%%%%%%%%%%%%%%%%%%%
%%%%%%%%%%%%%%%%%%%%%%%%%%%%%%%%%%%%%%%%%%%%%%%%%%%%%%%%%%%%%%%%%%%%%%%%%%%%%%%%%%
% warnings
\pdfsuppresswarningpagegroup=1

%%%%%%%%%%%%%%%%%%%%%%%%%%%%%%%%%%%%%%%%%%%%%%%%%%%%%%%%%%%%%%%%%%%%%%%%%%%%%%%%%%
%%%%%%%%%%%%%%%%%%%%%%%%%%%%%%%%%%%%%%%%%%%%%%%%%%%%%%%%%%%%%%%%%%%%%%%%%%%%%%%%%%
% title information
\title[]{MUSI-6201~---~Computational Music Analysis}   
\author[alexander lerch]{alexander lerch} 
%\institute{~}
%\date[Alexander Lerch]{}
\titlegraphic{\vspace{-16mm}\includegraphics[scale=.25]{title}}


\newcommand{\listspectralfeature}[1]{
\vspace{-5mm}
\begin{equation*}
    \input{eq/Spectral#1}
\end{equation*}
\only<2>{
    \question{discuss the feature}
    \begin{itemize}
        \item   how would you describe its 'meaning'
        \item   value of this feature for the hypothetical prototype signals
            \begin{itemize}
                \item   silence
                \item   single peak (@ $k_\mathrm{s}$)
                \item   white noise (flat magnitude spectrum)
            \end{itemize}
    \end{itemize}
}
\only<3>{
    \vspace{-3mm}
    \figwithref{FeaturesSpectral#1}{matlab source: matlab/displayFeatures.m}
}
\only<4>{
    \vspace{10mm}
    \textbf{common variants}:
    \begin{itemize}
        \item	restricted frequency range
        \item	power spectrum 
    \end{itemize}
    }
\vspace{50mm}
}                    


\subtitle{Part 4.2: Instantaneous Features~---~Spectral Shape}

%%%%%%%%%%%%%%%%%%%%%%%%%%%%%%%%%%%%%%%%%%%%%%%%%%%%%%%%%%%%%%%%%%%%%%%%%%%%
\begin{document}
    % generate title page
	

\begin{frame}
    \titlepage
    %\vspace{-5mm}
    \begin{flushright}
        \href{http://www.gtcmt.gatech.edu}{\includegraphics[height=.8cm,keepaspectratio]{logo_GTCMT_black}}
    \end{flushright}
\end{frame}


    \section[overview]{lecture overview}
        \begin{frame}{instantaneous features}{overview}
            \begin{itemize}
                \item   \textbf{text book}  
                    \begin{itemize}
                        \item   \href{http://ieeexplore.ieee.org/xpl/ebooks/bookPdfWithBanner.jsp?fileName=6331120.pdf&bkn=6266785&pdfType=chapter}{\underline{\textit{Chapter 3: Instantaneous Features} (pp.~41--63)}}
                    \end{itemize}
                \bigskip
                \item<2->   \textbf{lecture content}
                    \begin{itemize}
                        \item<2->   spectral shape and timbre
                        \item<3->   spectral features
                        \item<4->   features of general signal properties
                    \end{itemize}
            \end{itemize}
        \end{frame}

    \section[intro]{introduction}
        \begin{frame}{instantaneous features}{introduction}
            remember the flow chart of a general ACA system:
            \vspace{-2mm}
            \begin{figure}
                \begin{footnotesize}
				\begin{picture}(96,26)
					\setcounter{iXOffset}{0}
					\setcounter{iYOffset}{5}
					\setcounter{iXBlockSize}{28}
					\setcounter{iYBlockSize}{16}
					\setcounter{iYBlockSizeDiv2}{8}
					\setcounter{iDistance}{8}
	
					\addtocounter{iYOffset}{\value{iYBlockSizeDiv2}}
					\addtocounter{iYOffset}{-2}
	
					%\addtocounter{iXOffset}{-1}
					\put(\value{iXOffset}, \value{iYOffset})
						{\text{{\shortstack[c]{audio\\ signal}}}}
					\addtocounter{iXOffset}{1}
	
					\addtocounter{iYOffset}{2}
					\addtocounter{iXOffset}{\value{iDistance}}
	
					\put(\value{iXOffset}, \value{iYOffset})
						{\vector(1,0){\value{iDistance}}}
	
					\addtocounter{iXOffset}{\value{iDistance}}
					\addtocounter{iYOffset}{-\value{iYBlockSizeDiv2}}
					
					\put(\value{iXOffset}, \value{iYOffset})
						{\framebox(\value{iXBlockSize}, \value{iYBlockSize}) {\color{highlight}{\shortstack[c]{feature\\ extraction}}}}
	
					\addtocounter{iXOffset}{\value{iXBlockSize}}
					\addtocounter{iYOffset}{\value{iYBlockSizeDiv2}}
	
					\put(\value{iXOffset}, \value{iYOffset})
						{\vector(1,0){\value{iDistance}}}
	
					\addtocounter{iXOffset}{\value{iDistance}}
					\addtocounter{iYOffset}{-\value{iYBlockSizeDiv2}}
	
					\put(\value{iXOffset}, \value{iYOffset})
						{\framebox(\value{iXBlockSize}, \value{iYBlockSize}) {\shortstack[c]{decision,\\ interpretation,\\ classification,\\ inference}}}
	
					\addtocounter{iXOffset}{\value{iXBlockSize}}
					\addtocounter{iYOffset}{\value{iYBlockSizeDiv2}}
	
					\put(\value{iXOffset}, \value{iYOffset})
						{\vector(1,0){\value{iDistance}}}
	
					\addtocounter{iXOffset}{\value{iDistance}}
					\addtocounter{iYOffset}{-2}
	
					\addtocounter{iXOffset}{1}
					\put(\value{iXOffset}, \value{iYOffset})
						{\text{{\shortstack[c]{meta\\ data}}}}
					
				\end{picture}
\end{footnotesize}

            \end{figure}
        \end{frame}
        
        \begin{frame}{instantaneous features}{introduction~---~timbre 1/2}
            \begin{block}{\textbf{definition (American Standards Association)}}
                ...that attribute of sensation in terms of which a listener can judge that two sounds having the same loudness and pitch are dissimilar
            \end{block}
            \pause
            \question{What is the problem with this definition?}
            \vspace{-5mm}
            Bregman:\footfullcite{bregman_auditory_1994}
                    \begin{enumerate}
                        \item   implies that timbre \textit{only} exists for sound with pitch!
                        \item   only says that timbre \textit{is not} loudness and pitch
                    \end{enumerate}
            \pause
            
            \begin{itemize}
                \item[$\rightarrow$]   [timbre is] "\textit{...the psychoacoustician's multidimensional waste-basket category for everything that cannot be labeled pitch or loudness.}"\footfullcite{mcadams_hearing_1979}
            \end{itemize}
        \end{frame}
        
        \begin{frame}{instantaneous features}{introduction~---~timbre 2/2}
        
            timbre is
            \begin{itemize}
                \item   a function of \textbf{temporal envelope}
                    \begin{itemize}
                        \item   attack time characteristics
                        \item   amplitude modulations
                        \item   \ldots
                    \end{itemize}
                \item<2->   a function of \color<3->{gtgold}{\textbf{spectral distribution}}
                    \begin{itemize}
                        \item   spectral envelope
                        \item   number of partials
                        \item   energy distribution of partials
                        \item   \ldots
                    \end{itemize}
            \end{itemize}

            \begin{itemize}
                \item<3->[] when dealing with complex mixtures of sound, it is very difficult (maybe impossible?) to extract detailed temporal information for individual tones
                \item<4->[$\Rightarrow$] features typically focus on the \textbf{spectral shape}
            \end{itemize}
        \end{frame}

    \section[spectral shape]{spectral features}
        \begin{frame}{instantaneous features}{spectral rolloff}
            \listspectralfeature{Rolloff}
        \end{frame}
        \begin{frame}{instantaneous features}{spectral flux}
            \listspectralfeature{Flux}
        \end{frame}
        \begin{frame}{instantaneous features}{spectral centroid}
            \listspectralfeature{Centroid}
        \end{frame}
        \begin{frame}{instantaneous features}{spectral spread}
            \listspectralfeature{Spread}
        \end{frame}
        \begin{frame}{instantaneous features}{spectral decrease}
            \listspectralfeature{Decrease}
        \end{frame}
        \begin{frame}{instantaneous features}{spectral slope}
            \listspectralfeature{Slope}
        \end{frame}
        \begin{frame}{instantaneous features}{spectral skewness}
            \listspectralfeature{Skewness}
        \end{frame}
        \begin{frame}{instantaneous features}{spectral kurtosis}
            \listspectralfeature{Kurtosis}
        \end{frame}

        %MFCCs
    \section[tonalness]{tonalness features}
        \begin{frame}{instantaneous features}{spectral crest factor}
            \listspectralfeature{CrestFactor}
        \end{frame}
        \begin{frame}{instantaneous features}{spectral flatness}
            \listspectralfeature{Flatness}
        \end{frame}
        \begin{frame}{instantaneous features}{spectral tonal power ratio}
            \listspectralfeature{TonalPowerRatio}
        \end{frame}
        
        %MaxACF
        %Predictivity Ratio
        %SpectralPredictivity
        %ACFCoeffs
        %ZCR
        
    \section[summary]{lecture summary}
        \begin{frame}{summary}{lecture content}
            \begin{enumerate}
                \item       
                \smallskip
                \item<2->   
                \smallskip
                \item<3->   
                \smallskip
                \item<4->   
                \smallskip
                \item<5->   
            \end{enumerate}
        \end{frame}
\end{document}

