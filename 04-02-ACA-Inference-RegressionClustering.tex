% move all configuration stuff into includes file so we can focus on the content
\documentclass[aspectratio=169,hyperref={pdfpagelabels=false,colorlinks=true,linkcolor=white,urlcolor=blue},t]{beamer}

%%%%%%%%%%%%%%%%%%%%%%%%%%%%%%%%%%%%%%%%%%%%%%%%%%%%%%%%%%%%%%%%%%%%%%%%%%%%%%%%%%
%%%%%%%%%%%%%%%%%%%%%%%%%%%%%%%%%%%%%%%%%%%%%%%%%%%%%%%%%%%%%%%%%%%%%%%%%%%%%%%%%%
% packages
\usepackage{pict2e}
\usepackage{epic}
\usepackage{amsmath,amsfonts,amssymb}
\usepackage{units}
\usepackage{fancybox}
\usepackage[absolute,overlay]{textpos} 
\usepackage{media9} % avi2flv: "C:\Program Files\ffmpeg\bin\ffmpeg.exe" -i TuneFreqFilterbank.avi -b 600k -s 441x324 -r 15 -acodec copy TuneFreqFilterbank.flv
\usepackage{animate}
\usepackage{gensymb}
\usepackage{multirow}
\usepackage{silence}
\usepackage{tikz}
\usepackage[backend=bibtex,style=ieee]{biblatex}
\AtEveryCitekey{\iffootnote{\tiny}{}}
\addbibresource{include/references}

%%%%%%%%%%%%%%%%%%%%%%%%%%%%%%%%%%%%%%%%%%%%%%%%%%%%%%%%%%%%%%%%%%%%%%%%%%%%%%%%%%
%%%%%%%%%%%%%%%%%%%%%%%%%%%%%%%%%%%%%%%%%%%%%%%%%%%%%%%%%%%%%%%%%%%%%%%%%%%%%%%%%%
% relative paths
\graphicspath{{graph/}}


%%%%%%%%%%%%%%%%%%%%%%%%%%%%%%%%%%%%%%%%%%%%%%%%%%%%%%%%%%%%%%%%%%%%%%%%%%%%%%%%%%
%%%%%%%%%%%%%%%%%%%%%%%%%%%%%%%%%%%%%%%%%%%%%%%%%%%%%%%%%%%%%%%%%%%%%%%%%%%%%%%%%%
% units
\setlength{\unitlength}{1mm}

%%%%%%%%%%%%%%%%%%%%%%%%%%%%%%%%%%%%%%%%%%%%%%%%%%%%%%%%%%%%%%%%%%%%%%%%%%%%%%%%%%
%%%%%%%%%%%%%%%%%%%%%%%%%%%%%%%%%%%%%%%%%%%%%%%%%%%%%%%%%%%%%%%%%%%%%%%%%%%%%%%%%%
% theme & layout
\usetheme{Frankfurt}
\beamertemplatenavigationsymbolsempty
%\setbeamertemplate{frametitle}[smoothbars theme]
\setbeamertemplate{frametitle}
{
    \begin{beamercolorbox}[ht=1.8em,wd=\paperwidth]{frametitle}
        \vspace{-.1em}%
        \hspace{.2em}{\strut\insertframetitle\strut}
        
        \hspace{.2em}\small\strut\insertframesubtitle\strut
        %\hfill
        %\includegraphics[height=.8cm,keepaspectratio]{CenterMusicTechnology-solid-2lines-white-CoAtag}
        
    \end{beamercolorbox}
    \begin{textblock*}{100mm}(11.6cm,.7cm)
        \includegraphics[height=.8cm,keepaspectratio]{logo_GTCMT_black}
    \end{textblock*}
}

% set this to ensure bulletpoints without subsections
\usepackage{remreset}
\makeatletter
\@removefromreset{subsection}{section}
\makeatother
\setcounter{subsection}{1}

%---------------------------------------------------------------------------------
% appearance
\setbeamercolor{structure}{fg=gtgold}
\setbeamercovered{transparent} %invisible
\setbeamercolor{bibliography entry author}{fg=black}
\setbeamercolor*{bibliography entry title}{fg=black}
\setbeamercolor*{bibliography entry note}{fg=black}
\setbeamercolor{frametitle}{fg=black}
\setbeamercolor{title}{fg=black}

%\usepackage{pgfpages}
%\setbeameroption{show notes}
%\setbeameroption{show notes on second screen=right}
%---------------------------------------------------------------------------------
% fontsize
\let\Tiny=\tiny

%%%%%%%%%%%%%%%%%%%%%%%%%%%%%%%%%%%%%%%%%%%%%%%%%%%%%%%%%%%%%%%%%%%%%%%%%%%%%%%%%%
%%%%%%%%%%%%%%%%%%%%%%%%%%%%%%%%%%%%%%%%%%%%%%%%%%%%%%%%%%%%%%%%%%%%%%%%%%%%%%%%%%
% warnings
\pdfsuppresswarningpagegroup=1
\WarningFilter{biblatex}{Patching footnotes failed}
\WarningFilter{latexfont}{Font shape}
\WarningFilter{latexfont}{Some font shapes}
\WarningFilter{gensymb}{Not defining}


%%%%%%%%%%%%%%%%%%%%%%%%%%%%%%%%%%%%%%%%%%%%%%%%%%%%%%%%%%%%%%%%%%%%%%%%%%%%%%%%%%
%%%%%%%%%%%%%%%%%%%%%%%%%%%%%%%%%%%%%%%%%%%%%%%%%%%%%%%%%%%%%%%%%%%%%%%%%%%%%%%%%%
% theme & layout
\usetheme{Frankfurt}
\useinnertheme{rectangles}


%%%%%%%%%%%%%%%%%%%%%%%%%%%%%%%%%%%%%%%%%%%%%%%%%%%%%%%%%%%%%%%%%%%%%%%%%%%%%%%%%%
\setbeamertemplate{frametitle}[default][colsep=-4bp,rounded=false,shadow=false]
\setbeamertemplate{frametitle}
{%
    \nointerlineskip%
    %\vskip-0.5ex
    \begin{beamercolorbox}[wd=\paperwidth,ht=3.5ex,dp=0.6ex]{frametitle}
        \hspace*{1.3ex}\insertframetitle%
        
        \hspace*{1.3ex}\small\insertframesubtitle%
    \end{beamercolorbox}%
    \begin{textblock*}{100mm}(11.6cm,.57cm)
        \includegraphics[height=.8cm,keepaspectratio]{graph/Logo_GTCMT_white}
    \end{textblock*}
}


%%%%%%%%%%%%%%%%%%%%%%%%%%%%%%%%%%%%%%%%%%%%%%%%%%%%%%%%%%%%%%%%%%%%%%%%%%%%%%%%%%
\setbeamertemplate{title page}[default][colsep=-4bp,rounded=false,shadow=false]
\setbeamertemplate{title page}
{
    %\nointerlineskip%
    \vskip-10ex
    \begin{beamercolorbox}[wd=\paperwidth,ht=.7\paperheight,dp=0.6ex]{frametitle} %35ex
        \hspace*{1.8ex}\LARGE\inserttitle%
        
        \vspace*{.5ex}
        
        \hspace*{1.3ex}\small\insertsubtitle%
        
        \vspace*{.5ex}
    \end{beamercolorbox}%
    \nointerlineskip%
    \begin{beamercolorbox}[wd=\paperwidth,ht=.4\paperheight,dp=0.6ex]{page number in head/foot}
        %\vspace*{-.5ex}
        \hspace*{1.7ex}\small\insertauthor%
        
        %\hspace*{1.7ex}\small }%
        
        \vspace*{10ex}
        
        \begin{flushright}
            \href{http://www.gtcmt.gatech.edu}{\includegraphics[height=.8cm,keepaspectratio]{graph/Logo_GTCMT_black}}\hspace*{2ex}
        \end{flushright}
    \end{beamercolorbox}%
}


%%%%%%%%%%%%%%%%%%%%%%%%%%%%%%%%%%%%%%%%%%%%%%%%%%%%%%%%%%%%%%%%%%%%%%%%%%%%%%%%%%
%\makeatother
\setbeamertemplate{footline}
{
  \leavevmode%
  \hbox{%
  \begin{beamercolorbox}[wd=.5\paperwidth,ht=2.25ex,dp=1ex,left,leftskip=1ex]{page number in head/foot}%
    \insertsubtitle
  \end{beamercolorbox}%
  \begin{beamercolorbox}[wd=.5\paperwidth,ht=2.25ex,dp=1ex,right,rightskip=1ex]{page number in head/foot}%
    \hfill
    \insertframenumber{} / \inserttotalframenumber
  \end{beamercolorbox}}%
  \vskip0pt%
}
%\makeatletter


%%%%%%%%%%%%%%%%%%%%%%%%%%%%%%%%%%%%%%%%%%%%%%%%%%%%%%%%%%%%%%%%%%%%%%%%%%%%%%%%%%
\beamertemplatenavigationsymbolsempty
\setbeamertemplate{navigation symbols}{}
\setbeamertemplate{blocks}[default]%[rounded=false,shadow=false]
\setbeamertemplate{itemize item}[square]
\setbeamertemplate{itemize subitem}[circle]
\setbeamertemplate{itemize subsubitem}[triangle]
\setbeamertemplate{enumerate item}[square]
\setbeamertemplate{enumerate subitem}[circle]
\setbeamertemplate{enumerate subsubitem}[circle]


%%%%%%%%%%%%%%%%%%%%%%%%%%%%%%%%%%%%%%%%%%%%%%%%%%%%%%%%%%%%%%%%%%%%%%%%%%%%%%%%%%
% colors
\setbeamercolor{structure}{fg=darkgray}
\setbeamercovered{transparent} %invisible
\setbeamercolor{bibliography entry author}{fg=black}
\setbeamercolor*{bibliography entry title}{fg=black}
\setbeamercolor*{bibliography entry note}{fg=black}
\setbeamercolor{frametitle}{fg=black}
\setbeamercolor{title}{fg=white}
\setbeamercolor{subtitle}{fg=white}
\setbeamercolor{frametitle}{fg=white}
\setbeamercolor{framesubtitle}{fg=white}
\setbeamercolor{mini frame}{fg=white, bg=black}
\setbeamercolor{section in head/foot}{fg=white, bg=darkgray}
\setbeamercolor{page number in head/foot}{fg=black, bg=gtgold}
\setbeamercolor{item projected}{fg=white, bg=black}

%---------------------------------------------------------------------------------
%%%%%%%%%%%%%%%%%%%%%%%%%%%%%%%%%%%%%%%%%%%%%%%%%%%%%%%%%%%%%%%%%%%%%%%%%%%%%%%%%%
%%%%%%%%%%%%%%%%%%%%%%%%%%%%%%%%%%%%%%%%%%%%%%%%%%%%%%%%%%%%%%%%%%%%%%%%%%%%%%%%%%
% title information
\title[]{Introduction to \textbf{Audio Content Analysis}}   
\author[alexander lerch]{alexander lerch} 
%\institute{~}
%\date[Alexander Lerch]{}
%\titlegraphic{\vspace{-16mm}\includegraphics[width=\textwidth,height=3cm]{title}}

%%%%%%%%%%%%%%%%%%%%%%%%%%%%%%%%%%%%%%%%%%%%%%%%%%%%%%%%%%%%%%%%%%%%%%%%%%%%%%%%%%
%%%%%%%%%%%%%%%%%%%%%%%%%%%%%%%%%%%%%%%%%%%%%%%%%%%%%%%%%%%%%%%%%%%%%%%%%%%%%%%%%%
% colors
\definecolor{gtgold}{HTML}{96caff} %0e7eed {rgb}{0.88,0.66,1,0.06} [234, 170, 0]/256
\definecolor{darkgray}{rgb}{.1, .1, .25}
\definecolor{lightblue}{HTML}{0e7eed}
\definecolor{highlight}{rgb}{0, 0, 1} %_less!40

%%%%%%%%%%%%%%%%%%%%%%%%%%%%%%%%%%%%%%%%%%%%%%%%%%%%%%%%%%%%%%%%%%%%%%%%%%%%%%%%%%
%%%%%%%%%%%%%%%%%%%%%%%%%%%%%%%%%%%%%%%%%%%%%%%%%%%%%%%%%%%%%%%%%%%%%%%%%%%%%%%%%%
% relative paths
\graphicspath{{../ACA-Plots/graph/}}


%%%%%%%%%%%%%%%%%%%%%%%%%%%%%%%%%%%%%%%%%%%%%%%%%%%%%%%%%%%%%%%%%%%%%%%%%%%%%%%%%%
%%%%%%%%%%%%%%%%%%%%%%%%%%%%%%%%%%%%%%%%%%%%%%%%%%%%%%%%%%%%%%%%%%%%%%%%%%%%%%%%%%
% units
\setlength{\unitlength}{1mm}

%%%%%%%%%%%%%%%%%%%%%%%%%%%%%%%%%%%%%%%%%%%%%%%%%%%%%%%%%%%%%%%%%%%%%%%%%%%%%%%%%%
%%%%%%%%%%%%%%%%%%%%%%%%%%%%%%%%%%%%%%%%%%%%%%%%%%%%%%%%%%%%%%%%%%%%%%%%%%%%%%%%%%
% math
\DeclareMathOperator*{\argmax}{argmax}
\DeclareMathOperator*{\argmin}{argmin}
\DeclareMathOperator*{\atan}{atan}
\DeclareMathOperator*{\arcsinh}{arcsinh}
\DeclareMathOperator*{\sign}{sign}
\DeclareMathOperator*{\tcdf}{tcdf}
\DeclareMathOperator*{\si}{sinc}
\DeclareMathOperator*{\princarg}{princarg}
\DeclareMathOperator*{\arccosh}{arccosh}
\DeclareMathOperator*{\hwr}{HWR}
\DeclareMathOperator*{\flip}{flip}
\DeclareMathOperator*{\sinc}{sinc}
\DeclareMathOperator*{\floor}{floor}
\newcommand{\e}{{e}}
\newcommand{\jom}{\mathrm{j}\omega}
\newcommand{\jOm}{\mathrm{j}\Omega}
\newcommand   {\mat}[1]    		{\boldsymbol{\uppercase{#1}}}		%bold
\renewcommand {\vec}[1]    		{\boldsymbol{\lowercase{#1}}}		%bold

%%%%%%%%%%%%%%%%%%%%%%%%%%%%%%%%%%%%%%%%%%%%%%%%%%%%%%%%%%%%%%%%%%%%%%%%%%%%%%%%%%
%%%%%%%%%%%%%%%%%%%%%%%%%%%%%%%%%%%%%%%%%%%%%%%%%%%%%%%%%%%%%%%%%%%%%%%%%%%%%%%%%%
% media9
\newcommand{\includeaudio}[1]{
\href{run:audio/#1.mp3}{\includegraphics[width=5mm, height=5mm]{graph/SpeakerIcon}}}

\newcommand{\includeanimation}[4]{{\begin{center}
                        \animategraphics[autoplay,loop,scale=.7]{#4}{animation/#1-}{#2}{#3}        
                        \end{center}
                        \addreference{matlab source: \href{https://github.com/alexanderlerch/ACA-Plots/blob/master/matlab/animate#1.m}{matlab/animate#1.m}}}
                        \inserticon{video}}
                        
%%%%%%%%%%%%%%%%%%%%%%%%%%%%%%%%%%%%%%%%%%%%%%%%%%%%%%%%%%%%%%%%%%%%%%%%%%%%%%%%%%
%%%%%%%%%%%%%%%%%%%%%%%%%%%%%%%%%%%%%%%%%%%%%%%%%%%%%%%%%%%%%%%%%%%%%%%%%%%%%%%%%%
% other commands
\newcommand{\question}[1]{%\vspace{-4mm}
                          \setbeamercovered{invisible}
                          \begin{columns}[T]
                            \column{.9\textwidth}
                                \textbf{#1}
                            \column{.1\textwidth}
                                \vspace{-8mm}
                                \begin{flushright}
                                     \includegraphics[width=.9\columnwidth]{graph/question_mark}
                                \end{flushright}
                                \vspace{6mm}
                          \end{columns}\pause\vspace{-12mm}}

\newcommand{\toremember}[1]{
                        \inserticon{lightbulb}
                        }

\newcommand{\matlabexercise}[1]{%\vspace{-4mm}
                          \setbeamercovered{invisible}
                          \begin{columns}[T]
                            \column{.8\textwidth}
                                \textbf{matlab exercise}: #1
                            \column{.2\textwidth}
                                \begin{flushright}
                                     \includegraphics[scale=.5]{graph/logo_matlab}
                                \end{flushright}
                                %\vspace{6mm}
                          \end{columns}}

\newcommand{\addreference}[1]{  
                  
                    \begin{textblock*}{\baselineskip }(.98\paperwidth,.5\textheight) %(1.15\textwidth,.4\textheight)
                         \begin{minipage}[b][.5\paperheight][b]{1cm}%
                            \vfill%
                             \rotatebox{90}{\tiny {#1}}
                        \end{minipage}
                   \end{textblock*}
                    }
                    
\newcommand{\figwithmatlab}[1]{
                    \begin{figure}
                        \centering
                        \includegraphics[scale=.7]{#1}
                        %\label{fig:#1}
                    \end{figure}
                    
                    \addreference{matlab source: \href{https://github.com/alexanderlerch/ACA-Plots/blob/main/matlab/plot#1.m}{plot#1.m}}}
\newcommand{\figwithref}[2]{
                    \begin{figure}
                        \centering
                        \includegraphics[scale=.7]{#1}
                        \label{fig:#1}
                    \end{figure}
                    
                    \addreference{#2}}  
                                    
\newcommand{\inserticon}[1]{
                    \begin{textblock*}{100mm}(14.5cm,7.5cm)
                        \includegraphics[height=.8cm,keepaspectratio]{graph/#1}
                    \end{textblock*}}            

%%%%%%%%%%%%%%%%%%%%%%%%%%%%%%%%%%%%%%%%%%%%%%%%%%%%%%%%%%%%%%%%%%%%%%%%%%%%%%%%%%
%%%%%%%%%%%%%%%%%%%%%%%%%%%%%%%%%%%%%%%%%%%%%%%%%%%%%%%%%%%%%%%%%%%%%%%%%%%%%%%%%%
% counters
\newcounter{i}
\newcounter{j}
\newcounter{iXOffset}
\newcounter{iYOffset}
\newcounter{iXBlockSize}
\newcounter{iYBlockSize}
\newcounter{iYBlockSizeDiv2}
\newcounter{iXBlockSizeDiv2}
\newcounter{iDistance}

\newcommand{\IEEELink}{https://ieeexplore.ieee.org/servlet/opac?bknumber=9965970}




\subtitle{module 4.2: regression \& clustering}

%%%%%%%%%%%%%%%%%%%%%%%%%%%%%%%%%%%%%%%%%%%%%%%%%%%%%%%%%%%%%%%%%%%%%%%%%%%%
\begin{document}
    % generate title page
	

\begin{frame}
    \titlepage
    %\vspace{-5mm}
    \begin{flushright}
        \href{http://www.gtcmt.gatech.edu}{\includegraphics[height=.8cm,keepaspectratio]{logo_GTCMT_black}}
    \end{flushright}
\end{frame}


    \section[overview]{lecture overview}
        \begin{frame}{introduction}{overview}
            \begin{block}{corresponding textbook section}
                    %\href{http://ieeexplore.ieee.org/xpl/articleDetails.jsp?arnumber=6331125}{Chapter 8: Musical Genre, Similarity, and Mood} (pp.~155)
                    sections~4.2 -- 4.4
            \end{block}

            \begin{itemize}
                \item   \textbf{lecture content}
                    \begin{itemize}
                        \item   regression: non-categorical data analysis
                        \item   clustering: unsupervised data analysis
                    \end{itemize}
                \bigskip
                \item<2->   \textbf{learning objectives}
                    \begin{itemize}
                        \item   describe the basic principles of data-driven machine learning approaches
                        \item   implement linear regression in Python
                        \item   implement kMeans clustering in Python 
                    \end{itemize}
            \end{itemize}
            \inserticon{directions}
        \end{frame}

    \section[intro]{introduction}
        \begin{frame}{regression}{introduction}
            remember the flow chart of a general ACA system:
            \vspace{-3mm}
            \begin{figure}
                \input{pict/introduction_ACASystem_3}
            \end{figure}
            \begin{itemize}
                \item<2->   \textit{classification}: 
                    \begin{itemize}
                        \item   assign class labels to data
                    \end{itemize}
                \item<2->   \color<3->{highlight}{\textit{regression}}:
                    \begin{itemize}
                        \item	estimate numerical labels for data
                    \end{itemize}
                \item<2->   \color<3->{highlight}{\textit{clustering}}:
                    \begin{itemize}
                        \item	find grouping patterns in data
                    \end{itemize}
            \end{itemize}
        \end{frame}
        
    \section{regression}
        \begin{frame}{regression}{introduction}
             \vspace{-3mm}
             \begin{itemize}
                 \item  given a set of pairs of data and corresponding output observations
                \item   find model that maps input to output
                \bigskip
                \item<2-> model can then be used to predict (continuous value) output for an unknown new input
             \end{itemize}
        \end{frame}
        
        \begin{frame}{regression}{linear regression}
             \vspace{-3mm}
             \begin{columns}
                \column{.4\linewidth} 
                   \begin{itemize}
                        \item   estimate the slope $m$ and offset $b$ of a straight line that fits the data best:
                            \begin{footnotesize}
                            \begin{equation*}
                                \hat{y}(r) = m\cdot v(r) + b
                            \end{equation*} 
                        \end{footnotesize}
                        \bigskip
                        \item   minimizing the mean squared error leads to:
                            \begin{footnotesize}
                            \begin{eqnarray*}
                                b &=& \mu_y - m\cdot \mu_v \\
                                m &=& \frac{\sum\limits_{r=0}^{\mathcal{R}-1}\left(y(r)- \mu_y\right)\cdot \left(v(r) - \mu_v\right)}{\sum\limits_{r=0}^{\mathcal{R}-1}\left(v(r) - \mu_v\right)^2}
                            \end{eqnarray*}
                        \end{footnotesize}
                    \end{itemize}
                \column{.6\linewidth} 
                \vspace{-10mm}
                    \figwithmatlab{LinearRegression}
            \end{columns}
        \end{frame}
        
    \section{clustering}
        \begin{frame}{clustering}{introduction}
             \begin{itemize}
                \item   clustering is usually unsupervised and exploratory
                \item  group observations 
                    \begin{itemize}
                        \item 'similar' observations are grouped together
                        \item   'dissimilar' observations are in different groups
                    \end{itemize}
                \item   depends on definition of 'similarity'/ distance
             \end{itemize}
            
            \begin{figure}%
                \centering
                %\begin{footnotesize}
    \begin{picture}(80,20)
        \setcounter{iXOffset}{0}
        \setcounter{iYOffset}{4}
        \setcounter{iXBlockSize}{70}
        \setcounter{iXBlockSizeDiv2}{34}
        \setcounter{iYBlockSize}{4}
        \setcounter{iDistance}{8}

        \put(\value{iXOffset}, \value{iYOffset})
            {\text{{\shortstack[c]{\textcolor{blue}{................................}\textcolor{black}{................................}}}}}
        
        \addtocounter{iYOffset}{\value{iDistance}}
        \put(\value{iXOffset}, \value{iYOffset})
            {\text{{\shortstack[c]{\textcolor{blue}{................................}\textcolor{black}{................................}}}}}

        \addtocounter{iYOffset}{-\value{iDistance}}
        \addtocounter{iXOffset}{\value{iXBlockSizeDiv2}}
        \only<2,4>{
        \put(\value{iXOffset}, \value{iYOffset})
            {\oval(\value{iXBlockSize}, \value{iYBlockSize})}
        }
        
        \addtocounter{iYOffset}{\value{iDistance}}
        \only<2,4>{
        \put(\value{iXOffset}, \value{iYOffset})
            {\oval(\value{iXBlockSize}, \value{iYBlockSize})}
        }
        
        \addtocounter{iXOffset}{36}
        \addtocounter{iYOffset}{-1}
        \only<2,4>{
        \put(\value{iXOffset}, \value{iYOffset})
            {\text{\footnotesize Cluster A1}}
        }
        \addtocounter{iYOffset}{-\value{iDistance}}
        \only<2,4>{
        \put(\value{iXOffset}, \value{iYOffset})
            {\text{\footnotesize Cluster A2}}
        }
        \addtocounter{iYOffset}{1}
        \addtocounter{iYOffset}{\value{iDistance}}

        \addtocounter{iYOffset}{-\value{iYBlockSize}}
        \addtocounter{iXOffset}{-17}
        \only<3,4>{
        \put(16.5, \value{iYOffset})
            {\oval(36, 18)}
        }

        \addtocounter{iXOffset}{\value{iXBlockSizeDiv2}}
        \addtocounter{iXOffset}{1}
        \only<3,4>{
        \put(52, \value{iYOffset})
            {\oval(36, 18)}

        \put(10,18)
            {\text{\footnotesize Cluster B1}}
        \put(44, 18)
            {\text{\footnotesize Cluster B2}}
        }
    \end{picture}
%\end{footnotesize}
	
            \end{figure}
        \end{frame}
        
        \begin{frame}{clustering}{kMeans clustering}
             \only<1>{
            \vspace{-5mm}
             \begin{columns}
                \column{.33\linewidth} 
                    \begin{enumerate}
                        \item	\emph{Initialization}: randomly select $K$ observations from the data set as initialization.
                        \item	\emph{Update}: compute the mean for each cluster.
                        \item	\emph{Assignment}: assign each observation to the cluster with the mean of the closest cluster.
                        \item	\emph{Iteration}: go to step $2$ until the clusters converge.
                    \end{enumerate}
                \column{.67\linewidth} 
                %\vspace{-10mm}
                \includeanimation
                    {Kmeans}
                    {00}
                    {04}
                    {.5}
                    %
            \end{columns}
            }
            \only<2>{\figwithmatlab{Kmeans}}
        \end{frame}

    \section{distances}
        \begin{frame}{distances}{overview }
            \vspace{-5mm}
             \begin{columns}
                \column{.5\linewidth} 
                    \begin{itemize}
                        \item<1->	\emph{\only<1>{\textcolor{blue}}{Euclidean Distance}} (L2 Distance)
                        \smallskip
                        \item<2->	\emph{\only<2>{\textcolor{blue}}{Manhattan Distance}} (L1 Distance)
                        \smallskip
                        \item<3->	\emph{\only<3>{\textcolor{blue}}{Cosine Similarity/Distance}}
                                \begin{itemize}
                                    \item   range is from $[-1;1]$ ($[0;1]$ for non-negative input),
                                    \item   not distance but similarity measure
                                    \item   independent of vector length, only on angle
                                \end{itemize}
                        \smallskip
                        \item<4->	\emph{\only<4>{\textcolor{blue}}{Kullback-Leibler Divergence}}
                                \begin{itemize}
                                    \item   not symmetric: $d_\mathrm{KL}(\vec{v}_\mathrm{a},\vec{v}_\mathrm{b})\neq d_\mathrm{KL}(\vec{v}_\mathrm{b},\vec{v}_\mathrm{a})$,
                                    \item   designed to measure distance between probability distributions
                                \end{itemize}
                            \end{itemize}
                \column{.5\linewidth} 
                \only<1>{
                    \begin{footnotesize}\begin{equation*}\label{eq:dist_eucl}
                        d_\mathrm{EU}(\vec{v}_\mathrm{a},\vec{v}_\mathrm{b}) = \left\|\vec{v}_\mathrm{a}-\vec{v}_\mathrm{b}\right\|_2 = \sqrt{\sum\limits_{j = 0}^{\mathcal{J}-1}{\big(v_\mathrm{a}(j)-v_\mathrm{b}(j)\big)^2}} .
                    \end{equation*}\end{footnotesize}
                }
                \only<2>{
                    \begin{footnotesize}\begin{equation*}\label{eq:dist_manh}
                        d_\mathrm{M}(\vec{v}_\mathrm{a},\vec{v}_\mathrm{b}) = \left\|\vec{v}_\mathrm{a}-\vec{v}_\mathrm{b}\right\|_1 = \sum\limits_{j = 0}^{\mathcal{J}-1}{\big|v_\mathrm{a}(j)-v_\mathrm{b}(j)\big|} .
                    \end{equation*}\end{footnotesize}
                }
                \only<3>{
                    \begin{footnotesize}\begin{equation*}\label{eq:dist_cos}
                        s_\mathrm{C}(\vec{v}_\mathrm{a},\vec{v}_\mathrm{b}) = \frac{\sum\limits_{j = 0}^{\mathcal{J}-1}{v_\mathrm{a}(j)\cdot v_\mathrm{b}(j)}}{\sqrt{\sum\limits_{j = 0}^{\mathcal{J}-1}{v_\mathrm{a}(j)^2}}\cdot\sqrt{ \sum\limits_{j = 0}^{\mathcal{J}-1}{v_\mathrm{b}(j)^2}}} .
                    \end{equation*}\end{footnotesize}
                     \begin{footnotesize}\begin{equation*}
                        d_\mathrm{C}(\vec{v}_\mathrm{a},\vec{v}_\mathrm{b}) = 1-s_\mathrm{C}(\vec{v}_\mathrm{a},\vec{v}_\mathrm{b}) .
                    \end{equation*}\end{footnotesize}
               }
                \only<4>{
                    \begin{footnotesize}\begin{equation*}\label{eq:dist_kl}
                        d_\mathrm{KL}(\vec{v}_\mathrm{a},\vec{v}_\mathrm{b}) = \sum\limits_{j = 0}^{\mathcal{J}-1}{v_\mathrm{a}(j)\cdot\log\left(\frac{v_\mathrm{a}(j)}{v_\mathrm{b}(j)}\right)} . %-v_\mathrm{a}(j)+v_\mathrm{b}(j)
                    \end{equation*}\end{footnotesize}
                }
            \end{columns}
            
        \end{frame}

    \section{summary}
        \begin{frame}{summary}{lecture content}
            \begin{itemize}
                \item   \textbf{regression}
                    \begin{itemize}
                        \item   model to estimate numeric labels from features
                        \item   linear regression assumes model is straight line
                    \end{itemize}
                \bigskip
                \item   \textbf{clustering}
                    \begin{enumerate}
                        \item   unsupervised grouping
                        \item   feature space and distance measure determine result
                        \item   number of clusters usually has to be known
                        \item   kMeans is simple way of clustering
                    \end{enumerate}
                \bigskip
                \item   \textbf{distances}
                    \begin{itemize}
                        \item   L1 and L2 are most common distances
                        \item   not all 'distances' are consistent. a real distance
                        \begin{itemize}
                            \item   cannot be negative
                            \item   is symmetric
                        \end{itemize}
                    \end{itemize}
            \end{itemize}
            \inserticon{summary}
        \end{frame}
\end{document}
