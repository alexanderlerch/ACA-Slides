% move all configuration stuff into one file so we can focus on the content
\documentclass[hyperref={pdfpagelabels=false,colorlinks=true,linkcolor=white,urlcolor=blue}]{beamer}

%%%%%%%%%%%%%%%%%%%%%%%%%%%%%%%%%%%%%%%%%%%%%%%%%%%%%%%%%%%%%%%%%%%%%%%%%%%%%%%%%%
%%%%%%%%%%%%%%%%%%%%%%%%%%%%%%%%%%%%%%%%%%%%%%%%%%%%%%%%%%%%%%%%%%%%%%%%%%%%%%%%%%
% packages
\usepackage{pict2e}
\usepackage{epic}
\usepackage{amsmath,amsfonts,amssymb}
\usepackage{units}
\usepackage{fancybox}
\usepackage[absolute,overlay]{textpos} 
\usepackage{media9} % avi2flv: "C:\Program Files\ffmpeg\bin\ffmpeg.exe" -i TuneFreqFilterbank.avi -b 600k -s 441x324 -r 15 -acodec copy TuneFreqFilterbank.flv
\usepackage{silence}
\usepackage[backend=bibtex,style=ieee]{biblatex}
\WarningFilter{biblatex}{Patching footnotes failed}
\AtEveryCitekey{\iffootnote{\tiny}{}}
\addbibresource{references}

%%%%%%%%%%%%%%%%%%%%%%%%%%%%%%%%%%%%%%%%%%%%%%%%%%%%%%%%%%%%%%%%%%%%%%%%%%%%%%%%%%
%%%%%%%%%%%%%%%%%%%%%%%%%%%%%%%%%%%%%%%%%%%%%%%%%%%%%%%%%%%%%%%%%%%%%%%%%%%%%%%%%%
% relative paths
\graphicspath{{graph/}}

%%%%%%%%%%%%%%%%%%%%%%%%%%%%%%%%%%%%%%%%%%%%%%%%%%%%%%%%%%%%%%%%%%%%%%%%%%%%%%%%%%
%%%%%%%%%%%%%%%%%%%%%%%%%%%%%%%%%%%%%%%%%%%%%%%%%%%%%%%%%%%%%%%%%%%%%%%%%%%%%%%%%%
% colors
\definecolor{gtgold}{HTML}{E0AA0F} %{rgb}{0.88,0.66,1,0.06} [234, 170, 0]/256

%%%%%%%%%%%%%%%%%%%%%%%%%%%%%%%%%%%%%%%%%%%%%%%%%%%%%%%%%%%%%%%%%%%%%%%%%%%%%%%%%%
%%%%%%%%%%%%%%%%%%%%%%%%%%%%%%%%%%%%%%%%%%%%%%%%%%%%%%%%%%%%%%%%%%%%%%%%%%%%%%%%%%
% math
\DeclareMathOperator*{\argmax}{argmax}
\DeclareMathOperator*{\argmin}{argmin}
\DeclareMathOperator*{\atan}{atan}
\DeclareMathOperator*{\arcsinh}{arcsinh}
\DeclareMathOperator*{\sign}{sign}
\DeclareMathOperator*{\tcdf}{tcdf}
\DeclareMathOperator*{\si}{sinc}
\DeclareMathOperator*{\princarg}{princarg}
\DeclareMathOperator*{\arccosh}{arccosh}
\DeclareMathOperator*{\hwr}{HWR}
\DeclareMathOperator*{\flip}{flip}
\DeclareMathOperator*{\sinc}{sinc}
\newcommand{\e}{{e}}
\newcommand{\jom}{\mathrm{j}\omega}
\newcommand{\jOm}{\mathrm{j}\Omega}
\newcommand   {\mat}[1]    		{\boldsymbol{\uppercase{#1}}}		%bold
\renewcommand {\vec}[1]    		{\boldsymbol{\lowercase{#1}}}		%bold

%%%%%%%%%%%%%%%%%%%%%%%%%%%%%%%%%%%%%%%%%%%%%%%%%%%%%%%%%%%%%%%%%%%%%%%%%%%%%%%%%%
%%%%%%%%%%%%%%%%%%%%%%%%%%%%%%%%%%%%%%%%%%%%%%%%%%%%%%%%%%%%%%%%%%%%%%%%%%%%%%%%%%
% media9
\newcommand{\includeaudio}[1]{{\includemedia[
                        addresource=audio/#1.mp3,
                        width=5mm,
                        height=5mm,
                        activate=onclick,
                        flashvars={
                            source=audio/#1.mp3  
                            &autoPlay=true
                        }]
                        {\includegraphics[width=5mm, height=5mm]{SpeakerIcon}}
                        {APlayer.swf}}}
\newcommand{\audioautoplay}[1]{{\begin{center}\includemedia[
                            addresource=audio/#1.mp3,
                            width=.1\linewidth,
                            height=.01\linewidth,
                            activate=pageopen,
                            flashvars={
                                source=audio/#1.mp3  
                                &autoPlay=true
                            }]
                            {}
                            {APlayer.swf}\end{center}}}

\newcommand{\includevideo}[1]{{\begin{center}\includemedia[
                        addresource=video/#1.mp4,
                        width=0.8\linewidth,
                        height=0.4\linewidth,
                        activate=onclick,
                        flashvars={
                            source=video/#1.mp4  
                            &autoPlay=true
                        }]
                        {}
                        {VPlayer.swf}\end{center}}}
\newcommand{\videowithmatlab}[1]{{\begin{center}\includemedia[
                        addresource=video/animate#1.mp4,
                        width=0.8\linewidth,
                        height=0.4\linewidth,
                        activate=onclick,
                        flashvars={
                            source=video/animate#1.mp4  
                            &autoPlay=true
                        }]
                        {}
                        {VPlayer.swf}\end{center}\addreference{matlab source: matlab/animate#1.m}}}
                        

%%%%%%%%%%%%%%%%%%%%%%%%%%%%%%%%%%%%%%%%%%%%%%%%%%%%%%%%%%%%%%%%%%%%%%%%%%%%%%%%%%
%%%%%%%%%%%%%%%%%%%%%%%%%%%%%%%%%%%%%%%%%%%%%%%%%%%%%%%%%%%%%%%%%%%%%%%%%%%%%%%%%%
% other commands
\newcommand{\question}[1]{\vspace{-4mm}
                          \setbeamercovered{invisible}
                          \begin{columns}
                            \column{.7\textwidth}
                                \textbf{#1}\\
                            \column{.3\textwidth}
                                \begin{flushright}
                                     \includegraphics[scale=.5]{question_mark}
                                \end{flushright}
                                \vspace{6mm}
                          \end{columns}\pause\vspace{-6mm}}

\newcommand{\toremember}[1]{\vspace{-4mm}
                          \begin{columns}
                            \column{.7\textwidth}
                                \textbf{#1}\\
                            \column{.3\textwidth}
                                \begin{flushright}
                                     \includegraphics[scale=.5]{exclamation_mark}
                                \end{flushright}
                                \vspace{6mm}
                          \end{columns}\vspace{-6mm}}

\newcommand{\matlabexercise}[1]{\vspace{-4mm}
                          \setbeamercovered{invisible}
                          \begin{columns}
                            \column{.7\textwidth}
                                \textbf{matlab exercise}: #1
                            \column{.3\textwidth}
                                \begin{flushright}
                                     \includegraphics[scale=.5]{logo_matlab}
                                \end{flushright}
                                \vspace{6mm}
                          \end{columns}}

\newcommand{\addreference}[1]{                    
                    \begin{textblock*}{\baselineskip }(1.15\textwidth,.5\textheight)
                        \rotatebox{90}{\tiny #1}
                    \end{textblock*}}
                    
\newcommand{\figwithmatlab}[1]{
                    \begin{figure}
                        \centering
                        \includegraphics{#1}
                        \label{fig:#1}
                    \end{figure}
                    
                    \addreference{matlab source: matlab/display#1.m}}                    
\newcommand{\figwithref}[2]{
                    \begin{figure}
                        \centering
                        \includegraphics{#1}
                        \label{fig:#1}
                    \end{figure}
                    
                    \addreference{#2}}                    

%%%%%%%%%%%%%%%%%%%%%%%%%%%%%%%%%%%%%%%%%%%%%%%%%%%%%%%%%%%%%%%%%%%%%%%%%%%%%%%%%%
%%%%%%%%%%%%%%%%%%%%%%%%%%%%%%%%%%%%%%%%%%%%%%%%%%%%%%%%%%%%%%%%%%%%%%%%%%%%%%%%%%
% units
\setlength{\unitlength}{1mm}

%%%%%%%%%%%%%%%%%%%%%%%%%%%%%%%%%%%%%%%%%%%%%%%%%%%%%%%%%%%%%%%%%%%%%%%%%%%%%%%%%%
%%%%%%%%%%%%%%%%%%%%%%%%%%%%%%%%%%%%%%%%%%%%%%%%%%%%%%%%%%%%%%%%%%%%%%%%%%%%%%%%%%
% counters
\newcounter{i}
\newcounter{iXOffset}
\newcounter{iYOffset}
\newcounter{iXBlockSize}
\newcounter{iYBlockSize}
\newcounter{iYBlockSizeDiv2}
\newcounter{iDistance}


%%%%%%%%%%%%%%%%%%%%%%%%%%%%%%%%%%%%%%%%%%%%%%%%%%%%%%%%%%%%%%%%%%%%%%%%%%%%%%%%%%
%%%%%%%%%%%%%%%%%%%%%%%%%%%%%%%%%%%%%%%%%%%%%%%%%%%%%%%%%%%%%%%%%%%%%%%%%%%%%%%%%%
% theme & layout
\usetheme{Frankfurt}
\beamertemplatenavigationsymbolsempty
%\setbeamertemplate{frametitle}[smoothbars theme]
\setbeamertemplate{frametitle}
{
    \begin{beamercolorbox}[ht=1.8em,wd=\paperwidth]{frametitle}
        \vspace{-.1em}%
        \hspace{.2em}{\strut\insertframetitle\strut}
        
        \hspace{.2em}\small\strut\insertframesubtitle\strut
        %\hfill
        %\includegraphics[height=.8cm,keepaspectratio]{CenterMusicTechnology-solid-2lines-white-CoAtag}
        
    \end{beamercolorbox}
    \begin{textblock*}{100mm}(8.3cm,.65cm)
        \includegraphics[height=.8cm,keepaspectratio]{CenterMusicTechnology-solid-2lines-white-CoAtag}
    \end{textblock*}
}

% set this to ensure bulletpoints without subsections
\usepackage{remreset}
\makeatletter
\@removefromreset{subsection}{section}
\makeatother
\setcounter{subsection}{1}

%---------------------------------------------------------------------------------
% appearance
\setbeamercolor{structure}{fg=gtgold}
\setbeamercovered{transparent} %invisible
\setbeamercolor{bibliography entry author}{fg=black}
\setbeamercolor*{bibliography entry title}{fg=black}
\setbeamercolor*{bibliography entry note}{fg=black}
%---------------------------------------------------------------------------------
% fontsize
\let\Tiny=\tiny

%%%%%%%%%%%%%%%%%%%%%%%%%%%%%%%%%%%%%%%%%%%%%%%%%%%%%%%%%%%%%%%%%%%%%%%%%%%%%%%%%%
%%%%%%%%%%%%%%%%%%%%%%%%%%%%%%%%%%%%%%%%%%%%%%%%%%%%%%%%%%%%%%%%%%%%%%%%%%%%%%%%%%
% warnings
\pdfsuppresswarningpagegroup=1

%%%%%%%%%%%%%%%%%%%%%%%%%%%%%%%%%%%%%%%%%%%%%%%%%%%%%%%%%%%%%%%%%%%%%%%%%%%%%%%%%%
%%%%%%%%%%%%%%%%%%%%%%%%%%%%%%%%%%%%%%%%%%%%%%%%%%%%%%%%%%%%%%%%%%%%%%%%%%%%%%%%%%
% title information
\title[]{MUSI-6201~---~Computational Music Analysis}   
\author[alexander lerch]{alexander lerch} 
%\institute{~}
%\date[Alexander Lerch]{}
\titlegraphic{\vspace{-16mm}\includegraphics[scale=.25]{title}}


\subtitle{Part 9.1: Genre Classification}

%%%%%%%%%%%%%%%%%%%%%%%%%%%%%%%%%%%%%%%%%%%%%%%%%%%%%%%%%%%%%%%%%%%%%%%%%%%%
\begin{document}
    % generate title page
	

\begin{frame}
    \titlepage
    %\vspace{-5mm}
    \begin{flushright}
        \href{http://www.gtcmt.gatech.edu}{\includegraphics[height=.8cm,keepaspectratio]{logo_GTCMT_black}}
    \end{flushright}
\end{frame}


    \section[overview]{lecture overview}
        \begin{frame}{temporal analysis}{overview}
            \begin{itemize}
                \item   \textbf{text book}  
                    \begin{itemize}
                        \item   \href{http://ieeexplore.ieee.org/xpl/articleDetails.jsp?tp=&arnumber=6331125&}{\underline{\textit{Chapter 8: Musical Genre, Similarity, and Mood} (pp.~151--155)}}
                    \end{itemize}
                \bigskip
                \item<2->   \textbf{lecture content}
                    \begin{itemize}
                        \item<2->   definition of musical genre
                        \item<3->   typical features and feature categories
                        \item<4->   simple classifiers and basic classifier properties
                    \end{itemize}
            \end{itemize}
        \end{frame}

    \section[intro]{introduction}
        \begin{frame}{musical genre classification}{introduction}
            \begin{itemize}
                \item	one of the oldest research topics in MIR
                \item<2->	classic \textit{machine learning }task
                \item<3->	related fields:
                    \begin{itemize}
                        \item	speech-music classification
                        \item	instrument recognition
                        \item   artist identification
                        \item   music emotion recognition
                    \end{itemize}
            \end{itemize}
        \end{frame}

        \begin{frame}{musical genre classification}{applications}
            \begin{itemize}
                \item	large music databases:
                    \begin{itemize}
                        \item	annotation
                        \item	sorting, browsing, retrieving
                    \end{itemize}
                \item recommendation systems
                \item	automatic playlist generation
                \item	mashup generation
            \end{itemize}
        \end{frame}

        \begin{frame}{musical genre classification}{genre: definition}
            \question{what is \textit{musical genre}}

            \begin{itemize}
                \item   clusters of musical similarity?
                \item[$\rightarrow$]<2->             hard to answer in general, there are many\\ \textbf{systematic problems}
                    \begin{enumerate}
                        \item<4->	\textbf{non-agreement on taxonomies}
                        \item<5->   \textbf{genre label scope}: song, album, artist, piece of a song
                        \item<6->	\textbf{ill-defined genre labels}: geographic (\textit{indian music}), historic (\textit{baroque}), technical (\textit{barbershop}), instrumentation (\textit{symphonic music}), usage (\textit{christmas songs})
                        \item<7->	\textbf{taxonomy scalability}: genres and subgenres evolve over time
                        \item<8->	\textbf{non-orthogonality}: several genres for one piece of music
                    \end{enumerate}

            \end{itemize}
            
        \end{frame}
        \begin{frame}{musical genre classification}{genre: taxonomy examples}
                \vspace{-20mm}
                \begin{center}
                \scalebox{.6}
                {
                    				\begin{footnotesize}
					\begin{picture}(150,120)
						%%%%%%%%%%%%%%%%%%%%%%%%%%%%%%%%%%%%%%%%%%%
						%tzanetakis
						\setcounter{iXOffset}{0}
						\put(\value{iXOffset}, 83) {\text{{\shortstack[c]{Speech}}}}
						\put(\value{iXOffset}, 41) {\text{{\shortstack[c]{Music}}}}

						\addtocounter{iXOffset}{10} %10
						\put(\value{iXOffset},49) {\line(1,7){4}}
						\put(\value{iXOffset},48) {\line(1,6){4}}
						\put(\value{iXOffset},47) {\line(1,5){3.5}}
						\put(\value{iXOffset},46) {\line(1,4){3.5}}
						\put(\value{iXOffset},45) {\line(1,3){3.5}}
						\put(\value{iXOffset},44) {\line(1,2){3.5}}
						\put(\value{iXOffset},43) {\line(1,1){3.5}}
						\put(\value{iXOffset},42) {\line(1,-1){3.5}}
						\put(\value{iXOffset},41) {\line(1,-5){3.5}}

						\addtocounter{iXOffset}{1} %11
						\put(\value{iXOffset},85) {\line(1,1){3.5}}
						\put(\value{iXOffset},84) {\line(1,0){3}}
						\put(\value{iXOffset},83) {\line(1,-1){3.5}}
						
						%%%%%%%%%%%%%%%%%%%%%%%%%%%%%%%%%%%%%%%%%%%
						\setcounter{iXOffset}{15}
						\put(\value{iXOffset}, 87) {\text{{\shortstack[c]{Male}}}}
						\put(\value{iXOffset}, 83) {\text{{\shortstack[c]{Female}}}}
						\put(\value{iXOffset}, 79) {\text{{\shortstack[c]{Sports}}}}
						\put(\value{iXOffset}, 75) {\text{{\shortstack[c]{Disco}}}}
						\put(\value{iXOffset}, 71) {\text{{\shortstack[c]{Country}}}}
						\put(\value{iXOffset}, 67) {\text{{\shortstack[c]{Hip Hop}}}}
						\put(\value{iXOffset}, 63) {\text{{\shortstack[c]{Rock}}}}
						\put(\value{iXOffset}, 59) {\text{{\shortstack[c]{Blues}}}}
						\put(\value{iXOffset}, 55) {\text{{\shortstack[c]{Reggae}}}}
						\put(\value{iXOffset}, 51) {\text{{\shortstack[c]{Pop}}}}
						\put(\value{iXOffset}, 47) {\text{{\shortstack[c]{Metal}}}}
						\put(\value{iXOffset}, 37) {\text{{\shortstack[c]{Classical}}}}
						\put(\value{iXOffset}, 17) {\text{{\shortstack[c]{Jazz}}}}
					
						\addtocounter{iXOffset}{8} %28
						\put(\value{iXOffset},20) {\line(1,0.8){8}}
						\put(\value{iXOffset},19) {\line(1,0.5){8}}
						\put(\value{iXOffset},18) {\line(1,0.1){8}}
						\put(\value{iXOffset},17) {\line(1,-0.2){8}}
						\put(\value{iXOffset},16) {\line(1,-0.5){8}}
						\put(\value{iXOffset},15) {\line(1,-0.8){8}}
					
						\addtocounter{iXOffset}{5} %28
						\put(\value{iXOffset},39) {\line(1,1){3.5}}
						\put(\value{iXOffset},38) {\line(1,0.5){3}}
						\put(\value{iXOffset},37) {\line(1,-0.3){3.5}}
						\put(\value{iXOffset},36) {\line(1,-1){3.5}}
					
						%%%%%%%%%%%%%%%%%%%%%%%%%%%%%%%%%%%%%%%%%%%
						\setcounter{iXOffset}{33}
						\put(\value{iXOffset}, 43) {\text{{\shortstack[c]{Choir}}}}
						\put(\value{iXOffset}, 39) {\text{{\shortstack[c]{Orchestra}}}}
						\put(\value{iXOffset}, 35) {\text{{\shortstack[c]{Piano}}}}
						\put(\value{iXOffset}, 31) {\text{{\shortstack[c]{String Quartet}}}}
						\put(\value{iXOffset}, 27) {\text{{\shortstack[c]{Big Band}}}}
						\put(\value{iXOffset}, 23) {\text{{\shortstack[c]{Cool}}}}
						\put(\value{iXOffset}, 19) {\text{{\shortstack[c]{Fusion}}}}
						\put(\value{iXOffset}, 15) {\text{{\shortstack[c]{Piano}}}}
						\put(\value{iXOffset}, 11) {\text{{\shortstack[c]{Quartet}}}}
						\put(\value{iXOffset}, 7) {\text{{\shortstack[c]{Swing}}}}

						%%%%%%%%%%%%%%%%%%%%%%%%%%%%%%%%%%%%%%%%%%%
						%burred
						\setcounter{iXOffset}{80}
						\put(\value{iXOffset}, 71) {\text{{\shortstack[c]{Background}}}}
						\put(\value{iXOffset}, 63) {\text{{\shortstack[c]{Speech}}}}
						\put(\value{iXOffset}, 27) {\text{{\shortstack[c]{Music}}}}

						\addtocounter{iXOffset}{11} 
						
						\put(\value{iXOffset},65) {\line(1,0.4){8}}
						\put(\value{iXOffset},64) {\line(1,0){8}}
						\put(\value{iXOffset},63) {\line(1,-0.4){8}}

						\addtocounter{iXOffset}{-1} 
						
						\put(\value{iXOffset},29) {\line(1,1.3){8}}
						\put(\value{iXOffset},27) {\line(1,-1){8}}
						
						%%%%%%%%%%%%%%%%%%%%%%%%%%%%%%%%%%%%%%%%%%%
						\addtocounter{iXOffset}{10} %100
						\put(\value{iXOffset}, 67) {\text{{\shortstack[c]{Male}}}}
						\put(\value{iXOffset}, 63) {\text{{\shortstack[c]{Female}}}}
						\put(\value{iXOffset}, 59) {\text{{\shortstack[c]{+Background}}}}
						\put(\value{iXOffset}, 42) {\text{{\shortstack[c]{Classical}}}}
						\put(\value{iXOffset}, 16) {\text{{\shortstack[c]{Non-Classical}}}}
						
						\addtocounter{iXOffset}{15} 
						\put(\value{iXOffset},44) {\line(1,1){6}}
						\put(\value{iXOffset},42) {\line(1,-0.8){6}}

						\addtocounter{iXOffset}{5} 
						\put(\value{iXOffset},18) {\line(1,3){2}}
						\put(\value{iXOffset},17) {\line(1,-0.1){2}}
						\put(\value{iXOffset},16) {\line(1,-3){2}}
						
						
						%%%%%%%%%%%%%%%%%%%%%%%%%%%%%%%%%%%%%%%%%%%
						\addtocounter{iXOffset}{3} % 123
						\put(\value{iXOffset}, 49) {\text{{\shortstack[c]{Chamber}}}}
						\put(\value{iXOffset}, 35) {\text{{\shortstack[c]{Orchestra}}}}
						\put(\value{iXOffset}, 25) {\text{{\shortstack[c]{Rock}}}}
						\put(\value{iXOffset}, 15) {\text{{\shortstack[c]{Electro/Pop}}}}
						\put(\value{iXOffset}, 7) {\text{{\shortstack[c]{Jazz/Blues}}}}

						\addtocounter{iXOffset}{8} %131
						\put(\value{iXOffset},27) {\line(1,0.1){10}}
						\put(\value{iXOffset},25) {\line(1,-0.1){10}}

						\addtocounter{iXOffset}{6} %137
						\put(\value{iXOffset},52) {\line(1,0.8){4}}
						\put(\value{iXOffset},51) {\line(1,0.1){4}}
						\put(\value{iXOffset},50) {\line(1,-0.2){4}}
						\put(\value{iXOffset},49) {\line(1,-1){4}}
											
						\addtocounter{iXOffset}{1} %138
						\put(\value{iXOffset},37) {\line(1,0.7){4}}
						\put(\value{iXOffset},36) {\line(1,0){4}}
						\put(\value{iXOffset},35) {\line(1,-0.7){4}}
											
						\addtocounter{iXOffset}{3} %141
						\put(\value{iXOffset},17) {\line(1,1){2}}
						\put(\value{iXOffset},16) {\line(1,0){2}}
						\put(\value{iXOffset},15) {\line(1,-1){2}}
						
						%%%%%%%%%%%%%%%%%%%%%%%%%%%%%%%%%%%%%%%%%%%
						\addtocounter{iXOffset}{3}
						\put(\value{iXOffset}, 55) {\text{{\shortstack[c]{Piano}}}}
						\put(\value{iXOffset}, 51) {\text{{\shortstack[c]{Solo}}}}
						\put(\value{iXOffset}, 47) {\text{{\shortstack[c]{String Quartet}}}}
						\put(\value{iXOffset}, 43) {\text{{\shortstack[c]{Other}}}}
						\put(\value{iXOffset}, 39) {\text{{\shortstack[c]{Symphonic}}}}
						\put(\value{iXOffset}, 35) {\text{{\shortstack[c]{+Choir}}}}
						\put(\value{iXOffset}, 31) {\text{{\shortstack[c]{+Soloist}}}}
						\put(\value{iXOffset}, 27) {\text{{\shortstack[c]{Soft Rock}}}}
						\put(\value{iXOffset}, 23) {\text{{\shortstack[c]{Hard Rock}}}}
						\put(\value{iXOffset}, 19) {\text{{\shortstack[c]{Hip Hop}}}}
						\put(\value{iXOffset}, 15) {\text{{\shortstack[c]{Techno/Dance}}}}
						\put(\value{iXOffset}, 11) {\text{{\shortstack[c]{Pop}}}}
					\end{picture}
				\end{footnotesize}

                }
                \end{center}
        \end{frame}

        \begin{frame}{musical genre classification}{observations with humans}
            \begin{columns}
                \column{.5\linewidth}
                    \begin{enumerate}
                        \item   human classification far from perfect: \unit[75--90]{\%} for limited set of classes
                        \item<2-> for many genres, humans need only a fraction of a second to classify
                        \item<2->[$\Rightarrow$]	short time timbre features sufficient?
                    \end{enumerate}
                \column{.5\linewidth}
                    \begin{figure}
                        \centering
                        \only<1>{
                            \includegraphics[scale=.2]{graph/genre_human_classification}
                            }
                        \only<2>{
                            \includegraphics[scale=.15]{graph/genre_shorttime_classification}
                            }
                    \end{figure}
            \end{columns}
            \begin{flushright}plots from \footfullcite{lippens_comparison_2004},\footfullcite{gjerdingen_scanning_2008}\end{flushright}
        \end{frame}
    
    \section[MGC]{automatic musical genre clasification}

        \begin{frame}{musical genre classification}{overview}
            \input{pict/genre_flowchart.tex}
            \begin{enumerate}
                    \item	\textbf{feature extraction}
                            \begin{itemize}
                                \item 	dimensionality reduction
                                \item	meaningful representation
                            \end{itemize}
                    \bigskip
                    \item<2->	\textbf{classification}
                            \begin{itemize}
                                \item	map or convert feature to comprehensible domain
                            \end{itemize}
            \end{enumerate}
        \end{frame}

        \begin{frame}{musical genre classification}{feature categories}
            \vspace{-3mm}
            \begin{itemize}
                \item	\textbf{high level similarities}?
                    \begin{itemize}
                        \item	melody, hook lines, bass lines, harmony progression
                        \item	rhythm \& tempo
                        \item	structure
                        \item	instrumentation \& timbre
                        \item	\ldots
                    \end{itemize}
                \smallskip
                \item<2->	\textbf{technical feature categories}
                    \begin{itemize}
                        \item	tonal
                        \item	technical
                        \item	timbral
                        \item	temporal
                        \item	intensity
                    \end{itemize}
                \smallskip
                \item<3->       \textbf{extracted features should be}
                    \begin{itemize}
                        \item   extractable (not: time envelope in polyphonic signals)
                        \item   relevant (not: pitch chroma for instrument ID)
                        \item   non-redundant
                        \item   have discriminative power
                        \item   (robust to noise)
                    \end{itemize}
            \end{itemize}
        \end{frame}

        \begin{frame}{musical genre classification}{instantaneous features}
            \begin{itemize}
                \item	spectral features (\textbf{timbre}):
                
                    Spectral Centroid, MFCCs, Spectral Flux, \ldots
                \smallskip
                \item<2->	pitch features (\textbf{tonal}):
                
                    pitch chroma distribution/change, \ldots
                \smallskip
                \item<3->	rhythm features (\textbf{temporal}):
                
                    onset density, beat histogram features, \ldots
                \smallskip
                \item<4->	statistical features (\textbf{technical}):
                
                    standard deviation, skewness, zero crossings, \ldots
                \smallskip
                \item<5->	\textbf{intensity} features:
                
                    level variation, number of ``pauses'', \ldots
            \end{itemize}	
        \end{frame}

        \begin{frame}{musical genre classification}{feature extraction}
            \begin{enumerate}
                \item	extract \textbf{instantaneous features}
                        \only<1>{
                            \vspace{-5mm}
                            \begin{flushright}
                                \includegraphics[scale=.05]{graph/FeatureExtraction}
                            \end{flushright}
                            \vspace{-7mm}
                        }
                \smallskip
                \item<2->	compute \textbf{derived features} (derivative, filtered)
                \smallskip
                \item<3->	compute \textbf{long term features} \& subfeatures per texture window
                \smallskip	
                \item<4->	compute \textbf{subfeatures} per file
                \smallskip
                \item<5->   \textbf{normalize} subfeatures
                \smallskip
                \item<6->   (select or) \textbf{transform} subfeatures
                \smallskip
                \item<7->	feature vector $\rightarrow$ \textbf{classifier input}
                            \only<7->{
                            \vspace{-5mm}
                            \begin{flushright}
                                \includegraphics[scale=.5]{graph/Scatter}
                            \end{flushright}
                            }
            \end{enumerate}
            \vspace{20mm}
        \end{frame}
        \begin{frame}{musical genre classification}{long term features 1/2}
            derived from beat histogram\footfullcite{tzanetakis_musical_2002}
            \begin{figure}
                \centering
                \includegraphics[scale=.25]{graph/genre_beat_histogram}
            \end{figure}
        \end{frame}
        \begin{frame}{musical genre classification}{long term features 2/2}
            derived from pitch histogram or pitch chroma\footfullcite{tzanetakis_pitch_2002}
            \begin{figure}
                \centering
                \includegraphics[scale=.12]{graph/genre_pitchhisto}
            \end{figure}
        \end{frame}
        \begin{frame}{musical genre classification}{additional feature examples}
            \begin{itemize}
                \item	\textbf{stereo features}
                    \begin{itemize}
                        \item	mid channel energy vs.\ side channel energy
                        \item	spectral channel differences
                    \end{itemize}
                \bigskip
                \item<2->	features at \textbf{higher semantic levels}:
                    \begin{itemize}
                        \item   tempo, structure, harmonic complexity, instrumentation
                    \end{itemize}
            \end{itemize}
        \end{frame}

    \section[classifiers]{quick and dirty introduction into simple classifiers}
        \begin{frame}{musical genre classification}{classification: general steps}
            \begin{enumerate}
                \item	\textbf{define training set}: annotated results
                \smallskip
                \item<2->	\textbf{normalize} training set
                \smallskip
                \item<3->	\textbf{train} classifier
                \smallskip
                \item<4->	\textbf{evaluate} classifier with test set
                \smallskip
                \item<5->	(\textbf{adjust} classifier settings, return to 4.)
            \end{enumerate}
        \end{frame}
        \begin{frame}{musical genre classification}{classifier: rules of thumb}
            \begin{itemize}
                \item   \textbf{training set}
                    \begin{itemize}
                        \item	training set size vs.\ number of features
                            \begin{itemize}
                                \item	training set too small $\Rightarrow$ \textit{overfitting}
                                \item	feature number too large $\Rightarrow$ \textit{overfitting}
                            \end{itemize}
                        \item<2->	training set \textbf{too noisy} $\Rightarrow$ \textit{underfitting}
                        \item<3->	training set \textbf{not representative} $\Rightarrow$ \textit{bad classification performance}
                    \end{itemize}
                \item<4->   \textbf{classifier}
                    \begin{itemize}
                        \item<4->	\textbf{poor classifier} $\Rightarrow$ \textit{bad classification performance}
                            \begin{itemize}
                                \item	different classifier
                            \end{itemize}
                    \end{itemize}
                \item<5->   \textbf{features}
                    \begin{itemize}
                        \item<5->	\textbf{poor features} $\Rightarrow$ \textit{bad classification performance}
                            \begin{itemize}
                                \item	feature selection
                                \item	new, better features
                            \end{itemize}
                        \item<6->	features \textbf{not normalized} $\Rightarrow$ possibly \textit{bad classification performance}
                            \begin{itemize}
                                \item	feature range
                                \item	feature mean
                                \item	feature distribution
                            \end{itemize}
                    \end{itemize}
            \end{itemize}
        \end{frame}
        \begin{frame}{musical genre classification}{classifier: evaluation}
            \begin{itemize}
                \item	define \textbf{test set} for evaluation
                    \begin{itemize}
                        \item	test set \textit{different} from training set
                        \item	otherwise, same requirements
                    \end{itemize}
                
                \bigskip
                \item<2->	example: \textbf{$N$-fold cross validation}
                    \begin{enumerate}
                        \item<2->	split training set into $N$ parts (randomly, but preferably identical number per class)
                        \item<3->	select one part as test set
                        \item<4->	train the classifier with all observations from remaining $N-1$ parts
                        \item<5->	compute the classification rate for the test set
                        \item<6->	repeat until all $N$ parts have been tested
                        \item<7->	overall result: \textit{average} classification rate
                    \end{enumerate}
            \end{itemize}
        \end{frame}
        \begin{frame}{musical genre classification}{classifier: kNN}
            \begin{itemize}
                \item	\textbf{training}: extract reference vectors from training set (keep class labels)
                    \only<1>{\figwithref{Knn-0}{matlab source: matlab/displayKnn.m}}
                \item<2->	\textbf{classification}: extract test vector and set class to majority of $k$ nearest reference vectors
                \setcounter{i}{1}
                \setcounter{j}{2}
                \whiledo{\value{i}<5}
                {
                    \only<\value{j}>{\figwithref{Knn-\arabic{i}}{matlab source: matlab/displayKnn.m}}
                    \stepcounter{i}
                    \stepcounter{j}
                }
                \item<6->	\textbf{classifier data}: all training vectors
            \end{itemize}
        \end{frame}
        \begin{frame}{musical genre classification}{classifier: GMM}
            \begin{itemize}
                \item	\textbf{training}: build model of each class distribution as superposition of Gaussian distributions
                \item<2->	\textbf{classification}: compute output of each Gaussian and select class with highest probability
                    \only<2>{
                        \vspace{-5mm}
                        \figwithmatlab{Gmm}
                        }
                \item<3->	\textbf{classifier data}: per class per Gaussian: $\mu$ and covariance, mixture weight?
            \end{itemize}
        \end{frame}
        \begin{frame}{musical genre classification}{classifier: SVM}
            \begin{itemize}
                \item	\textbf{training}:
                    \begin{itemize}
                        \item   map features to high dimensional space
                            \figwithref{SVM}{\href{https://en.wikipedia.org/wiki/Support\_vector\_machine}{https://en.wikipedia.org/wiki/Support\_vector\_machine}}
                        \item   find separating hyperplane (linear classification) through maximum distance of support vectors (data points)
                    \end{itemize}
                \item<2->	\textbf{classification}: apply feature transform and proceed with 'linear' classification
                \item<3->	\textbf{classifier data}: support vectors, kernel, kernel parameters
            \end{itemize}
        \end{frame}

        \begin{frame}{musical genre classification}{results}
            \begin{itemize}
                \item	classification results depend on training set, test set, and number of classes
                \smallskip
                \item<2->	typical ranges: 10 classes $\Rightarrow$ 50--80\%
                \smallskip
                \item<3->	note: results vary largely between datasets
                    \begin{itemize}
                        \item   ill-defined genre boundaries
                        \item   non-uniformly distributed classes
                        \item   overfitting through songs from same album or artist
                        \item   \ldots
                    \end{itemize}
            \end{itemize}
        \end{frame}
    \section[example]{real world example}
        \begin{frame}{musical genre classification}{speech/music classification baseline example}
            \begin{enumerate}
                \item	extract features
                \smallskip
                \item   represent each file with its 2-dimensional feature vector
                \smallskip
                \item   kNN to classify unknown audio files
                \smallskip
                \item   evaluate classification performance
            \end{enumerate}
        \end{frame}

        \begin{frame}{musical genre classification}{speech/music classification example: features 1/2}
            for each audio file
            \begin{enumerate}
                \item	split input signal into (overlapping) blocks
                \item	compute 2 feature series (spectral centroid, RMS)
                \item<2->	aggregate feature series to one value each
                    \begin{itemize}
                        \item	\textit{mean} of Spectral Centroid
                            \begin{equation*}
                                \mu_\mathrm{SC} = \frac{1}{N}\sum_{\forall n}{v_\mathrm{SC}(n)}
                            \end{equation*}
                        \item	\textit{standard deviation} of RMS
                            \begin{equation*}
                                \sigma_\mathrm{RMS} = \sqrt{\frac{1}{N}\sum_{\forall n}{(v_\mathrm{RMS}(n)-\mu_\mathrm{RMS})^2}}
                            \end{equation*}
                    \end{itemize}
                \item<3->	represent each file as 2-dimensional vector
                    \begin{equation*}
                        \big(\mu_\mathrm{SC}, \sigma_\mathrm{RMS}\big)^\mathrm{T}
                    \end{equation*}
            \end{enumerate}				
        \end{frame}

        \begin{frame}{musical genre classification}{speech/music classification example: features 2/2}
            \figwithmatlab{Scatter}
        \end{frame}

        \begin{frame}{musical genre classification}{speech/music classification example: training set}
            \begin{itemize}
                \item	use \textbf{dataset} annotated as speech and music:
                    \begin{itemize}
                        \item	requirements
                            \begin{itemize}
                                \item	large compared to number of features
                                \item	representative for use case (diverse)
                            \end{itemize}
                        \item	here:
                            \begin{itemize}
                                \item	110 speech files
                                \item	119 music files
                            \end{itemize}
                    \end{itemize}
                \bigskip
                \item	extract the features for the dataset
            \end{itemize}
        \end{frame}

        %\begin{frame}{musical genre classification}{speech/music classification example: evaluation}
            %\begin{itemize}
                %\item 
                    %\textbf{problem}: 
                        %\begin{itemize}
                            %\item   classifier has to be tested with observations \textit{unknown} during training
                            %\item   since annotation is tedious, datasets are often not large enough to split in training and test set
                        %\end{itemize}
                %\bigskip
                %\item<2-> 	\textbf{solution}: \textit{N-Fold Cross Validation}
                    %\begin{enumerate}
                        %\item	split training set into $N$ parts (randomly, but identical number per class)
                        %\item<3->	select one part as test set
                        %\item<4->	train the classifier with all observations from \textit{remaining} $N-1$ parts
                        %\item<5->	compute the classification rate for the \textit{test set}
                        %\item<6->	repeat until all $N$ parts have been tested
                        %\item<7->	overall result: \textit{average} classification rate
                    %\end{enumerate}
            %\end{itemize}
        %\end{frame}

        \begin{frame}{musical genre classification}{speech/music classification example: results (kNN)}
            \begin{itemize}
                \item   \textbf{confusion matrix}:
                    \begin{table}
                        \centering
                        \begin{tabular}{l|cc|ccccccccc} %{\textwidth}{@{\extracolsep{\fill}}ccccccccccccc}
                            \bf{\emph{}}	 & \bf{\emph{speech}}	 & \bf{\emph{music}} & \# files	 \\ 
                             \hline
                            \bf{speech}	 & $\mathbf{93}$	 & $17$	 & $110$\\
                            \bf{music}	 & $19$	 & $\mathbf{100}$ & $119$
                        \end{tabular}
                    \end{table}
                \item<2->$\Rightarrow$ \textbf{classification rate}: 
                    \begin{equation*}
                        \frac{100 + 93}{110 + 119} = 84.2\%
                    \end{equation*}
                \smallskip
                \item<3->   single feature classification results
                    \begin{itemize}
                        \item	Spectral Centroid: $56.7\%$
                        \item	RMS: $85.1\%$
                    \end{itemize}
            \end{itemize}
        \end{frame}


    \section[summary]{lecture summary}
        \begin{frame}{summary}{lecture content}
            \begin{enumerate}
                \item   name three possible problems in the definition of the ground truth for genre classification
                \smallskip
                \item<2->   is it possible for genre classifiers to yield better accuracy than human experts
                \smallskip
                \item<3->   list the feature processing steps from audio to the input of the classifier
            \end{enumerate}
        \end{frame}
\end{document}

