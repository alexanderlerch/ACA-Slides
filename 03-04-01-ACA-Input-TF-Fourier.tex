% move all configuration stuff into includes file so we can focus on the content
\documentclass[aspectratio=169,hyperref={pdfpagelabels=false,colorlinks=true,linkcolor=white,urlcolor=blue},t]{beamer}

%%%%%%%%%%%%%%%%%%%%%%%%%%%%%%%%%%%%%%%%%%%%%%%%%%%%%%%%%%%%%%%%%%%%%%%%%%%%%%%%%%
%%%%%%%%%%%%%%%%%%%%%%%%%%%%%%%%%%%%%%%%%%%%%%%%%%%%%%%%%%%%%%%%%%%%%%%%%%%%%%%%%%
% packages
\usepackage{pict2e}
\usepackage{epic}
\usepackage{amsmath,amsfonts,amssymb}
\usepackage{units}
\usepackage{fancybox}
\usepackage[absolute,overlay]{textpos} 
\usepackage{media9} % avi2flv: "C:\Program Files\ffmpeg\bin\ffmpeg.exe" -i TuneFreqFilterbank.avi -b 600k -s 441x324 -r 15 -acodec copy TuneFreqFilterbank.flv
\usepackage{animate}
\usepackage{gensymb}
\usepackage{multirow}
\usepackage{silence}
\usepackage{tikz}
\usepackage[backend=bibtex,style=ieee]{biblatex}
\AtEveryCitekey{\iffootnote{\tiny}{}}
\addbibresource{include/references}

%%%%%%%%%%%%%%%%%%%%%%%%%%%%%%%%%%%%%%%%%%%%%%%%%%%%%%%%%%%%%%%%%%%%%%%%%%%%%%%%%%
%%%%%%%%%%%%%%%%%%%%%%%%%%%%%%%%%%%%%%%%%%%%%%%%%%%%%%%%%%%%%%%%%%%%%%%%%%%%%%%%%%
% relative paths
\graphicspath{{graph/}}


%%%%%%%%%%%%%%%%%%%%%%%%%%%%%%%%%%%%%%%%%%%%%%%%%%%%%%%%%%%%%%%%%%%%%%%%%%%%%%%%%%
%%%%%%%%%%%%%%%%%%%%%%%%%%%%%%%%%%%%%%%%%%%%%%%%%%%%%%%%%%%%%%%%%%%%%%%%%%%%%%%%%%
% units
\setlength{\unitlength}{1mm}

%%%%%%%%%%%%%%%%%%%%%%%%%%%%%%%%%%%%%%%%%%%%%%%%%%%%%%%%%%%%%%%%%%%%%%%%%%%%%%%%%%
%%%%%%%%%%%%%%%%%%%%%%%%%%%%%%%%%%%%%%%%%%%%%%%%%%%%%%%%%%%%%%%%%%%%%%%%%%%%%%%%%%
% theme & layout
\usetheme{Frankfurt}
\beamertemplatenavigationsymbolsempty
%\setbeamertemplate{frametitle}[smoothbars theme]
\setbeamertemplate{frametitle}
{
    \begin{beamercolorbox}[ht=1.8em,wd=\paperwidth]{frametitle}
        \vspace{-.1em}%
        \hspace{.2em}{\strut\insertframetitle\strut}
        
        \hspace{.2em}\small\strut\insertframesubtitle\strut
        %\hfill
        %\includegraphics[height=.8cm,keepaspectratio]{CenterMusicTechnology-solid-2lines-white-CoAtag}
        
    \end{beamercolorbox}
    \begin{textblock*}{100mm}(11.6cm,.7cm)
        \includegraphics[height=.8cm,keepaspectratio]{logo_GTCMT_black}
    \end{textblock*}
}

% set this to ensure bulletpoints without subsections
\usepackage{remreset}
\makeatletter
\@removefromreset{subsection}{section}
\makeatother
\setcounter{subsection}{1}

%---------------------------------------------------------------------------------
% appearance
\setbeamercolor{structure}{fg=gtgold}
\setbeamercovered{transparent} %invisible
\setbeamercolor{bibliography entry author}{fg=black}
\setbeamercolor*{bibliography entry title}{fg=black}
\setbeamercolor*{bibliography entry note}{fg=black}
\setbeamercolor{frametitle}{fg=black}
\setbeamercolor{title}{fg=black}

%\usepackage{pgfpages}
%\setbeameroption{show notes}
%\setbeameroption{show notes on second screen=right}
%---------------------------------------------------------------------------------
% fontsize
\let\Tiny=\tiny

%%%%%%%%%%%%%%%%%%%%%%%%%%%%%%%%%%%%%%%%%%%%%%%%%%%%%%%%%%%%%%%%%%%%%%%%%%%%%%%%%%
%%%%%%%%%%%%%%%%%%%%%%%%%%%%%%%%%%%%%%%%%%%%%%%%%%%%%%%%%%%%%%%%%%%%%%%%%%%%%%%%%%
% warnings
\pdfsuppresswarningpagegroup=1
\WarningFilter{biblatex}{Patching footnotes failed}
\WarningFilter{latexfont}{Font shape}
\WarningFilter{latexfont}{Some font shapes}
\WarningFilter{gensymb}{Not defining}


%%%%%%%%%%%%%%%%%%%%%%%%%%%%%%%%%%%%%%%%%%%%%%%%%%%%%%%%%%%%%%%%%%%%%%%%%%%%%%%%%%
%%%%%%%%%%%%%%%%%%%%%%%%%%%%%%%%%%%%%%%%%%%%%%%%%%%%%%%%%%%%%%%%%%%%%%%%%%%%%%%%%%
% theme & layout
\usetheme{Frankfurt}
\useinnertheme{rectangles}


%%%%%%%%%%%%%%%%%%%%%%%%%%%%%%%%%%%%%%%%%%%%%%%%%%%%%%%%%%%%%%%%%%%%%%%%%%%%%%%%%%
\setbeamertemplate{frametitle}[default][colsep=-4bp,rounded=false,shadow=false]
\setbeamertemplate{frametitle}
{%
    \nointerlineskip%
    %\vskip-0.5ex
    \begin{beamercolorbox}[wd=\paperwidth,ht=3.5ex,dp=0.6ex]{frametitle}
        \hspace*{1.3ex}\insertframetitle%
        
        \hspace*{1.3ex}\small\insertframesubtitle%
    \end{beamercolorbox}%
    \begin{textblock*}{100mm}(11.6cm,.57cm)
        \includegraphics[height=.8cm,keepaspectratio]{graph/Logo_GTCMT_white}
    \end{textblock*}
}


%%%%%%%%%%%%%%%%%%%%%%%%%%%%%%%%%%%%%%%%%%%%%%%%%%%%%%%%%%%%%%%%%%%%%%%%%%%%%%%%%%
\setbeamertemplate{title page}[default][colsep=-4bp,rounded=false,shadow=false]
\setbeamertemplate{title page}
{
    %\nointerlineskip%
    \vskip-10ex
    \begin{beamercolorbox}[wd=\paperwidth,ht=.7\paperheight,dp=0.6ex]{frametitle} %35ex
        \hspace*{1.8ex}\LARGE\inserttitle%
        
        \vspace*{.5ex}
        
        \hspace*{1.3ex}\small\insertsubtitle%
        
        \vspace*{.5ex}
    \end{beamercolorbox}%
    \nointerlineskip%
    \begin{beamercolorbox}[wd=\paperwidth,ht=.4\paperheight,dp=0.6ex]{page number in head/foot}
        %\vspace*{-.5ex}
        \hspace*{1.7ex}\small\insertauthor%
        
        %\hspace*{1.7ex}\small }%
        
        \vspace*{10ex}
        
        \begin{flushright}
            \href{http://www.gtcmt.gatech.edu}{\includegraphics[height=.8cm,keepaspectratio]{graph/Logo_GTCMT_black}}\hspace*{2ex}
        \end{flushright}
    \end{beamercolorbox}%
}


%%%%%%%%%%%%%%%%%%%%%%%%%%%%%%%%%%%%%%%%%%%%%%%%%%%%%%%%%%%%%%%%%%%%%%%%%%%%%%%%%%
%\makeatother
\setbeamertemplate{footline}
{
  \leavevmode%
  \hbox{%
  \begin{beamercolorbox}[wd=.5\paperwidth,ht=2.25ex,dp=1ex,left,leftskip=1ex]{page number in head/foot}%
    \insertsubtitle
  \end{beamercolorbox}%
  \begin{beamercolorbox}[wd=.5\paperwidth,ht=2.25ex,dp=1ex,right,rightskip=1ex]{page number in head/foot}%
    \hfill
    \insertframenumber{} / \inserttotalframenumber
  \end{beamercolorbox}}%
  \vskip0pt%
}
%\makeatletter


%%%%%%%%%%%%%%%%%%%%%%%%%%%%%%%%%%%%%%%%%%%%%%%%%%%%%%%%%%%%%%%%%%%%%%%%%%%%%%%%%%
\beamertemplatenavigationsymbolsempty
\setbeamertemplate{navigation symbols}{}
\setbeamertemplate{blocks}[default]%[rounded=false,shadow=false]
\setbeamertemplate{itemize item}[square]
\setbeamertemplate{itemize subitem}[circle]
\setbeamertemplate{itemize subsubitem}[triangle]
\setbeamertemplate{enumerate item}[square]
\setbeamertemplate{enumerate subitem}[circle]
\setbeamertemplate{enumerate subsubitem}[circle]


%%%%%%%%%%%%%%%%%%%%%%%%%%%%%%%%%%%%%%%%%%%%%%%%%%%%%%%%%%%%%%%%%%%%%%%%%%%%%%%%%%
% colors
\setbeamercolor{structure}{fg=darkgray}
\setbeamercovered{transparent} %invisible
\setbeamercolor{bibliography entry author}{fg=black}
\setbeamercolor*{bibliography entry title}{fg=black}
\setbeamercolor*{bibliography entry note}{fg=black}
\setbeamercolor{frametitle}{fg=black}
\setbeamercolor{title}{fg=white}
\setbeamercolor{subtitle}{fg=white}
\setbeamercolor{frametitle}{fg=white}
\setbeamercolor{framesubtitle}{fg=white}
\setbeamercolor{mini frame}{fg=white, bg=black}
\setbeamercolor{section in head/foot}{fg=white, bg=darkgray}
\setbeamercolor{page number in head/foot}{fg=black, bg=gtgold}
\setbeamercolor{item projected}{fg=white, bg=black}

%---------------------------------------------------------------------------------
%%%%%%%%%%%%%%%%%%%%%%%%%%%%%%%%%%%%%%%%%%%%%%%%%%%%%%%%%%%%%%%%%%%%%%%%%%%%%%%%%%
%%%%%%%%%%%%%%%%%%%%%%%%%%%%%%%%%%%%%%%%%%%%%%%%%%%%%%%%%%%%%%%%%%%%%%%%%%%%%%%%%%
% title information
\title[]{Introduction to \textbf{Audio Content Analysis}}   
\author[alexander lerch]{alexander lerch} 
%\institute{~}
%\date[Alexander Lerch]{}
%\titlegraphic{\vspace{-16mm}\includegraphics[width=\textwidth,height=3cm]{title}}

%%%%%%%%%%%%%%%%%%%%%%%%%%%%%%%%%%%%%%%%%%%%%%%%%%%%%%%%%%%%%%%%%%%%%%%%%%%%%%%%%%
%%%%%%%%%%%%%%%%%%%%%%%%%%%%%%%%%%%%%%%%%%%%%%%%%%%%%%%%%%%%%%%%%%%%%%%%%%%%%%%%%%
% colors
\definecolor{gtgold}{HTML}{96caff} %0e7eed {rgb}{0.88,0.66,1,0.06} [234, 170, 0]/256
\definecolor{darkgray}{rgb}{.1, .1, .25}
\definecolor{lightblue}{HTML}{0e7eed}
\definecolor{highlight}{rgb}{0, 0, 1} %_less!40

%%%%%%%%%%%%%%%%%%%%%%%%%%%%%%%%%%%%%%%%%%%%%%%%%%%%%%%%%%%%%%%%%%%%%%%%%%%%%%%%%%
%%%%%%%%%%%%%%%%%%%%%%%%%%%%%%%%%%%%%%%%%%%%%%%%%%%%%%%%%%%%%%%%%%%%%%%%%%%%%%%%%%
% relative paths
\graphicspath{{../ACA-Plots/graph/}}


%%%%%%%%%%%%%%%%%%%%%%%%%%%%%%%%%%%%%%%%%%%%%%%%%%%%%%%%%%%%%%%%%%%%%%%%%%%%%%%%%%
%%%%%%%%%%%%%%%%%%%%%%%%%%%%%%%%%%%%%%%%%%%%%%%%%%%%%%%%%%%%%%%%%%%%%%%%%%%%%%%%%%
% units
\setlength{\unitlength}{1mm}

%%%%%%%%%%%%%%%%%%%%%%%%%%%%%%%%%%%%%%%%%%%%%%%%%%%%%%%%%%%%%%%%%%%%%%%%%%%%%%%%%%
%%%%%%%%%%%%%%%%%%%%%%%%%%%%%%%%%%%%%%%%%%%%%%%%%%%%%%%%%%%%%%%%%%%%%%%%%%%%%%%%%%
% math
\DeclareMathOperator*{\argmax}{argmax}
\DeclareMathOperator*{\argmin}{argmin}
\DeclareMathOperator*{\atan}{atan}
\DeclareMathOperator*{\arcsinh}{arcsinh}
\DeclareMathOperator*{\sign}{sign}
\DeclareMathOperator*{\tcdf}{tcdf}
\DeclareMathOperator*{\si}{sinc}
\DeclareMathOperator*{\princarg}{princarg}
\DeclareMathOperator*{\arccosh}{arccosh}
\DeclareMathOperator*{\hwr}{HWR}
\DeclareMathOperator*{\flip}{flip}
\DeclareMathOperator*{\sinc}{sinc}
\DeclareMathOperator*{\floor}{floor}
\newcommand{\e}{{e}}
\newcommand{\jom}{\mathrm{j}\omega}
\newcommand{\jOm}{\mathrm{j}\Omega}
\newcommand   {\mat}[1]    		{\boldsymbol{\uppercase{#1}}}		%bold
\renewcommand {\vec}[1]    		{\boldsymbol{\lowercase{#1}}}		%bold

%%%%%%%%%%%%%%%%%%%%%%%%%%%%%%%%%%%%%%%%%%%%%%%%%%%%%%%%%%%%%%%%%%%%%%%%%%%%%%%%%%
%%%%%%%%%%%%%%%%%%%%%%%%%%%%%%%%%%%%%%%%%%%%%%%%%%%%%%%%%%%%%%%%%%%%%%%%%%%%%%%%%%
% media9
\newcommand{\includeaudio}[1]{
\href{run:audio/#1.mp3}{\includegraphics[width=5mm, height=5mm]{graph/SpeakerIcon}}}

\newcommand{\includeanimation}[4]{{\begin{center}
                        \animategraphics[autoplay,loop,scale=.7]{#4}{animation/#1-}{#2}{#3}        
                        \end{center}
                        \addreference{matlab source: \href{https://github.com/alexanderlerch/ACA-Plots/blob/master/matlab/animate#1.m}{matlab/animate#1.m}}}
                        \inserticon{video}}
                        
%%%%%%%%%%%%%%%%%%%%%%%%%%%%%%%%%%%%%%%%%%%%%%%%%%%%%%%%%%%%%%%%%%%%%%%%%%%%%%%%%%
%%%%%%%%%%%%%%%%%%%%%%%%%%%%%%%%%%%%%%%%%%%%%%%%%%%%%%%%%%%%%%%%%%%%%%%%%%%%%%%%%%
% other commands
\newcommand{\question}[1]{%\vspace{-4mm}
                          \setbeamercovered{invisible}
                          \begin{columns}[T]
                            \column{.9\textwidth}
                                \textbf{#1}
                            \column{.1\textwidth}
                                \vspace{-8mm}
                                \begin{flushright}
                                     \includegraphics[width=.9\columnwidth]{graph/question_mark}
                                \end{flushright}
                                \vspace{6mm}
                          \end{columns}\pause\vspace{-12mm}}

\newcommand{\toremember}[1]{
                        \inserticon{lightbulb}
                        }

\newcommand{\matlabexercise}[1]{%\vspace{-4mm}
                          \setbeamercovered{invisible}
                          \begin{columns}[T]
                            \column{.8\textwidth}
                                \textbf{matlab exercise}: #1
                            \column{.2\textwidth}
                                \begin{flushright}
                                     \includegraphics[scale=.5]{graph/logo_matlab}
                                \end{flushright}
                                %\vspace{6mm}
                          \end{columns}}

\newcommand{\addreference}[1]{  
                  
                    \begin{textblock*}{\baselineskip }(.98\paperwidth,.5\textheight) %(1.15\textwidth,.4\textheight)
                         \begin{minipage}[b][.5\paperheight][b]{1cm}%
                            \vfill%
                             \rotatebox{90}{\tiny {#1}}
                        \end{minipage}
                   \end{textblock*}
                    }
                    
\newcommand{\figwithmatlab}[1]{
                    \begin{figure}
                        \centering
                        \includegraphics[scale=.7]{#1}
                        %\label{fig:#1}
                    \end{figure}
                    
                    \addreference{matlab source: \href{https://github.com/alexanderlerch/ACA-Plots/blob/main/matlab/plot#1.m}{plot#1.m}}}
\newcommand{\figwithref}[2]{
                    \begin{figure}
                        \centering
                        \includegraphics[scale=.7]{#1}
                        \label{fig:#1}
                    \end{figure}
                    
                    \addreference{#2}}  
                                    
\newcommand{\inserticon}[1]{
                    \begin{textblock*}{100mm}(14.5cm,7.5cm)
                        \includegraphics[height=.8cm,keepaspectratio]{graph/#1}
                    \end{textblock*}}            

%%%%%%%%%%%%%%%%%%%%%%%%%%%%%%%%%%%%%%%%%%%%%%%%%%%%%%%%%%%%%%%%%%%%%%%%%%%%%%%%%%
%%%%%%%%%%%%%%%%%%%%%%%%%%%%%%%%%%%%%%%%%%%%%%%%%%%%%%%%%%%%%%%%%%%%%%%%%%%%%%%%%%
% counters
\newcounter{i}
\newcounter{j}
\newcounter{iXOffset}
\newcounter{iYOffset}
\newcounter{iXBlockSize}
\newcounter{iYBlockSize}
\newcounter{iYBlockSizeDiv2}
\newcounter{iXBlockSizeDiv2}
\newcounter{iDistance}

\newcommand{\IEEELink}{https://ieeexplore.ieee.org/servlet/opac?bknumber=9965970}




\subtitle{Module 3.4.1: Time-Frequency Representations~---~Fourier Transform}

%%%%%%%%%%%%%%%%%%%%%%%%%%%%%%%%%%%%%%%%%%%%%%%%%%%%%%%%%%%%%%%%%%%%%%%%%%%%
\begin{document}
    % generate title page
	

\begin{frame}
    \titlepage
    %\vspace{-5mm}
    \begin{flushright}
        \href{http://www.gtcmt.gatech.edu}{\includegraphics[height=.8cm,keepaspectratio]{logo_GTCMT_black}}
    \end{flushright}
\end{frame}


    \section[overview]{lecture overview}
        \begin{frame}{introduction}{overview}
            \begin{block}{corresponding textbook section}
                    %\href{http://ieeexplore.ieee.org/xpl/articleDetails.jsp?tp=&arnumber=6331119&}{Chapter 2~---~Fundamentals}: pp.~20--23\\
                    %\href{http://ieeexplore.ieee.org/xpl/articleDetails.jsp?arnumber=6331115}{Appendix B~---~Fourier Transform}: pp.~185--197
                    Section 3.4.1\\
                    Appendix B
            \end{block}

            \begin{itemize}
                \item   \textbf{lecture content}
                    \begin{itemize}
                        \item   FT of continuous signals  
                        \item   FT properties
                        \item   FT of sampled signals
                        \item   Short Time FT (STFT)
                        \item   DFT
                    \end{itemize}
                \bigskip
                \item<2->   \textbf{learning objectives}
                    \begin{itemize}
                        \item   name and explain definition and properties of the FT
                    \end{itemize}
            \end{itemize}
            \inserticon{directions}
        \end{frame}

    \section[introduction]{introduction}
        \begin{frame}{fourier transform}{introduction}
            \figwithmatlab{FourierTransform}
            
            \begin{table}%
            \begin{tabular}{lll}
                top & time domain signal & magnitude spectrum in dB\\
                bottom & real spectrum & imaginary spectrum
            \end{tabular}
            \end{table}
            
        \end{frame}	

    \section[definition]{definition}
        \begin{frame}{fourier transform}{definition (continuous)}
            \begin {equation*}\label{eq:fourier_transformation}
                X(\jom) = \mathfrak{F}[x(t)] = \int\limits_{-\infty}^{\infty} {x(t) \e^{-\jom t}\, dt}
            \end {equation*}

            \pause
            sidenote: 
            \begin{footnotesize}
            Fourier series coefficients 
            \begin {equation*}\label{eq:fourier_coeff}
                a_k = \frac{1}{T_0}\int\limits_{-\nicefrac{T_0}{2}}^{\nicefrac{T_0}{2}} x(t) \e^{-\jom_0kt}\, dt \nonumber
            \end {equation*}
            
            \begin{itemize}
                \item	$T_0\rightarrow \infty$ to allow the analysis of aperiodic functions
                \item[$\Rightarrow$] $k\omega_0 \rightarrow \omega$
            \end{itemize}
            \end{footnotesize}
        \end{frame}	

        \begin{frame}{fourier transform}{representations}
            \vspace{-5mm}
            \begin{eqnarray*}
                X(\jom) &=& \Re[X(\jom)] + \Im[X(\jom)]\\
                &=& \underbrace{|X(\jom)|}_{\text{magnitude}} \cdot \underbrace{\Phi_\mathrm{X}(\omega)}_{\text{phase}}\\
%
                \pause
                &&\\
                &&\\
                \bigskip
                |X(\jom)|  &=& \sqrt{\Re[X(\jom)]^2 + \Im[X(\jom)]^2}\\
                \Phi_\mathrm{X}(\omega)  &=& \atan2\left(\frac{\Im[X(\jom)]}{\Re[X(\jom)]}\right)
            \end{eqnarray*}

            \bigskip
            \only<2->{\textbf{note}: }
            \begin{itemize}
                \item<2->   complex spectrum either represented as magnitude \& phase or as real \& imaginary
                \item<2->   magnitude spectrum has no negative values
            \end{itemize}
        \end{frame}	
    
    \section{properties}
        \begin{frame}{fourier transform}{property 1: invertibility}
            \begin{eqnarray*}\label{eq:ift}
                x(t) &=& \mathfrak{F}^{-1}[X(\jom)]\nonumber\\
                 &=& \frac{1}{2\pi}\int\limits_{-\infty}^{\infty} X(\jom) \e^{\jom t}\, d\omega 
            \end{eqnarray*}

            \bigskip
            \only<2->{\textbf{note}: }
            \begin{itemize}
                %\item<2->   for comparison: Fourier series equation
                    %\begin {equation*}
                        %x(t) = \sum\limits_{k=-\infty}^{\infty} a_k \e^{\jom_0 k t} \nonumber
                    %\end {equation*}
                \item<2->   time domain signal can be \textbf{perfectly reconstructed}~---~no information loss
                \item<2->   FT and IFT are very similar, largely equivalent
            \end{itemize}
        \end{frame}	

        \begin{frame}{fourier transform}{property 2: superposition}
            \begin{eqnarray*}
                y(t) &=& c_1\cdot x_1(t) + c_2\cdot x_2(t)\nonumber\\
                \mapsto&&\nonumber\\
                Y(\jom) &=& c_1\cdot X_1(\jom) + c_2\cdot X_2(\jom)\nonumber
            \end{eqnarray*}
            \bigskip
            \only<2->{\textbf{note}: }
            \begin{itemize}
                %\item<2->   for comparison: Fourier series equation
                    %\begin {equation*}
                        %x(t) = \sum\limits_{k=-\infty}^{\infty} a_k \e^{\jom_0 k t} \nonumber
                    %\end {equation*}
                \item<2->   FT is a \textit{linear }transform
            \end{itemize}
            
            %\pause
            %\begin{itemize}
                %\item[]	\textbf{derivation}
                        %\begin{footnotesize}
                            %\begin{eqnarray*}
                                %Y(\jom) &=& \int\limits_{-\infty}^{\infty} {\big(c_1\cdot x_1(t) + c_2\cdot x_2(t)\big)\cdot \e^{-\jom t}\, dt}\nonumber\\
                                %\pause
                                %&=& c_1\cdot \int\limits_{-\infty}^{\infty} {x_1(t)  \e^{-\jom t}\, dt} + c_2\cdot \int\limits_{-\infty}^{\infty} {x_2(t) \e^{-\jom t}\, dt}\nonumber\\
                                %\pause
                                %&=& c_1\cdot X_1(\jom) + c_2\cdot X_2(\jom) 
                            %\end{eqnarray*}
                        %\end{footnotesize}
            %\end{itemize}
        \end{frame}	

        \begin{frame}{fourier transform}{property 3: convolution and multiplication}
            \vspace{-5mm}
            \begin{eqnarray*}
                y(t) &=& \int_{-\infty}^{\infty} {h(\tau) \cdot x(t-\tau)\, d\tau}\\\mapsto &&\\
                Y(\jom) &=& H(\jom)\cdot X(\jom) \nonumber
            \end{eqnarray*}
            
            \bigskip
            \only<2->{\textbf{note}: }
            \begin{itemize}
                \item<2->   convolution in time domain means multiplication in frequency domain
                \item<2->   convolution in frequency domain means multiplication in time domain
            \end{itemize}
            
            %\pause
            %\vspace{-2mm}
            %\begin{itemize}
                %\item[]	\textbf{derivation}
                        %\begin{footnotesize}
                    %\begin{eqnarray*}
                        %Y(\jom)	&=& \int_{-\infty}^{\infty} {y(t) \e^{-\jom t}\, dt} = \int_{-\infty}^{\infty} {\left(\int_{-\infty}^{\infty} {h(\tau) \cdot x(t-\tau)\, d\tau}\right) \e^{-\jom t}\, dt}\nonumber\\
                                %\pause
                                    %&=& \int_{-\infty}^{\infty} {h(\tau) \int_{-\infty}^{\infty} {x(t-\tau)} \e^{-\jom t}\, dt\, d\tau}\nonumber\\
                                %\pause
                                    %&=& \int_{-\infty}^{\infty} {h(\tau)  \e^{-\jom \tau} \underbrace{\int_{-\infty}^{\infty} {x(t-\tau)} \e^{-\jom (t-\tau)}\, d(t-\tau)}_{X(\jom)}\, d\tau}\nonumber\\
                                %\pause
                                    %&=& \int_{-\infty}^{\infty} {h(\tau) \e^{-\jom \tau}\, d\tau} \cdot X(\jom)\nonumber\\
                                %\pause
                                    %&=& H(\jom) \cdot X(\jom)\label{eq:mult_conv} 
                    %\end{eqnarray*}
                        %\end{footnotesize}
            %\end{itemize}
        \end{frame}	

        \begin{frame}{fourier transform}{property 4: Parseval's theorem}
            \begin{equation*}
                \int\limits_{-\infty}^{\infty}{x^2(t)\, dt} = \frac{1}{2\pi}\int\limits_{-\infty}^{\infty} {\left|X(\jom)\right|^2\, d\omega} 
            \end{equation*}
            
            \bigskip
            \only<2->{\textbf{note}: }
            \begin{itemize}
                \item<2->   energy of the signal is preserved when switching between time and frequency domains
            \end{itemize}
            %\pause
            %\begin{itemize}
                %\item[]	\textbf{derivation}
                %\begin{footnotesize}
                    %\begin{equation*}
                        %\int_{-\infty}^{\infty}{h(\tau)\cdot x(t-\tau)\, d\tau} = \frac{1}{2\pi}\int_{-\infty}^{\infty} {H(\jom)\cdot X(\jom) \e^{\jom t} d\omega}\nonumber
                    %\end{equation*}
                    %
                    %\centering with $H(\jom) \longrightarrow X^\ast (\jom)$ and $h(\tau)\longrightarrow x(-\tau)$, $t = 0$
                    %
                    %\pause
                    %\begin{eqnarray*}
                        %\int_{-\infty}^{\infty}{x(-\tau)\cdot x(-\tau)\, d\tau} &=& \frac{1}{2\pi}\int_{-\infty}^{\infty} {X^\ast (\jom)\cdot X(\jom) \, d\omega}\nonumber\\
                        %\int_{-\infty}^{\infty}{x^2(t)\, dt} &=& \frac{1}{2\pi}\int_{-\infty}^{\infty} {\left|X(\jom)\right|^2\, d\omega} \nonumber
                    %\end{eqnarray*}
                %\end{footnotesize}
            %\end{itemize}
        \end{frame}	

        \begin{frame}{fourier transform}{property 5: time \& frequency shift}
            \begin{itemize}
                \item \textbf{time shift}
                \begin{equation*}\label{eq:fft_timeshift}
                    x(t-t_0) \mapsto X(\jom)\e^{-\jom t_0} 
                \end{equation*} 
                %\pause
                %\begin{itemize}
                    %\item[]	\textit{derivation}
                    %\begin{footnotesize}
                        %\begin{eqnarray*}
                            %\int\limits_{-\infty}^{\infty} {x(t-t_0) \e^{-\jom t}\, dt} &=& \int\limits_{-\infty}^{\infty} {x(\tau) \e^{-\jom (\tau + t_0)}\, d\tau}\nonumber\\
                            %\pause
                            %&=& \e^{-\jom t_0}\underbrace{\int\limits_{-\infty}^{\infty} {x(\tau) \e^{-\jom \tau}\, d\tau}}_{X(\jom)}\nonumber\\
                        %\end{eqnarray*}
                    %\end{footnotesize}
                %\end{itemize}
                %\pause
                \item \textbf{frequency shift}
                \begin{equation*}
                            \frac{1}{2\pi}\int\limits_{-\infty}^{\infty} X(\jom-\omega_0) \e^{\jom t}\, d\omega = \e^{\jom_0 t}\cdot x(t) 		
                \end{equation*} 
            
            \end{itemize}
            
            \bigskip
            \only<2->{\textbf{note}: }
            \begin{itemize}
                \item<2->   time shift results in phase shift
                \item<2->   frequency shift results in modulation of time signal
            \end{itemize}

        \end{frame}	

        \begin{frame}{fourier transform}{property 6: symmetry}
            \begin{eqnarray*}
                |X(\jom)| &=& |X(-\jom)|\\
                \Phi_\mathrm{X}(\omega) &=& -\Phi_\mathrm{X}(-\omega) 
            \end{eqnarray*}
            
            \bigskip
            \only<2->{\textbf{note}: }
            \begin{itemize}
                \item<2->   spectrum of (real) signal is conjugate complex
                    \begin{itemize}
                        \item   magnitude spectrum is symmetric to ordinate
                        \item   phase spectrum is symmetric to origin
                    \end{itemize}
                \item<2->   even signals have no imaginary spectrum
                \item<2->   odd signals have no real spectrum
            \end{itemize}
            %\pause
            %\vspace{-5mm}
            %\begin{itemize}
                %\item[]	\textbf{derivation}
                %
                %\begin{footnotesize}
                %\item model time signal as sum of even and odd component $x_e(t), x_o(t)$:
                    %\begin{equation*}
                        %x(t) = \underbrace{\frac{1}{2}(x(t) + x(-t))}_{x_e(t)} + \underbrace{\frac{1}{2}(x(t) - x(-t))}_{x_o(t)} 
                    %\end{equation*}
                    %\only<3-4>{
                    %\begin{equation*}
                        %X_e(\jom) = \int\limits_{-\infty}^{\infty}{x_e(t)\cos(\omega t)\,dt} - \mathrm{j} \underbrace{\int\limits_{-\infty}^{\infty}{x_e(t)\sin(\omega t)\,dt}}_{= 0}\nonumber
                    %\end{equation*}
                    %\pause
                    %%\vspace{-5mm}
                    %\begin{itemize}
                        %\item[$\Rightarrow$]	$X_e(\jom)$ is \textit{real}
                        %\item[$\Rightarrow$]	$X_e(\jom) = X_e(-\jom)$ (substitute $x(t)$ with $x(-t)$)
                    %\end{itemize}
                    %}
                    %\only<5-6>{
                    %\begin{equation*}
                        %X_o(\jom) = \underbrace{\int\limits_{-\infty}^{\infty}{x_o(t)\cos(\omega t)\,dt}}_{=0} - \mathrm{j} \int\limits_{-\infty}^{\infty}{x_o(t)\sin(\omega t)\,dt} \nonumber
                    %\end{equation*}
                    %\pause
                    %%\vspace{-5mm}
                    %\begin{itemize}
                        %\item[$\Rightarrow$]	$X_o(\jom)$ is \textit{imaginary}
                        %\item[$\Rightarrow$]	$X_o(\jom) = -X_o(-\jom)$ (substitute $x(t)$ with $-x(-t)$)
                    %\end{itemize}
                    %}
                %\end{footnotesize}
                %\vspace{500mm}
            %\end{itemize}
        \end{frame}	


        \begin{frame}{fourier transform}{property 7: time \& frequency scaling}
            \vspace{-5mm}
            \begin{eqnarray*}
                y(t) &=& x(c\cdot t)\\ \mapsto&&\\ Y(\jom) &=& \frac{1}{|c|}X\left(j\frac{\omega}{c}\right) 
            \end{eqnarray*}
            
            \bigskip
            \only<2->{\textbf{note}: }
            \begin{itemize}
                \item<2->   scaling of abscissa in one domain leads to inverse scaling in the other domain
            \end{itemize}
            %\pause
            %\begin{itemize}
                %\item[]	\textbf{derivation} (positive $c$)
                %\begin{footnotesize}
                    %\begin{eqnarray*}
                        %Y(\jom) &=& \int\limits_{-\infty}^{\infty} {x(c\cdot t) \e^{-\jom t}\, dt}\nonumber\\
                        %\pause
                        %&=& \int\limits_{-\infty}^{\infty} {x(\tau) \e^{-\jom \frac{\tau}{c}}\, d\frac{\tau}{c}}\nonumber\\
                        %\pause
                        %&=& \frac{1}{c}\int\limits_{-\infty}^{\infty} {x(\tau) \e^{-\mathrm{j} \frac{\omega}{c} \tau}\, d\tau}\nonumber\\
                        %\pause
                        %&=& \frac{1}{c} X\left(\mathrm{j}\frac{\omega}{c}\right) \nonumber
                    %\end{eqnarray*}
                %\end{footnotesize}
            %\end{itemize}
        \end{frame}	
    %%\section{examples}
        %%\begin{frame}{fourier transform}{examples}
                %%\begin{itemize}
                    %%\item	cosine
                    %%\item	delta function
                    %%\item	constant
                    %%\item	rectangular window
                    %%\item	delta pulse
                %%\end{itemize}
        %%\end{frame}	
%
    \section[sampled]{fourier transform of sampled time signals}
        \begin{frame}{fourier transform}{sampled time signals 1/2}
            \begin{itemize}
                \item   sampled time signal can be modeled as multiplication of original signal with delta pulse $\delta_\mathrm{T}(t)$
                \item   multiplication in time domain $\mapsto$ convolution in frequency domain
            \end{itemize}
            \begin{eqnarray*}\label{eq:ft_sampled}
                \mathfrak{F}[x(i)] 	&=& \mathfrak{F}[x(t)\cdot \delta_\mathrm{T}(t)]\nonumber\\
                \pause
                                    &=& \mathfrak{F}[x(t)]\ast \mathfrak{F}[\delta_\mathrm{T}(t)]\nonumber\\
                                    &=& X(\jom)\ast \Delta_\mathrm{T}(\jom) 
            \end{eqnarray*}
            \pause
            \textbf{note}
            \begin{itemize}
                \item   even if time domain signal is discrete, its Fourier transform is \textit{still continuous}
                \item   spectrum is \textit{repeated periodically}
            \end{itemize}
            
        \end{frame}	

        \begin{frame}{fourier transform}{sampled time signals 2/2}
            \includeanimation{Sampling}{01}{48}{10}
        \end{frame}	

    \section[STFT]{Short Time Fourier Transform}
        \begin{frame}{fourier transform}{STFT 1/2}
            short time Fourier transform (STFT):\\ Fourier transform of a short time segment

            \bigskip	
                \begin{itemize}
                    \item<2->   \textbf{reasons}:
                        \begin{itemize}
                            \item	remember block based processing
                            \item	segment can be seen as quasi-periodic or stationary
                        \end{itemize}
                    \smallskip
                    \item <3->  \textbf{implementation}:
                        \begin{itemize}
                            \item   pretend signal is $0$ outside of the segment
                            \item[$\Rightarrow$] multiplication of signal and \textit{window function}
                        \end{itemize}
                \end{itemize}
        \end{frame}	
        \begin{frame}{fourier transform}{STFT 2/2}
            \includeanimation{Stft}{001}{005}{1}
            \inserticon{video}
        \end{frame}	

        \begin{frame}{fourier transform}{STFT: window functions}
            \begin{itemize}
                \item   time domain multiplication $\mapsto$ frequency domain convolution 
                \item   time domain shape determines frequency domain shape of the window
            \end{itemize}
            \only<2>{
            %\vspace{-3mm}
            \figwithmatlab{SpectralWindows}
            }
            \only<3>{
            \bigskip
            \textbf{spectral leakage characterization}
            \begin{itemize}
                \item	main lobe width
                \item	side lobe height
                \item	side lobe attenuation
            \end{itemize}
            }
            \vspace{70mm}
        \end{frame}	

    \section[DFT]{Discrete Fourier Transform}
        \begin{frame}{fourier transform}{DFT}
            digital domain: requires discrete frequency values:
            
            $\Rightarrow$ discrete Fourier transform
            \begin{equation*}\label{eq:dft}
                X(k) = \sum\limits_{i=0}^{\mathcal{K}-1}{x(i)\e^{-\mathrm{j}ki\frac{2\pi}{\mathcal{K}}}}
            \end{equation*}
            with
            \begin{equation*}
                \Delta\Omega = \frac{2\pi}{\mathcal{K}T_{\mathrm{S}}} = \frac{2\pi f_{\mathrm{S}}}{\mathcal{K}}
            \end{equation*}
            
            \pause
            2 interpretations:
            \begin{itemize}
                \item	sampled continuous Fourier transform
                \item	continuous Fourier transform of periodically extended time domain segment
            \end{itemize}
        \end{frame}	

        \begin{frame}{fourier transform}{spectrogram}
            \vspace{-2mm}
            \begin{itemize}
                \item   spectrogram allows to visualize temporal changes in the spectrum
                \item   displays the \textit{magnitude spectrum }only
            \end{itemize}
            \vspace{-8mm}
            \begin{columns}
                \column{.9\linewidth}
                    \figwithmatlab{Specgram}
                \column{.1\linewidth}
                    \begin{figure}
                    \includeaudio{sax_example}
                    \end{figure}
            \end{columns}
            
        \end{frame}	

    \section{summary}
        \begin{frame}{summary}{lecture content}
            \begin{itemize}
                \item   \textbf{Fourier Transform (of a real signal)}
                    \begin{itemize}
                        \item   is conjugate complex
                        \item   often represented as magnitude $+$ phase
                    %\end{itemize}
                %\item    \textbf{FT properties}
                    %\begin{itemize}
                        \item   invertible
                        \item   linear
                        \item   convolution in time domain is multiplication in frequency domain
                        \item   energy preserving
                        \item   time shift result in phase shift, frequency shift results in amplitude modulation
                        \item   symmetric
                        \item   time scaling result in inverse frequency scaling
                    \end{itemize}
                \item   \textbf{FT of sampled signals}:
                    \begin{itemize}
                        \item   is periodic with sample rate
                    \end{itemize}
                \item   \textbf{STFT}
                    \begin{itemize}
                        \item   window results in spectral leakage (convolution in freq domain)
                    \end{itemize}
                \item   \textbf{DFT}
                    \begin{itemize}
                        \item   discrete in both time and freq domain
                    \end{itemize}
            \end{itemize}
            \inserticon{summary}
        \end{frame}
\end{document}
