% move all configuration stuff into one file so we can focus on the content
\documentclass[hyperref={pdfpagelabels=false,colorlinks=true,linkcolor=white,urlcolor=blue}]{beamer}

%%%%%%%%%%%%%%%%%%%%%%%%%%%%%%%%%%%%%%%%%%%%%%%%%%%%%%%%%%%%%%%%%%%%%%%%%%%%%%%%%%
%%%%%%%%%%%%%%%%%%%%%%%%%%%%%%%%%%%%%%%%%%%%%%%%%%%%%%%%%%%%%%%%%%%%%%%%%%%%%%%%%%
% packages
\usepackage{pict2e}
\usepackage{epic}
\usepackage{amsmath,amsfonts,amssymb}
\usepackage{units}
\usepackage{fancybox}
\usepackage[absolute,overlay]{textpos} 
\usepackage{media9} % avi2flv: "C:\Program Files\ffmpeg\bin\ffmpeg.exe" -i TuneFreqFilterbank.avi -b 600k -s 441x324 -r 15 -acodec copy TuneFreqFilterbank.flv
\usepackage{silence}
\usepackage[backend=bibtex,style=ieee]{biblatex}
\WarningFilter{biblatex}{Patching footnotes failed}
\AtEveryCitekey{\iffootnote{\tiny}{}}
\addbibresource{references}

%%%%%%%%%%%%%%%%%%%%%%%%%%%%%%%%%%%%%%%%%%%%%%%%%%%%%%%%%%%%%%%%%%%%%%%%%%%%%%%%%%
%%%%%%%%%%%%%%%%%%%%%%%%%%%%%%%%%%%%%%%%%%%%%%%%%%%%%%%%%%%%%%%%%%%%%%%%%%%%%%%%%%
% relative paths
\graphicspath{{graph/}}

%%%%%%%%%%%%%%%%%%%%%%%%%%%%%%%%%%%%%%%%%%%%%%%%%%%%%%%%%%%%%%%%%%%%%%%%%%%%%%%%%%
%%%%%%%%%%%%%%%%%%%%%%%%%%%%%%%%%%%%%%%%%%%%%%%%%%%%%%%%%%%%%%%%%%%%%%%%%%%%%%%%%%
% colors
\definecolor{gtgold}{HTML}{E0AA0F} %{rgb}{0.88,0.66,1,0.06} [234, 170, 0]/256

%%%%%%%%%%%%%%%%%%%%%%%%%%%%%%%%%%%%%%%%%%%%%%%%%%%%%%%%%%%%%%%%%%%%%%%%%%%%%%%%%%
%%%%%%%%%%%%%%%%%%%%%%%%%%%%%%%%%%%%%%%%%%%%%%%%%%%%%%%%%%%%%%%%%%%%%%%%%%%%%%%%%%
% math
\DeclareMathOperator*{\argmax}{argmax}
\DeclareMathOperator*{\argmin}{argmin}
\DeclareMathOperator*{\atan}{atan}
\DeclareMathOperator*{\arcsinh}{arcsinh}
\DeclareMathOperator*{\sign}{sign}
\DeclareMathOperator*{\tcdf}{tcdf}
\DeclareMathOperator*{\si}{sinc}
\DeclareMathOperator*{\princarg}{princarg}
\DeclareMathOperator*{\arccosh}{arccosh}
\DeclareMathOperator*{\hwr}{HWR}
\DeclareMathOperator*{\flip}{flip}
\DeclareMathOperator*{\sinc}{sinc}
\newcommand{\e}{{e}}
\newcommand{\jom}{\mathrm{j}\omega}
\newcommand{\jOm}{\mathrm{j}\Omega}
\newcommand   {\mat}[1]    		{\boldsymbol{\uppercase{#1}}}		%bold
\renewcommand {\vec}[1]    		{\boldsymbol{\lowercase{#1}}}		%bold

%%%%%%%%%%%%%%%%%%%%%%%%%%%%%%%%%%%%%%%%%%%%%%%%%%%%%%%%%%%%%%%%%%%%%%%%%%%%%%%%%%
%%%%%%%%%%%%%%%%%%%%%%%%%%%%%%%%%%%%%%%%%%%%%%%%%%%%%%%%%%%%%%%%%%%%%%%%%%%%%%%%%%
% media9
\newcommand{\includeaudio}[1]{{\includemedia[
                        addresource=audio/#1.mp3,
                        width=5mm,
                        height=5mm,
                        activate=onclick,
                        flashvars={
                            source=audio/#1.mp3  
                            &autoPlay=true
                        }]
                        {\includegraphics[width=5mm, height=5mm]{SpeakerIcon}}
                        {APlayer.swf}}}
\newcommand{\audioautoplay}[1]{{\begin{center}\includemedia[
                            addresource=audio/#1.mp3,
                            width=.1\linewidth,
                            height=.01\linewidth,
                            activate=pageopen,
                            flashvars={
                                source=audio/#1.mp3  
                                &autoPlay=true
                            }]
                            {}
                            {APlayer.swf}\end{center}}}

\newcommand{\includevideo}[1]{{\begin{center}\includemedia[
                        addresource=video/#1.mp4,
                        width=0.8\linewidth,
                        height=0.4\linewidth,
                        activate=onclick,
                        flashvars={
                            source=video/#1.mp4  
                            &autoPlay=true
                        }]
                        {}
                        {VPlayer.swf}\end{center}}}
\newcommand{\videowithmatlab}[1]{{\begin{center}\includemedia[
                        addresource=video/animate#1.mp4,
                        width=0.8\linewidth,
                        height=0.4\linewidth,
                        activate=onclick,
                        flashvars={
                            source=video/animate#1.mp4  
                            &autoPlay=true
                        }]
                        {}
                        {VPlayer.swf}\end{center}\addreference{matlab source: matlab/animate#1.m}}}
                        

%%%%%%%%%%%%%%%%%%%%%%%%%%%%%%%%%%%%%%%%%%%%%%%%%%%%%%%%%%%%%%%%%%%%%%%%%%%%%%%%%%
%%%%%%%%%%%%%%%%%%%%%%%%%%%%%%%%%%%%%%%%%%%%%%%%%%%%%%%%%%%%%%%%%%%%%%%%%%%%%%%%%%
% other commands
\newcommand{\question}[1]{\vspace{-4mm}
                          \setbeamercovered{invisible}
                          \begin{columns}
                            \column{.7\textwidth}
                                \textbf{#1}\\
                            \column{.3\textwidth}
                                \begin{flushright}
                                     \includegraphics[scale=.5]{question_mark}
                                \end{flushright}
                                \vspace{6mm}
                          \end{columns}\pause\vspace{-6mm}}

\newcommand{\toremember}[1]{\vspace{-4mm}
                          \begin{columns}
                            \column{.7\textwidth}
                                \textbf{#1}\\
                            \column{.3\textwidth}
                                \begin{flushright}
                                     \includegraphics[scale=.5]{exclamation_mark}
                                \end{flushright}
                                \vspace{6mm}
                          \end{columns}\vspace{-6mm}}

\newcommand{\matlabexercise}[1]{\vspace{-4mm}
                          \setbeamercovered{invisible}
                          \begin{columns}
                            \column{.7\textwidth}
                                \textbf{matlab exercise}: #1
                            \column{.3\textwidth}
                                \begin{flushright}
                                     \includegraphics[scale=.5]{logo_matlab}
                                \end{flushright}
                                \vspace{6mm}
                          \end{columns}}

\newcommand{\addreference}[1]{                    
                    \begin{textblock*}{\baselineskip }(1.15\textwidth,.5\textheight)
                        \rotatebox{90}{\tiny #1}
                    \end{textblock*}}
                    
\newcommand{\figwithmatlab}[1]{
                    \begin{figure}
                        \centering
                        \includegraphics{#1}
                        \label{fig:#1}
                    \end{figure}
                    
                    \addreference{matlab source: matlab/display#1.m}}                    
\newcommand{\figwithref}[2]{
                    \begin{figure}
                        \centering
                        \includegraphics{#1}
                        \label{fig:#1}
                    \end{figure}
                    
                    \addreference{#2}}                    

%%%%%%%%%%%%%%%%%%%%%%%%%%%%%%%%%%%%%%%%%%%%%%%%%%%%%%%%%%%%%%%%%%%%%%%%%%%%%%%%%%
%%%%%%%%%%%%%%%%%%%%%%%%%%%%%%%%%%%%%%%%%%%%%%%%%%%%%%%%%%%%%%%%%%%%%%%%%%%%%%%%%%
% units
\setlength{\unitlength}{1mm}

%%%%%%%%%%%%%%%%%%%%%%%%%%%%%%%%%%%%%%%%%%%%%%%%%%%%%%%%%%%%%%%%%%%%%%%%%%%%%%%%%%
%%%%%%%%%%%%%%%%%%%%%%%%%%%%%%%%%%%%%%%%%%%%%%%%%%%%%%%%%%%%%%%%%%%%%%%%%%%%%%%%%%
% counters
\newcounter{i}
\newcounter{iXOffset}
\newcounter{iYOffset}
\newcounter{iXBlockSize}
\newcounter{iYBlockSize}
\newcounter{iYBlockSizeDiv2}
\newcounter{iDistance}


%%%%%%%%%%%%%%%%%%%%%%%%%%%%%%%%%%%%%%%%%%%%%%%%%%%%%%%%%%%%%%%%%%%%%%%%%%%%%%%%%%
%%%%%%%%%%%%%%%%%%%%%%%%%%%%%%%%%%%%%%%%%%%%%%%%%%%%%%%%%%%%%%%%%%%%%%%%%%%%%%%%%%
% theme & layout
\usetheme{Frankfurt}
\beamertemplatenavigationsymbolsempty
%\setbeamertemplate{frametitle}[smoothbars theme]
\setbeamertemplate{frametitle}
{
    \begin{beamercolorbox}[ht=1.8em,wd=\paperwidth]{frametitle}
        \vspace{-.1em}%
        \hspace{.2em}{\strut\insertframetitle\strut}
        
        \hspace{.2em}\small\strut\insertframesubtitle\strut
        %\hfill
        %\includegraphics[height=.8cm,keepaspectratio]{CenterMusicTechnology-solid-2lines-white-CoAtag}
        
    \end{beamercolorbox}
    \begin{textblock*}{100mm}(8.3cm,.65cm)
        \includegraphics[height=.8cm,keepaspectratio]{CenterMusicTechnology-solid-2lines-white-CoAtag}
    \end{textblock*}
}

% set this to ensure bulletpoints without subsections
\usepackage{remreset}
\makeatletter
\@removefromreset{subsection}{section}
\makeatother
\setcounter{subsection}{1}

%---------------------------------------------------------------------------------
% appearance
\setbeamercolor{structure}{fg=gtgold}
\setbeamercovered{transparent} %invisible
\setbeamercolor{bibliography entry author}{fg=black}
\setbeamercolor*{bibliography entry title}{fg=black}
\setbeamercolor*{bibliography entry note}{fg=black}
%---------------------------------------------------------------------------------
% fontsize
\let\Tiny=\tiny

%%%%%%%%%%%%%%%%%%%%%%%%%%%%%%%%%%%%%%%%%%%%%%%%%%%%%%%%%%%%%%%%%%%%%%%%%%%%%%%%%%
%%%%%%%%%%%%%%%%%%%%%%%%%%%%%%%%%%%%%%%%%%%%%%%%%%%%%%%%%%%%%%%%%%%%%%%%%%%%%%%%%%
% warnings
\pdfsuppresswarningpagegroup=1

%%%%%%%%%%%%%%%%%%%%%%%%%%%%%%%%%%%%%%%%%%%%%%%%%%%%%%%%%%%%%%%%%%%%%%%%%%%%%%%%%%
%%%%%%%%%%%%%%%%%%%%%%%%%%%%%%%%%%%%%%%%%%%%%%%%%%%%%%%%%%%%%%%%%%%%%%%%%%%%%%%%%%
% title information
\title[]{MUSI-6201~---~Computational Music Analysis}   
\author[alexander lerch]{alexander lerch} 
%\institute{~}
%\date[Alexander Lerch]{}
\titlegraphic{\vspace{-16mm}\includegraphics[scale=.25]{title}}


\subtitle{Part 12.1: Music Structure Analysis}

%%%%%%%%%%%%%%%%%%%%%%%%%%%%%%%%%%%%%%%%%%%%%%%%%%%%%%%%%%%%%%%%%%%%%%%%%%%%
\begin{document}
    % generate title page
	

\begin{frame}
    \titlepage
    %\vspace{-5mm}
    \begin{flushright}
        \href{http://www.gtcmt.gatech.edu}{\includegraphics[height=.8cm,keepaspectratio]{logo_GTCMT_black}}
    \end{flushright}
\end{frame}


    \section[overview]{lecture overview}
        \begin{frame}{temporal analysis}{overview}
            \begin{itemize}
                \item   \textbf{reading}  
                    \begin{itemize}
                        \item   \href{http://www.ismir2010.ismir.net/proceedings/ismir2010-107.pdf}{\underline{\textit{Audio-Based Music Structure Analysis}}}\footfullcite{paulus_audio-based_2010}
                    \end{itemize}
                \item   \textbf{sources}: slides (latex) \& Matlab  
                    \begin{itemize}
                        \item   \href{https://github.com/alexanderlerch/ACA-Slides}{\underline{github repository}}
                    \end{itemize}
                \bigskip
                \item<2->   \textbf{lecture content}
                    \begin{itemize}
                        \item<2->   features describing structural properties
                        \item<3->   general approaches for extracting structure from music
                        \item<4->   challenges in ground truth generation and evaluation
                    \end{itemize}
            \end{itemize}
        \end{frame}

    \section[intro]{introduction}
        \begin{frame}{music structure}{introduction}
            \begin{itemize}
                \item   \textbf{music is inherently formal}/organized/structural
                \smallskip
                \item<2->   various \textbf{hierarchical structural levels}
                    \begin{itemize}
                        \item   \textit{groups of notes} build rhythmic/melodic/harmonic patterns
                        \item   \textit{measures} group multiple events
                        \item   \textit{phrases} group several measures
                        \item   \textit{sections} contain several phrases
                        \item   several sections can comprise \textit{piece/movement}
                        \item   \ldots
                    \end{itemize}
                \smallskip
                \item<3->    \textbf{grouping} of musical elements/patterns is influenced by
                    \begin{enumerate}
                        \item   \textit{contrasts}
                            \begin{itemize}
                                \item   rhythmic, harmonic, melodic patterns
                            \end{itemize}
                        \item   \textit{similarity and repetitions}
                            \begin{itemize}
                                \item   rhythmic, harmonic, melodic patterns
                            \end{itemize}
                        \item   \textit{homogeneity} within a section 
                            \begin{itemize}
                                \item   instrumentation, tempo, harmony
                            \end{itemize}
                    \end{enumerate}
            \end{itemize}
        \end{frame}
        \begin{frame}{music structure analysis}{introduction}
            \begin{itemize}
                \item   \textbf{objective}
                    \begin{itemize}
                        \item   reveal structural properties and relationships
                        \item   generate a list of parts and repetitions
                    \end{itemize}
                \bigskip
                \item   typical \textbf{processing steps}
                    \begin{enumerate}
                        \item   feature extraction
                        \item   Self Distance Matrix (SDM) or Self Similarity Matrix (SSM)
                        \item   detect segments
                            \begin{itemize}
                                \item novelty
                                \item homogeneity
                                \item repetition
                            \end{itemize}
                    \end{enumerate}
            \end{itemize}
        \end{frame}
    \section[features]{features for structural analysis}
        \begin{frame}{music structure analysis}{features 1/2}
            \begin{itemize}
                \item   features from \textbf{all categories} can have impact on structure
                    \begin{itemize}
                        \item   timbre
                            \begin{itemize}
                                \item   instrumentation, playing technique, effects, \ldots
                            \end{itemize}
                        \item   tonal content
                            \begin{itemize}
                                \item   melodic and harmonic patterns, range, \ldots
                            \end{itemize}
                        \item   rhythm content
                            \begin{itemize}
                                \item   tempo, rhythmic patterns, \ldots
                            \end{itemize}
                        \item   dynamics
                            \begin{itemize}
                                \item   loudness, range, \ldots
                            \end{itemize}
                    \end{itemize}
               \bigskip
               \item<2->    \textbf{feature aggregation}
                \begin{itemize}
                    \item   use texture window, or
                    \item   aggregate features per beat or downbeat
                \end{itemize}
            \end{itemize}
        \end{frame}
        \begin{frame}{music structure analysis}{features 2/2}
            \only<1>{\figwithref{StructureFeatures_0}{matlab source: matlab/displayStructureFeatures.m}}
            \only<2>{\figwithref{StructureFeatures_1}{matlab source: matlab/displayStructureFeatures.m}}
            
        \end{frame}
        \begin{frame}{music structure analysis}{distance matrix}
            \only<1>{\figwithref{Sdm_0}{matlab source: matlab/displaySdm.m}}
            \only<2>{\figwithref{Sdm_1}{matlab source: matlab/displaySdm.m}}
            \only<3>{\figwithref{Sdm_2}{matlab source: matlab/displaySdm.m}}
            \only<4>{\figwithref{Sdm_3}{matlab source: matlab/displaySdm.m}}
            \only<5>{\figwithref{Sdm_4}{matlab source: matlab/displaySdm.m}}
        \end{frame}
        \begin{frame}{music structure analysis}{similarity matrix}
            \only<1>{\figwithref{Ssm_0}{matlab source: matlab/displaySsm.m}}
            \only<2>{\figwithref{Ssm_1}{matlab source: matlab/displaySsm.m}}
            \only<3>{\figwithref{Ssm_2}{matlab source: matlab/displaySsm.m}}
            \only<4>{\figwithref{Ssm_3}{matlab source: matlab/displaySsm.m}}
            \only<5>{\figwithref{Ssm_4}{matlab source: matlab/displaySsm.m}}
        \end{frame}
    \section[novelty]{novelty analysis}
        \begin{frame}{music structure analysis}{novelty analysis}
            \only<1>{\figwithref{SsmNovelty_0}{matlab source: matlab/displaySsmNovelty.m}}
            \only<2>{example: Gaussian checker board kernel \figwithref{SsmNovelty_1}{matlab source: matlab/displaySsmNovelty.m}}
            \only<3>{\figwithref{SsmNovelty_2}{matlab source: matlab/displaySsmNovelty.m}}
            \only<4>{\figwithref{SsmNovelty_3}{matlab source: matlab/displaySsmNovelty.m}}
        \end{frame}
    \section[homogeneity]{homogeneity analysis}
        \begin{frame}{music structure analysis}{homogeneity analysis 1/2}
            \only<1>{\figwithref{SsmHomogeneity_0}{matlab source: matlab/displaySsmHomogeneity.m}}
            \only<2>{oversimplified way: MA kernel\figwithref{SsmHomogeneity_1}{matlab source: matlab/displaySsmHomogeneity.m}}
            \only<3>{\figwithref{SsmHomogeneity_2}{matlab source: matlab/displaySsmHomogeneity.m}}
        \end{frame}
        \begin{frame}{music structure analysis}{homogeneity analysis 2/2}
            \begin{itemize}
                \item   also used as post-processing step after novelty-based approach, e.g.
                    \begin{enumerate}
                        \item   describe each segment with features
                        \item   cluster and see which segments are grouped together
                    \end{enumerate}
            \end{itemize}
        \end{frame}
    \section[repetition]{repetition analysis}
        \begin{frame}{music structure analysis}{repetition analysis 1/2}
            \only<1>{\figwithref{Sdlm_0}{matlab source: matlab/displaySdlm.m}}
            \only<2>{\figwithref{Sdlm_1}{matlab source: matlab/displaySdlm.m}}
            \only<3>{\figwithref{Sdlm_2}{matlab source: matlab/displaySdlm.m}}
        \end{frame}
        \begin{frame}{music structure analysis}{repetition analysis 2/2}
            \begin{itemize}
                \item   while in many cases it 'looks' easy, automatic extraction is \textbf{error-prone}
                \bigskip
                \item<2->[$\Rightarrow$]   typical approaches for \textbf{enhancing} the distance/similarity/lag matrix
                    \begin{itemize}
                        \item   filtering (low pass smoothing, high pass edge detecting)
                        \item   use matrices with different time resolutions
                        \item   image processing methods (e.g., erosion \& dilation)
                        \item   thresholding
                        \item   ``path search'' through probability matrix
                    \end{itemize}
            \end{itemize}
        \end{frame}
    \section[evaluation]{evaluation}
        \begin{frame}{music structure analysis}{evaluation}
            \begin{itemize}
                \item   evaluation of structure detection \textbf{challenging}
                    \begin{itemize}
                        \item   \textit{ground truth}
                            \begin{itemize}
                                \item   structure itself may be ambiguous
                                \item   depending on annotator, varying hierarchical level of labels, e.g.
                            \end{itemize}
                    \end{itemize}
                    \begin{table}
                        \centering
                        \footnotesize
                            \begin{tabular}{l|c|c|c|c|c|c|c|c|c|c|}
                                    \hline
                                  \textbf{ann 1} & intro & \multicolumn{4}{c|}{A} & \multicolumn{4}{c|}{A} & outro\\ \hline
                                  \textbf{ann 2} & intro & \multicolumn{2}{c|}{verse} & \multicolumn{2}{c|}{chorus} & \multicolumn{2}{c|}{verse} & \multicolumn{2}{c|}{chorus} & outro\\ \hline
                                  \textbf{ann 3} & intro & V$_1$ & V$_2$ & C$_1$ &C$_2$ &  V$_1$ & V$_2$ & C$_1$ &C$_2$ & outro\\
                                    \hline
                            \end{tabular}
                    \end{table}
                \bigskip
                \item<2->   \textit{method and metric}
                    \begin{itemize}
                        \item   frame level, e.g., pairwise match
                    \end{itemize}
                \bigskip
                \item<3->   typical range of results
                    \begin{itemize}
                        \item   $F = 50\ldots 70\% $
                    \end{itemize}
            \end{itemize}
        \end{frame}
       
    \section[summary]{lecture summary}
        \begin{frame}{summary}{lecture content}
            \begin{enumerate}
                \item       name typical features for structural analysis
                \smallskip
                \item<2->   how does the SSM change for drastically changing tempi
                \smallskip
                \item<3->   does the lag matrix have any advantages over the SSM or SDM
            \end{enumerate}
        \end{frame}
\end{document}

